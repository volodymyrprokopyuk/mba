\environment layout

%\usemodule[fnt-10]
%\ShowCompleteFont{name:fontawesome}{20pt}{1}

\starttext

\startfrontmatter

\midaligned{Escuela de Administración de Empresas}\blank[14*big]
\midaligned{Plan de empresa}\blank[2*big]
\midaligned{Volodymyr Prokopyuk}
\midaligned{volodymyrprokopyuk@gmail}
\vfill\midaligned{Madrid}
\midaligned{\currentdate}

\completecontent
\stopfrontmatter

\startbodymatter

\chapter{Resumen ejecutivo}%{Executive summary}

\Comment{Key points, clear, brief}

\section{Elevator speech}

\startitemize
\item What business you are in
\item What is your industry and target customer
\item What are your products and services (features)
\item What is your customer value and benefits
\item What sets your business apart (uniqueness)
\item What is your competition
\stopitemize

La empresa es una empresa de soluciónes IT para el negocio de un solo empleado
yo que trabaja en la industria de tecnología de información. El sector a que se
dedica la empresa es el sector de las soluciones IT para el negocio tanto el
desarrollo de software como los servicios de consultoria IT. Las situación y las
tendencias en la industria favorecen el crecimiento de la emrpesa. Las
tendencias favorambles en la industria son:
\startitemize
\item Avances tecnológicos en la virtualización de hardware permiten reducir los
costes de hardware de los servidores que utiliza la empresa
\item Crecimiento de la nube promueve las plataformas de bajo coste de despliege
de las soluciones IT que ofrece la empresa a sus clientes utilizando el modelo
B2B (Business to Business)
\item Desarrollo activo y utilización en producción de los sistemas de código
abierto que reducen los costes y mejoran la calidad de las soluciones IT de la
empresa
\item Expansión de los dispositivos móviles extiende el mercado potencial para
las soluciones de la empresa
\item Adopción industrial de HTML5 y CSS3 permite a la empresa utilizar la misma
tecnología para el desarrollo de las aplicaciones web y de las aplicaciones para
los dispositivos móviles aprovechando las sinergias de experiencia de desarrllo
de aplicaciones web para la creación de aplicaciones para los dispositivos
móviles
\item Utilización del modelo asíncrono de ejecución tanto en el servidor como en
el cliente unifica las prácticas de diñeno e implementación de las aplicaciones
y consolida los conocimientos necesarios para desarrllar los sistemas completos
de la arquitectura cliente/servidor
\stopitemize

El cliente objetivo para la empresa son las empresas medianas y pequeñas que
necesitan alinear sus sistemas de información con las tendencias y los cambios
que precisa el entrono para el crecimiento y el éxito. Los clientes de la
empresa habitualmente tienen las siguientes necesidades:
\startitemize
\item Implantación de sistemas IT que implementan la lógica de negocio
\item Creación de sitios Web y las aplicaciones Web que realizan la presencia
Web de la empresa
\item Publicación de aplicaciones para los dispositivos móviles que extienden la
presencias Web de la empresa
\item Integración de sistema IT de clientes con los terceros y los sistemas de
pago
\stopitemize

La empresa ofrece los siguientes productos y servicios IT para el negocio de sus
clientes:
\startitemize
\item Desarrollo e impalnación de las soluciones IT para el negocio de los
clientes
\item Los servicios de consultoria IT para la orientación y seleción de las
tecnologias IT más apropiadas para los clientes
\item Mantenimiento y modificación de las soluciones IT de los clientes
\stopitemize

Las ventajas clave para los clientes al confiar en la empresa son los
siguientes:
\startitemize
\item Soluciones IT y los servicios de consultoría IT de alta calidad lo cual
representa la conformidad total de las soluciones IT con los requerimientos del
cliente y su satisfacción alta
\item Soluciones IT de alta fiabilidad y disponibilidad que asegura la calidad y
la continuidad en el tiempo de los servicios que presta el cliente
\item Soluciones IT flexible que benefician al cliente con las posibilidad de
modificar la lógica de las soluciones de forma rápida y barata para adaptar las
soluciones a los cambios del entrono de negocio
\item Cortos plazos de entrega de la solucioens IT permiten al cliente realizar
rápido las nuevas ideas de negocio y llegar al mercado antes que al competencia
\item Relacion calidad/precio superior que la del competencia permite al cliente
ahorrar en los costes de las soluciones IT sin renunciar a las ventajas de un
sistema IT de calidad
\stopitemize

El valor único que aporta la empresa para los clientes consiste en ofrecer a los
clientes las soluciones IT flexibles de alta calidad, fiabilidad y
disponibilidad en cortos plazos de entrega por precios reducidos. Las soluciones
IT y los servicios de consultoria IT son únicos porque la empresa considera la
innovación y la mejora contínua como el motor de crecimiento de la compañía y la
clave para el éxito de sus clientes. El entender la necesidades de los clientes
y la máxima satisfacción de los clientes son para le empresa los valores más
importanes que guín todas sus actividades.

La competencia de la empresa son otras empresas del sector que ofrecen las
soluciones IT dentro del modelo B2B. La diferencia más importante que sitúa
a la empresa en una posición de ventaja en comparación con la competencia es el
compromiso fuerte de la empresa con la innovación y la relación calidad/precio
superior.

\section{Modelo de negocio}%{Business model}

El modelo de negocio consiste en vender las soluciones IT de calidad
personalizadas a los clientes y prestar los servicios de consultoria IT a los
clientes. Los ingresos principales son de las ventas de las soluciones IT y de
los servicios de consultoria IT. Los gastos comprenden el coste de hora de
diseño y de desarrollo de software, los gastos de electricidad y los gastos de
las comunicaciones e Internet. Yo asumo el coste de hora de diseño y de
desarrollo de software porque soy el empresario y en único empleado de la
empresa.

\section{Estrategia de crecimiento}%{Growth strategy}

La estrategia de crecimiento principal de la empresa es de creat productos
nuevos y conseguir los clientes nuevos. El enfoque de la empresa son las
soluciones IT de calidad personalizadas basadas en la innovación y la
creatividad. Las inversiones en la innovación y la mejora contínua resultan en
la creación de productos sustitutivos y el cambio de las fuerzas en el
mercado. La visión de la empresa es crear el valor para el cliente mediante la
innovación y la tecnologia.

\chapter{Breve descripcón de la empresa}%{Company overview}

\section{Idea de negocio}%{Business idea}

\Comment{Business model summary}

Custom-made quality IT solutions and consultancy

La idea de negocio consiste en vender las soluciones IT de calidad
personalizadas a los clientes y prestar los servicios de consultoria IT a los
clientes. Los ingresos principales son de las ventas de las soluciones IT y de
los servicios de consultoria IT. Los gastos comprenden el coste de hora de
diseño y de desarrollo de software, los gastos de electricidad y los gastos de
las comunicaciones e Internet. Yo asumo el coste de hora de diseño y de
desarrollo de software porque soy el empresario y en único empleado de la
empresa.

\section{Misión}%{Mission}

\Comment{Who you are, what you do}

Constant innovation in custom-made quality IT solutions and consultancy

\Paragraph{Innovación constante en soluciones IT personalizadas de calidad y IT
consultoria}

La empresa se crea con la intención de ser un lider en la aplicación de
soluciones innovadoras y creativas al diseño y desarrollo de las soluciones a
los clientes. La empresa pretende promover la innovación y mejora continua en
todos los niveles de su actividad desde la organización y gestión de los
procesos internos hasta las relaciónes con los clientes y elaboración del
producto final. La espiración de la empresa está en transmitir a los clientes
el espiritu de innovación y creatividad como los pilares principales en los que
se basa toda la actividad de la empresa y enfoque prinipal en busqueda de las
soluciones y la resolución de los problemas.

La empresa se construye como una emprese de servicios IT para los negocios de
distinto índole. La actividad de empresa principalmente va a estar dirigida a
dar las soluciones IT a otras empresas utilizando del modelo de interacción B2B
(Business To Business). La dos principales ramas de actividad son el diseño y el
desarrollo de productos IT y la consultoria IT. El diseño y el desarrollo de
productos IT comprende las aplicaciones que implementan la lógica de negocio de
los clientes, automiatizan sus procesos internos, los sistemas de integración
con terceros, sitios y aplicaciones Web y las aplicaciones para los móviles. La
consultoría IT se dedica a orientar y asesorar a los clientes para elegir la
mejor solución y diseñar el sistema más adecuado según las necesidades de cada
cliente en particular.

En todas las actividades de la empresa la innovación es el enfoque más
importante porque gracias a la innovación se pueden encontrar las soluciónes
óptimas y nuevas formas de afrontar los problemas. Cada solución echa por la
empresa reflejará el espiritu creativo e inovador en todas sus facetas. Es
porque cada proyecto va a ser personalizado con la seleción de métodos más
adequados en cada caso. Es importante que la innovación sea constante porque
el mundo de tecnología cambia muy rápido y para ofrecer a los clientes el
mejor servicios es necesario constantemente actualizarse.

La mejora continua es otro pilar de la actividad de empresa. Cada solución va a
ser fiable y va a satisfacer todas las necesidades de cliente. La satisfacción
del cliente es el mayor objetivo de la empresa. Las soluciones se van a
desarrollar en el marco de mejora contunua ya que solo las relaciones a largo
plazo aseguran el éxtito para la empresa y el desarrollo sostenible de la
empresa.

IT consultoría pretende orientar a los clientes en el mundo de tecnologia de
informaicón y sugerirles las óptimas soluciones y métodos de afrontar los
problemas que surgen y satisfacer las necesidades. Una vez elegido la solución
que mejor se adapta a las necesidades de los clientes se procedera a su
implementación e implantación en el cliente.

\section{Visión}%{Vision}

\Comment{Precise set of words that defines business highest aspiration you
hope to achieve, enduring purpose of the business}

Create client value through innovation and IT

\Paragraph{Crear el valor para el cliente mediante la innovación y la
tecnología}

La aspiración más alta de la empresa consiste en ser el líder en la aplicación
de la innovación y la creatividad en las soluciones tecnológicas para los
clientes. La innovación constante es lo que quia toda la actividad de la
empresa. El apostar por la innovación constante asegura el desarrollo sostenible
de la empresa y aumenta el valor aportado a los clientes. Las soluciones
basadas en la innovación y la creatividad aportan más valor a los clientes
sobre todo en el campo de la tecnología de información.

La visión de aportar valor a los clientes mediante la innovación y la tecnología
garantiza la sostenibilidad y interes futuro de los clientes en la empresa. La
visión de la empresa pretende construir las relaciones duraderas, las relaciones
de confianza y las relaciones a largo plazo con los clientes. La satisfacción de
los clientes el lo más importente en la actividad de la empresa. Las tecnologías
aplicadas en cada solución se cambiarán con el paso de tiempo pero el compromiso
con la innovación, la mejora continua y la satisfación del cliente perdurarán el
el tiempo garantizando en futuro de la empresa y el futuro de sus clientes.

\section{Valores}%{Values}

\Comment{Principles to guide your business activities, preferable mode of
conduct}

\startitemize
\item Client value creation
\item Commitment to customer satisfaction
\item Commitment to innovation and excellence
\item Commitment to quality and continuous improvement
\item Passion for what I do
\item Build open and honest relationships
\item Commitment to sustainability and respansibility
\stopitemize

\Paragraph{Creación de valor para el cliente}

El propósito principal de la actividad de empresa es la creación de valor para
el cliente. La forma más adecuada de crear el valor para el cliente es
satisfacer de la forma más complete sus necesidades. Para satisfacer las
necesidades del cliente hay que entregarle las soluciónes de calidad y
conformidad con los requerimiento del cliente, en los plazos mínimos por el
precio mínimo. Crear valor para el clieten significa gestionar los recursos
limitados para conseguir los objetivos. En el centro de la actividad de la
empresa está el cliente y su satisfacción. La empresa crea el valor para el
cliente madiante la entrega de la soluciónes IT o mediante la prestación de
los servicios de consultoría IT.

\Paragraph{Compromiso con la satisfacción del cliente}

Lo más importante para la empresa es asegurar el cresimiento y el desarrollo
sostenible y contínuo para el futuro lo cual es posible solo a través de la
construcción de las relaciónes duraderas, de confianza y a largo plazo con los
clientes. El criterio más importante en las relaciones entre la empresa y los
clientes es la satisfacción de los clientes. Es porque toda la activida de la
empresa está enfocada para conseguir este objetivo. El cliente se seinte
satisfecho cuando el recibie un trato adecuado, soluciones de calidad que
resuelven sus problemas, consultoria profesional. Es muy importante escuchar al
cliente y dar las soluciones a sus necesidades de forma eficaz y eficiente en
los plazos mínimos por el precio mínimo.

\Paragraph{Compromiso con la innovación y la excelencia}

Para crear valor para el cliente y satisfacer las necesidades del cliente hay
que emplear métodos eficacez y eficientes. En la empresa la innovacion se
considera como la opción más eficiente de satisfacer al cliente. La innovación
signifaca la aplicación de métodos nuemos y optimizados en las soluciones y
entonces métodos más eficientes que dan mejor resultado en menor tiempo por
menor coste. El compromiso con la excelencia asegura la calidad de los
productos, la consultoria de confianza y entonces la lealtad de los clientes,
la seguridade de los clientes en que sus necesidades serán satisfechos con la
mejor solución posible. El compromiso con la excelencia permite ganar y retener
a los clientes por un lado y mejorar la imagen de la empresa por otro.

\Paragraph{Compromiso con la caliad y la mejora contínua}

Para el cliente la calidad significa la conformidad de las soluciones entregadas
con los requerimientos previametne acordados. Mientras más se aproxima la
solución entregada a lo que tenía pensado en cliente y mientras mejor la
solución resuelve las necesidades del clietente mayor calidad tendrá el producto
ofrecido. En la empresa la mejora contínua garantiza la alta calidad de los
productos debido a que todas las incidencias se resuelven y todas las formas de
mejorar el producto serán aplicados. La mejora contínua permite la aproximación
gradual a la calidad de las soluciones acumulando los pequeños avances hacia la
perfección. La mejora continua se aplica a todas las actividades de la empresa y
así se demuestra el compromiso de la empresa con los clientes a largo plazo.

\Paragraph{Pasión por lo que hago}

Sin el compromiso personal no hay compromiso social. Todo lo que se hace en la
empresa es porque las personas quieren hacerlo y se sientes horgullosos de lo
que hacen. Si las personas se sienten responsables y tienen poder de tomas las
decisiones el trabajo saldrá mucho mejor, creativo e innovador. Yo creo la
empresa porque siento pación por lo que hago y estoy preparado afrontar los
retos que supone crear y gestionar con éxito una empresa. Solo los profecionales
en su cambo de actividad que creen en si mismos pueden llegar al éxtito. Si te
gusta tu trabajo y si quieres de verdad ayudar a la gente creando las soluciones
y asesorando las decisiones con el trabajo contínuo se llega al éxito.

\Paragraph{Construir las relaciones honestas y abiertas}

Aunque el propósito principal de cada empresa con ánimo de lucro es ganar dinero
sin una cartera amplia de clientes loyales a largo plazo este objetivo no es
aclanzable. Solo las relaciones de confianza y a largo palazo aseguran el éxito
de la empresa. Para construir la relaciones de confianza y a largo plazo es
necesario tratar a los clientes como a sí mismo, enteder sus necesidades y dar
las soluciones adecuadas. En la empresa con cada cliente se construyen las
relaciones honestas y abiernas para que el cliente siente la sinceridad y la
confianza en la emrpesa. Es cuando se construye el futuro de la emrpesa y de los
clientes.

\Paragraph{Compromiso con la sostenibilidad y la responsabilidad}

Para asegurar el éxitno de la empresa a largo plazo todas las actividades de la
empresa tienen que generar valor para todos los agentes del entrono donde
interactúa la empresa. El desarrollo sostenible en undo de los pilares más
importantes de la empresa. El desarrollo sostenible con los clientes se
consigue con las soluciones de calidad, con el trato íntimo de los clientes,
con el compormiso con su satisfacción. El desarrollo con el estado se consigue
mediante el cumplimiento de la legislación vigente y participación en los
programas gubernamentales. La empresa es responsable de todas las actividades
que hacer ya que todas las actividades de la empresa están encaminadas para
mejorar el mundo, crear el valor para los clientes y hacer a sonreir a la gente.

\section{Metas}%{Goals}

\Comment{Where to go and when to get there}

\startitemize
\item Constant innovation \AwRArrowLong\
Use ultimate technology in the solutions
\item Custom-made IT solutions \AwRArrowLong\
Create broad and loyal clientele
\item Quality IT solutions \AwRArrowLong\
Establish quality assurance processes for the solutions
\item Custom-made quality consultancy \AwRArrowLong\
Increase IT consultancy service awareness
\stopitemize

\Paragraph{Innovación constante \AwRArrowLong\ Utilizar la última tecnologia en
las soluciones}

La innovación constante supote la utilización de la última tecnología en las
soluciones para los clientes. La meta consiste en estar siempre actualizado en
el campo de las novedades de tecnología y aplicar la tecnologia la más adecuada
en cada caso específico. La aplicación de la tecnología más adecuada para cada
solución aumenta el valor aportado al cliente por parte de la empresa. Al
aplicar la innovación en cada solución el cliente se beneficiará de mejor
calidad de los clientes, de la fiabilidad y la disponibilidad más alta de la
solución, de los plazos de entrega más cortos y de mayor satisfacción total.

\Paragraph{Soluciones IT personalizadas \AwRArrowLong\ Crear una amplia cartera
de clietens loyales}

La empresa ofrece las soluciones IT personalizadas para todos y cada uno de los
clientes. La soluciones IT personalizadas atraen los clientes porque con este
enfoque se eliminan las limitaciones e los inconvenientes de las soluciones
estrandarizadas. El cliente sabe que la empresa en cada momento de va a ofrecer
la mejor solución de alta calidad que está en conformidad con todos y cada uno
de los requerimientos y las necediades del cliente. La meta consisten en ampliar
la cartera de clientes al forecer les las soluciones personalizadas que mejor se
ajustan a las necesidades de clientes. El compormiso con la mejora continua y la
máxima atención a cada incidencia o cambio de requerimientos por parte del
cliente asegura la lealtad de los clientes que saben que cada una de sus
necesidades, cambios o actualizaciones va a ser antendiad y resuelta de la mejor
forma.

\Paragraph{Soluciones IT de calidad \AwRArrowLong\ Establecer los procesos de QA
(Quality Assurance) para las soluciones}

El compromiso de la empresa con la calidad de soluciones a entregar supone el
establecimiento de los procesos de QA en todas las etapas de la elaboración de
las soluciones y en todas las actividades de empresa en general. Los procesos de
QA garantizan la conformidad del funcionaniento de los sistemas desarrollados
con los requerimientos previamente acordados con los clientes. Los procesos de
QA asegurand que las actualizaciones de software inducidos por los cambios de
los requerimientos solo implementas estos cambios y no influyen de forma
destructiva en la funcionalidad anteriormente desarrollada. Los procesos de QA
permiten mejorar la eficiencias del procesos de desarrollo, reducir plazos de
entrega y mejorar la imagen de la empresa como un proveedores de soluciones
fiables. El establecimiento de los procesos de calidad mejora la eficacia y la
eficiencia de todas las actividades de empresa en general.

\Paragraph{Servicios personalizados de calidad de consultoria \AwRArrowLong\
Incrementar el conocimiento de los servicios de IT consultoria}

Aparte de los servicios de diseño y desarrollo de las colusicones para los
clientes la empresa también orienta y sugiere a los clientes en cuanto a las
tecnologias y mejores formas a crear los sistemas informáticos de los clietes.
La meta consiste en incrementar el conocimiento de los servicios de IT
consultoria que son muy importantes para los clientes porque vale mucho más
pensar antes de empredner un desarrollo de un sistema que no es la que mejor
responde a las necesidades del cliente o un sistema que utilice una tecnología
menos adecuada para este tipo de sistemas. Las implicaciones de una decisión no
acertada repercuten en la fiablidad y la disponitilidad del sistema final, en
la calidad del sistema final, en los plazos de entrega y en los costes del
sistema final. La meta pretende ampliar los ingresos por parte de la consultoria
IT explicando a los clientes que contratando estos servicios los clientes
ahorran dinero que se podrían haber gastado liquidando los desperfectos del
sistema mal deseñada desde su principio.

\section{Objetivos}%{Objectives}

\Comment{How to get there, specific steps to reach your goals, what you must
do and when}

\startitemize
\item Use ultimate technology in the solutions, decrease time to market,
reduce operational costs (innovation goal - new ways to improve)
  \startitemize
  \item Select the most appropriate new technology for each solution
  \item Apply effectively selected techonology in each solution
  \item Increase solution quality and reduce solution delivery time by 10\%
  \stopitemize
\item Create broad and loyal clientele, find new markets/clients (day-to-day
work - increase every day effectiveness)
  \startitemize
  \item Organize 20 meetings with potential clients
  \item Send 100 emails with business offerings to potential clients
  \item Get 5 new clients
  \stopitemize
\item Establish quality assurance processes for the solutions, improve
customer satisfaction (development goal - skills, experience)
  \startitemize
  \item Select the most appropriate testing framework
  \item Apply selected testing framework in each solution
  \item Increase solution quality by 10\%
  \stopitemize
\item Establish IT consultancy service awareness, expand product/service
lines, generate new sorces of revenue (problem-solving goal - address
specific challenges)
  \startitemize
  \item Offer IT consultancy services to potential clients in each meeting
  \item Include in all business offerings emails information about IT
  consultancy services
  \item Get 3 new IT consultancy projects (added value to the solution)
  \stopitemize
\stopitemize

\Paragraph{Utilizar la última tecnologia en las soluciones}

\startitemize
\item Elegir la nueva tecnología más apropiada para cada solución. Antes de
empezar a diseñar y desarrollar cada nuevo proyecto familiarizarse y elegir
la tecnologia que mejor solucione los problemas de la solución y así congribuye
a la calidad del producto, los plazos de engrega y a la satisfacción de cliente
\item Aplicar efectivamente la tecnología elegida en cada solución. Controlar la
aplicación correcta de la tecnología elegida, analizar los resultados y medir
las mejoras consiguidas por utilizar la nueva tecnología. Prevenir los usos de
tecnologia elegida que no dan los resultados esperados
\item Incrementar la calidad de las soluciones en 10\%. Aumentar la satisfacción
de los clientes con las soluciones engregadas, reducir el número de incidencias,
mejorar la estabilidad, la fiabilidad y las disponibilidad de las soluciones
\item Reducir los plazos de entrega de las soluciones en 10\%. Aprovechando las
nuevas tecnologias y la aplicación de proceso de automatización reducir los
plazos de entrega y tiempo necesario para ajustar y adaptar la solución a las
necesidades del cliente
\stopitemize

\Paragraph{Crear una amplia cartera de clietens loyales}

\startitemize
\item Organizar 50 reuniones con los clientes potenciales. Hacer a conocer la
empresa al número máximo de los clientes potenciales. Conseguir las reuniones
con los clientes potenciales y presentarles las ventajas y los servicios únicos
de la empresa. Hacerles conocer la utilidad de los servicios que ofrece la
empresa
\item Enviar 200 correos electrónicos con las ofertas de soluciones IT de
interés a los clientes potenciales. Utilizar el correo electrónico y las redes
sociales para extender la red de contactos potenciales y difundir la misión, la
visión, los valores y el propósito principas de la empresa enviando material
informativo, las ofertas y convocatorias a las presentaciones de la empresa
\item Conseguir 10 nuevos clientes. Firmas los contratos con los clientes para
los proyectos ambiciosos insprando la confianza, la responsabilidad, el
profecionalizmo y la importancia de la satisfacción del cliente para la empresa.
Empezar los desarrollos de las soluciones y entregar los proudctos a los
clientes.
\stopitemize

\Paragraph{Establecer los procesos de QA (Quality Assurance) para las
soluciones}

\startitemize
\item Establecer los más apropiados procesos de QA. Determinar los criterios de
funcionamiento correcto de la empresa y de la elaboración deseada de las
soluciones IT. Esteplecer los controles y los procesos QA que controlarán las
actividades de la empresa y las etapas del desarrollo de las soluciones para los
clientes indicando en cada momento las posición de la empresa o la solución
respecto al objetivo
\item Elegir el más adecuado framework de pruebas unitarias y de pruebas de
integración. Incorporar el uso de los frameworks de pruebas en el flujo de
desarrollo de las soluciones. Determinar los aspectos que se podrían mejorar y
utilizar los automatizmos adecuados para agilizar el proceso de desarrollo
\item Aplicar los procesos y los frameworks en cada solución. Implantar en la
infraestructura de desarrollo los sistemas de pruebas unitarias y de pruebas de
integración. Controlar los resultados de las pruebas y hacer las correciones
correspondientes cuando procede
\item Incrementar la calidad de las soluciones en 10\%. Mejorar la conformidad
de las soluciones entregadas con los requerimiento de los clientes. Anticipar
los cambios futuros de modificación o ampliación de la fucionalidad. Reduric
el número de incidencias y fallos de los sistemas entregados. Asesorar las
decisiones con errores potenciales típicos y subir el nivel de satisfacción de
los clientes
\stopitemize

\Paragraph{Incrementar el conocimiento de los servicios de IT consultoria}

\startitemize
\item Ofrecer los servicios de IT consultoría a los clientes potenciales en cada
reunión. Presentar las ventajas de la consultoría IT para los clientes, su
implicación en los resultados finales de la solución. Intentar la contratación
de los servicios de consultoría IT por los clientes
\item Incluir en todas las ofertas de soluciones IT la información sobre los
servicios de IT consultoría. Promocionar los servicios de consultoría IT en
todos los materiales de soluciones IT. Hacer descuentos y paquetes de servicios
que mejor respondad a las necesidades de los clientes
\item Conseguir 5 nuevos proyectos de IT consultoría (añadir el valor a los
proyectos de las soluciones IT). Firmar por lo menos 5 contratos de consultoría
IT con los clientes. Ofrecerles el mejor posible servicio de orientación y
asesoramiento en el mundo de tecnologia y formas de construir los sistemas
informáticos
\stopitemize

\chapter{Entorno}%{Business environment}

\Comment{Opportunities, threats}

\section{Industria}%{Industry}

\startitemize
\item Information technology (IT) industry
\stopitemize

La empresa prentende operar en la industria de tecnologia de inforamción. La
empresa ofrece los servicios IT para los negocios de los clientes. Las
actividades principales de la empresa son las siguientes:
\startitemize
\item Desarrollo de las soluciones IT
\item Los servicios de la consultoria IT
\item El mantenimiento de las soluciones IT
\stopitemize
Para operar en la industria de la tecnologia de información la empresa dispone
de los conocimientos apropiados y la experiencia de desarrollo de software
durante más que 6 años. La estrategia de la empresa está basada en el crecimiento
sostenible, la innovación y la mejora conginua.

\subsection{Tendencias en la industria de tecnologia de información}%{Trends}

\Comment{Economy, technologies, regulations}

\Paragraph{Virtualización/IaaS (Infrastructure as a Service)}
\Source{ref:5-it-industry-trends}

En los últimos años han tenido mucho éxito las tecnologías de virtualización de
hardware para ofrecer la infraestructura y la plataforma de ejecución para las
aplicaciones. La virtualización permite reducir costes de la infraestructura
mediante devisión y reparto de las prestaciones de un servidor físico entre
multiples servidores virtuales. La tecnologia ofrece los medios de asignación de
las prestaciones y de las prioridades de servicio entres los servidores
virtuales.

La tecnología de virtualización mejora la disponibilidad y la fiabilidad de la
infraestructura de hardware ofreciendo posibilidad de realojar en tiempo real
una máquina virtual de un servidor físico donde se ha encontrado un problema a
otro servidor físico disponible sin interrupción de servicio al cliente.

La virtualización facilita la forma de aumento incremental de pontencia
necesaria de cálculo añadiendo o quitando las máquinas virtuales bajo demanda
para satisfacer las necesidades de la carga de sistema en cada momento. El
enfoque de aumento incremental permite reducir el número de máquinas virtuales
en los momentos de carga baja y aumentorlo solo cuando haya necesidad.

La última tencologia de virtualización hace un paso más y virtualiza no los
sistemas operativos sino las aplicaciones concretas. Esta tecnología se llama
contenerización y es aún más eficiente en comparación con la virtualización
tradicional. Contenerización virtualiza y aisla las aplicaciones dentro del
mismo servidor virtual y aprovecha al máximo los recursos de servidor.

Tanto virtualizacion como contenerización presentan un interes grande para las
empresas porque permiten reducir los costes de la infraestructura de sistemas de
información, mejorar la disponibilidad y la fiabilidad de sistemas, escalar las
prestaciones segun demanda.

\Paragraph{Nube/SaaS (Software as a Service)}
\Source{ref:5-it-industry-trends}

La nube presenta una alternativa muy atractiva al modelo tradocional de mentener
un centro de proceso de datos por parte de la empresa. El modelo de la nube
supone prestación de servicio tanto de hardware como de software por parte de
proveedor. El cliente cuenta con un servidor fiable, disponible y seguro
funcionando 24/7.

Las tecnologías de virtualización han hecho posible el servicio en la
nube. Habitualmente un servidor en la nube tiene menor coste en comapración con
el mantenimiento de un servidor físico por parte de la empresa. La nube es una
opción barata y fiable para las empresas pequeñas y medianas ya que permite
externalizar los servicios de infraestructura de hardware, la red de
comunicaciones y la administración de sistemas operativos centrandose en las
tareas de propio negocio.

Los proveedores de servicios en la nube han echo un paso más y ofrecen a los
clientes no solo los servidores virtuales aptos para servir cualquier tipo de
aplicaciones sino las aplicaciones en sí ya desplegadas y configuradas para el
cliente concreto. Esta estrategia de externalizar el mantenimiento y la
administración de las aplicaciones minimiza los costes de infraestructura de
software para las empresas.

La selección de las aplicaciones disponibles en la nube satisface las
necesidades del ciclo completo de funcionamiento de una empresa desde el
análisis del entorno y la fijación de los objetivos, la elaboración de las
estrategias, la planificación, la ejecución de los planes hasta el control y la
toma de medidas correctivas. Las aplicaciones más comunes ofrecidos en la nube
son las siguientes:

\startitemize
\item ERP (Enterprise Resource Planning)
\item PM (Project Management)
\item CRM (Customer Relationship Management)
\item e-Commerce (Comercio electrónico)
\item BI (Business Intelligence)
\item HR (Human Resources)
\stopitemize

\Paragraph{Software de código abierto}
\Source{ref:3-software-dev-trends}

El papel de software de código abierto sigue creciendo en la infraestructura de
sistemas de información tanto en las empresas pequeñas y medidas como en las
grandes. El software de código abierto ofrece muchas ventajas tanto para los
proveedores como para los consumidores de los productos de software.

Para los consumidores la ventaja principal consiste en reducción considerable de
costes de licencias en comparación con la adquisición del software propietario.
Por otro lado el software de código abierto es más fiable y más seguro porque lo
revisan y corrigen conginuamente muchos ingenieros de sofware altamente
cualifacados no por dinero sino por vocación y su buena voluntad a contribuir a
proyectos ambiciosos y grandes. Otra ventaja que aprovechan los consumidores de
software de código abierto es el soporte del producto y resolución de
incidencias. La pregunta o una incidencia puede ser resuelta en pocos días por
los profecionales de todo el mundo que tienen ganas de ayudar y explicar los
detalles de funcionamiento de su producto.

Para los proveedores de software de código abierto las ventajas consisten en la
oportunidad de influir en la evolución de productos y contribuir a la innovación
de tecnologias de informaicón. Muchas veces su trabajo es esponsorizado por las
grandes empresas que tienen interes en el desarrollo de la tecnologia concreta.
A veces las grandes empresas contribuyen los módulos a los proyectos de código
abierto y de esta manera promocionan las tecnologias de código abierto.

Últimamente muchos proyectos con código cerrado han sido publicados para ganar
el mercado y mejorar su posición frente a la competencia. El modelo de código
abierto ha demostrado su viabilidad y sostenibilidad en las tecnologías de
informaicón y seguirá creciendo.

\Paragraph{Dispositivos móviles}
\Source{ref:5-it-industry-trends}

Cada vez más accesos a Internet se hacen desde los dispositivos móviles. En 2014
el trafico total de acceso a Internet generado por los teléfonos móviles y los
táblets ha superado el trafico total de acceso a internet generado por los PCs.

Cada vez más los empresarios y los hombres de negocio prefieren un teléfono
móvil o un táblet a un portatil o PC. La nube, las comunicaciones de alta
velocidad y las redes sociales contribuyen al proceso de transformación en los
dispositivos de acceso a Internet de portatil/PC a móvil/táblet.

Para una empresa es muy importante estar presente en el ámbito de dispositivos
móviles con el fin de ofrecer el mejor servicio a sus clientes a traves de
todos los canales de acceso a los servicios. En el desarrollo de las
aplicaciones hay que hacer el enfoque en las plataformas de dispositivos
móviles debido a su aceptación y uso universal.

\Paragraph{Adopción de HTML5 y CSS3}
\Source{ref:18-trends-in-app-dev}

Desde hace unos años tiene lugar el dilema con que se enfrentan muchas empresas
que ofrecen servicios en la Web tanto para los navegadores de PCs como para los
dispositivos móviles. El dilema consiste en que para ofrecer las mejores
experiencias para los usuarios de los dispositivos móviles hay que desarrollar
las aplicaciones nativas para todas las plataformas que se necesita soportar
(iOS, Androdid). Este enfoque supote un coste elevado para las empresas de
desarrollar y mantener la aplicación en varias plataformas. Otro enconveniente
de este método es la falta de consistencia de contenido y de experiencia al
usuario debido a las diferencias en las plataformas de dispositivos móviles.
Otra alternativa supote utilización de una única tecnología para servir la
aplicación en todas las paltaformas. En este caso se utiliza la tecnologia
HTML5 con CSS3 y JavaScript. El inconventiente que anteriormente dominaba en
este ámbito era la experiencia visual limitada que podía ofrecer HTML5 en
aquel momento.

La ventaja de HTML5 con CSS3 consiste en que esta tecnologia es un estandar que
soportan todos los navegadores. HTML5 con CSS3 está en continuo desarrollo
aplicando los métodos más innovadores y probados en la industria de tecnologia
de información. HTML5 asegura la consistencia en todas las plataformas de
dispositivos móviles. Las empresas tienen que desarrollar y mantener solo una
única aplicación que serve todo el mercado. Últimamente las experiencias
visuales que ofrece HTML5 se comparan con las experiencias visuales de las
aplicaciones nativas lo cual nivela ambas tecnologias en cuanto y las
experiencias visuales. Mientras que la ventaja de una única aplicaciones para
todo el mercado apunta a favor de HTML5.

Un argumento más a favor de HTML5 con CSS3 y JavaScript radica en que Facebook
ha desarrollado un framework para desarrollo de aplicaciones nativas para iOS y
Android utilizando HTML5 y JavaScript. La tecnología se llama React.js Native.
La publicación del proyecto como software de código abierto demuestra un alto
interes en promoción de HTML5 por parte de las grandes compañías como la
tecnología para el desarrollo de las aplicaciones para dispositivos móviles.

\Paragraph{Modelo asíncrono de ejecución en el servidor/cliente}
\Source{ref:3-software-dev-trends}

Con el aumento de trafico en Internet la carga a los servidores Web ha crecido
mucho. Ha surgido la necesidad de incrementar el número de servidores para
satisfacer la demanda creciente de clientes. El modelo de ejecución en los
servidores tradicionales está basado en hilos. Cada petición requiere el acceso
a subsistemas relativamente lentas tales como el sistema de ficheros, las
bases de datos relacionales, el acceso a Web APIs de terceros. Estos acceso
lentos bloquean hilos en el servidor Web consumiendo los recursos limitados del
servidor. Cuando el número de peticiones y su frequencia es grande el modelo de
ejecución síncrona no es aceptable porque no escala bien.

Para resolver el problema de muchas peticiones concurrentes con el mínimo de
recursos disponibles en los úlitmos años se ha aplicado el modelo asíncrono de
ejecución en el servidor. El modelo asícrono de ejecución no bloquea la ejeción
en el servidor Web mientras se espara la respuesta de sistema de ficheros o la
respuesta de la bases de datos o la respuesta de Web API, sino que se crea una
función que se va a llamar cuando la respuesta esté lista para su procesamiento
posterior y el servidor Web sigue serviendo más peticiones.

El modelo asícrono de ejecución ha sido la clave de éxito del servidor Web
NGINX. Muchas empresas de todo tipo han podido reducir considerablemente sus
granjas de servidores Web tradicionales sustituyendolos por servidores Web
NGINX. El servidor Web NGINX está dominando el mercado de servidores Web. Otro
ejemplo de modelo asíncrono de ejecución es el framework Node.js. Node.js es
un entorno de entrada/salida asíncrono que utiliza JavaScript para implementar
la funcionalidad de aplicaciones en la parte de servidor.

En la parte de cliente el modelo asíncrono de ejecución dominaba desde su
invento. JavaScript es el lenguage de programación de la Web. Con la
introducción de Node.js en la parte de servidor JavaScript se ha convertido
un un lenguage de programación universal de la Web tanto en la parte de cliente
como en la parte de servidor. Las aplicaciones web más innovadores hoy en día
son isomorficas. Las aplicaciones isomorficas se pueden ejecutar tanto en el
servidor como en el cliente. Si la peticion al servidor Web se realiza desde un
cliente con capacidades de procesamiento de JavaScript (por ejemplo un navegador
Web) el servidor responde inmediatemente al cliente con la aplicación que se va
a ejecutar en el cliente omitiendo la carga de ejecutar la aplicación en el
servidor Web. Este tipo de aplicaciones se llaman aplicaciones SPA (Single Page
Application). Si por el contrario el servidor recibe la peticion desde un
cliente sin capacidad de procesamiento de JavaScript (por ejemplo un robot de
una máquina de búsqueda) el servidor ejecuta la aplicación y responde al cliente
con el HTML ya procesado. Este modelo de funcionamiento de servidor Web se llama
SEO (Search Engine Optimization) y permite servir el contenido ya procesado a
los clientes sin capacidad de procesamiento de JavaScript.

El modelo asíncrono de ejecución es la evlución natural de proceso de
interacción entre distintos agentes en Internet ya que resuelve los problemas de
eficiencia y escalado y además es más cercado a la conversasión natural.
JavaScript se ha convertido en un lenguage universal de la Web y su naturaleza
asíncrona encaja perfectamente en el modelo asíncrono de ejecución.

\section{Clientes B2B}%{Customers}

El cliente objetivo para la empresa es el cliente B2B (Business to Business). La
empresa se posiciona como la empresa que ofrece los servicios IT para el negocio
de los clientes. En concreto la empresa ofrece las soluciones IT de calidad
personalizadas, los sorvicios de consultoria IT y los servicios de mantenimiento
de las soluciones IT. La naturalesa de los productos y servicios que efrece la
empresa implican el modelo de negocio B2B porque la soluciones IT que ofrece la
empresa son los servicios IT para el negocio de los clientes.

Los clientes B2B de la empresa son las compaías medianas y pequeñas de distinas
insustrias y sectores. Lo que une a los clientes de la empresa son las
necesidades de los servicios IT para el negocio. Con el desarrollo tecnológico y
la digitalización de la sociedad los servicios IT para el negocio son muy
demandados por todas las empresas de todas las industrias de todos los sectores.

Los clientes B2B de la empresa geograficamente pueden estar en cualquier punto
del mundo. Las tecnologias de comunicación y transferencia de datos permiten
establecer los contacnos con los clientes en cualquier punto del mundo. El
producto de la empresa lon las soluciones IT. La transferencia, la instalación y
la configuración de las soluciones IT se puede hacer de forma remota desde los
sistemas informáticos de la empresa. No obstante los primero clientes de la
empresa van a ser las empresa esñolas localizadas en Madrid. La empresa puede
establecer mejor el contacto con las empresas locales.

Las características del cliente objetivo de la empresa son las siguientes:

\Source{ref:b2b-market-differences}

\startitemize
\item Los clientes B2B buscan la solución que mejor satisface sus necesidades.
Los clientes B2B se encuentaran en el medio de la cadena de valor y tienen las
necesidades precisas para ofrecer su producto a sus clientes que están en el
siguiente tramo de cadena de valor. Los clientes B2B vienen a la empresa con el
propósito de comprar las solución que satisface sus necesidades de la mejor
forma. La empresa no tiene que gastar mucho dinero en la publicidad debido a que
si la solución de la empresa satisface mejor las necesidades de los clientes B2B
los clientes B2B la comprarán porque es un producto requerido en la operativa
del cliente para ser competitivo, eficiente y tener buena rentabilidad
\item Las caracterísiticas del rendimiento de los productos es lo más importante
para los clientes B2B. No hay necesidad de gastar mucho dinero en la publicidad
y en la imagen visual del producto. Si el rendimiento de producto satisface de
mejor forma las necesidades de de los clientes se podrá vender
\item El tamaño de mercado y la cantidad de los clientes en el caso del modelo
B2B es más pequeño en comparación con el modelo B2C (Business to Client). El
tamaño de mercado B2B más pequeño permite a la empresa ofrecer las soluciones IT
personalizadas a cada cliente y adaptar mejor la comunicación de marketing a las
necesidades de cada cliente. La empresa establecerá el contacto personal con
cada cliente con el objetivo de entender mejor las necesidades del cliente y
ofrecer las soluciones IT personalizadas
\item El volumen de dinero que se maneja en cada transacción con los clientes
B2B es considerablemente más grande en comparación con los clientes B2C debido a
la mayor complejidad de los productos para los clientes B2B y el valor que
aportan las soluciones IT a los clientes B2B. Con cada cliente la empresa puede
ganar mucho dinero en el caso de que le ofrece la mejor solución a las
necesidades de los clientes. Unos cuanto clientes de la empresa puden generar la
mayor parte de los ingresos de la empresa. La empresa ofrece las soluciones IT a
los clientes B2B cuya valor para los clientes B2B es muy grande porque de estas
soluciones IT depende todo el negocio de los clientes de la empresa. Las
decisiones de los clientes B2B sobre las solucioens IT a utilizar tienen mayor
impracto sobre el negocio de los clientes B2B a largo plazo es donde reside el
valor que aporta la empresa a los clientes B2B
\item Las relaciones de la empresa con los clientes B2B son más personales, más
íntimas lo cual garantiza el mejor entendimiento de las necesidades de los
clientes B2B y como consecuencia las mejores soluciones IT. La empresa
establecerá con cada cliente las relaciones uno a uno y de esta menera
fortalecerá las relaciones con cada cliente B2B. Las relaciones de la empresa
con los clientes B2B son a largo plazo porque las desiciones sobre las
soluciones IT hacen depender los negocios de los clientes de la empresa durante
la vida útil de las soluciones IT implantadas. El valor que aporta la empresa a
los clientes B2B es el valor a largo plazo
\item Las soluciones IT para los clientes B2B son más complejos para satisfacer
todos los requerimiento de los clientes B2B. Los clientes B2B precisas mayo
nivel de configuración y adaptación de las soluciones IT a los entronos de
producción. Con el objetivo de mantener a los clientes actuales y de ganer los
clientes nuevos cada solución IT que ofrece la empresa es personalizada y está
diseñada para responder a todas las necesidades y todos los requerimeintos del
cliene. Ofrecer el producto IT de valor para los clientes B2B es muy importante
para la emprea, es la garantia del futuro crecimient y éxito de la empresa
\item El proceso de compra de los clientes B2B comprende varias etapas e
involucra muchas personas que son profecionales en sus áreas de actividad. El
proceso de compra de las soluciones IT es más consciente y formal. Los clientes
B2B tienen las necesidades y los requerimeintos bien pensados. Los clientes B2B
evaluan varias opciones y la decición de compra es más pensada y evaluada por
varias personas. Las personas involucradas en el proceso de compra de las
soluciones IT son profecionales y conocen bien el mercado, las soluciones
alternativas y los beneficios de cada una de las soluciones IT
\item Los clientes B2B están mucho mejor informados sobre la solución IT que
quieren comprar. Antes de comprar los clientes estudian las diferentes
alternativas de las solucioens IT. En este proceso los clientes B2B aprenden
mucho sobre las prestaciones y las características de cada solución IT, sobre el
procesos internos de desarrllo y sobre los valores y los compromisos de las
empresas fabricantes. El compromiso de la empresa con la innovación y la mejora
contínua mejora considerablemente la posición de la empresa frente a la
competencia
\stopitemize

%\subsection{Segmentación de clientes}%{Customer segmentation}

%\subsection{Perfil de cliente}%{Customer profile}

\section{Competencia}%{Competition}

La competencia de la empresa son otras empresas del sector que ofrecen las
soluciones IT dentro del modelo B2B. La diferencia más importante que sitúa
a la empresa en una posición de ventaja en comparación con la competencia es el
compromiso fuerte de la empresa con la innovación y la relación calidad/precio
superior.

\chapter{Descripción de la empresa}%{Company description}

\Comment{Strengths, weaknesses}

\section{Productos y servicios}%{Products and services}

\Comment{Unique}

\subsection{Resumen de productos y servicios}%{Products and services summary}

\startitemize
\item Custom-made quality software systems
\item IT consultancy services
\stopitemize

\Paragraph{Soluciones IT}

La actividad principal de la emresa va a estar dirigida a la elaboración de las
soluciones IT para los clientes. Habitualmente las soluciones IT consisten en el
software y los sistemas de información por un lado y la infraestructura de
hardware y comunicaciones por otro lado. La empresa se va a centrar en la parte
de diseño y desarrollo de software porque es la parte donde más se emplea la
capacidad intelectual y los conocimientos, la innovación y la creatividad. En el
diseño y el desarrollo de software hay mucho margen de maniobra y es dodne la
empresa destaca ofreciendo mejores soluciones innovadores y creativas a los
clientes. Hay mucha formas de construir el software y los que los hacer
realmente bien logran el éxito, incrementan su cartera de clientes loyales y
determinan las futuras diracción del desarrollo de sistemas informáticos. La
parte de infraestructura de hardware y las redes de conumicaciones de sistemas
diseñados se va a subcontratar. De esta manera la empresa puede mejor centrarse
en el desarrollo y perfección se la actividad principal de diseño e
implementación de software. La subcontratación de servcios de infraestructura
de hardware y de las redes de comunicaciones permite reducir los costes y los
plazos de entrega de los productos IT. La subcontratación beneficia tanto al
cliente como a la empresa.

Los sistemas de software son muy diversos en muchas caracterísiticas: sistemas
complejos y sistemas simples, sistemas en la parte cliente y en la parte
servidor, sistemas que impelmenta la lógica de negoción y sistemas de control
de calidad de datos, sistemas que ofrecen los intarfaces de conexión y sistemas
de integración. La empresa pretende ofrecer al cliente las soluciones IT
completas para su negecio lo cual obliga a controls un abánico muy amplio de
tecnologias y métodos tecnológicos. A pesar de tanta variedad de tecnologías el
cliente objetivo normalmente tiene problemas comunes a resolver y es dodne se
aprovecha del efecto experiencias y las sinergias de conocimientos adquiridos en
los proyectos anteriores. Los principales soluciones IT que ofrece la empresa
son los siguientes:

\startitemize
\item Sistemas front end que implementan la lógica de interacción con los
sistemas core de negocio de cliente
\item Sistemas back end que implementan la lógica core de negocio de los
clientes
\item Integración de sistemas del clientes con terceros
\item Systemas de presencia Web en Intenet: sistios web de las companias o
productos específicos de los clientes
\item Aplicaciones web de venta de los productos o consumo de los servicios que
ofrece el cliente
\item Aplicaciones para los móviles de venta de los productos o consumo de los
servicios que ofrece el cliente
\item Sistemas de pago
\item Sistemas de facturación y contabilidad
\item Sistemas de monitorización y control de procesos
\stopitemize

\Paragraph{Consultoría IT}

La empresa ofrece a los clientes los servicios de consultoria IT. Los servicios
de consultoria IT tienen un gran impacto en el éxito futuro de los proyectos de
los clientes. La orientación y en el campo tecnológico y la selección de las
tecnologias más adecuadas en mucha parte determinan la calidad del futuro
sistema. La empresa puede prestar los servicios de consultoria IT por separado o
dentro de un paquete se consultoria IT más la solución IT basada en los
resultados de la consultoría IT. Los servicios de consultoria IT permiten a los
clietens evaluar las ventajas y los inconvenientes de varias opciones
tecnológicas y al final elegir la opción que más se adapta a las necesidades del
clietene o incluso combinar las opciones disponibles en un formula de éxito.
Dentro de las opciones que seirve la consultoría IT se encuantran los
siguientes:

\startitemize
\item Asesoramiento de arquitectura de sistemas informaticos y las solucioens a
proporcionar
\item Asesoramiento de las tecnologías y modelos de ejecución a utilizar en las
soluciones
\item Diseño de alto nivel de los sistemas a desarrollar
\item Sugerencias de infraestructura de hardware que mejor responde a las
necesidades de cliente
\item Valoración cualitativa y valoración cuantitativas entre varias opciones
porpuestas a elegir
\item Evaluación de capaciadad de procesamiento según la carga de peticiones al
sistema con la posibilidades se futuras ampliaciones
\item Evaluación de la disponibilidad y la fiabilidad total del sistemas a
desarrollar
\stopitemize

\Paragraph{Mantenimiento y modificaión de las soluciones}

La empresa tambien ofrece los servicios de mantenimiento de las soluciones
desarrolladas, las actualizaciones de software y las modificación y cambio
necesarios para adaptar las soluciones a los cambios futuros imprevistos en la
fase de diseño inicial de las soluciones. Tenciendo en cuenta que el mundo de
negocio y el mundo de tecnologias de información cambia muy rápido las
necesidades de adaptar las solucioens surgen muy amenudo. Para responde a estos
cambios la empresa ofrece los servicios de mantenimiento y modificaión de las
soluciones. Estos servicios se pueden contratar aparte de la solución IT
principal o estar incluidos dentor del paquete junto con los servicios de
consultoria IT y las soluciones IT a desarrollar. La empresa ofrece los
siguientes servicios:

\startitemize
\item Mantenimiento de software y las solucioens desarrolladas
\item Realización de cambios y modificaciones en las soluciones
\item Amplicación de capacidad y escalado de servicio que prestan las soluciones
\item Incremento de la fiabilidad y la disponibilidad de los servicios
\item Monitorización y el control del funcionamiento de las soluciones
\stopitemize

\subsection{Características de los productos y los servicios clave}
%{Key product and service features}

\startitemize
\item Quality, reliability, flexibility, short delivery time, aceptable price
\stopitemize

\Paragraph{Calidad}

La calidad de las soluciones IT para los clientes significa la conformidad con
los requerimeintos previamente acordados entre la empresa y el cliente. Para la
empresa la satisfacción de los clientes es lo más importante. Con el objetivo de
satisfacer al cliente la empresa hace lo mejor para que el producto final sea en
la mayor medida identico a lo que tenía pensado el cliente. La empresa entiende
la calidad en el sentido de que la solución tiente que funcionar en todos los
casos de uso que especificó el cliente y como especificó el cliente. La solución
no debe tener los comportamientos idefinidos en casos de uso no especificados.
La solución tiene que estar conforme con las normas establecidas de desarrollo
de sistemas informaticos.

\Paragraph{Fiabilidad y disponibilidad}

La solucion que ofrece la empresa tiene que ser fiable. La fiabilidad significa
que las respuestas del sistema tienen que ser correctas idependientemente de las
condiciones en que trabaja el sistema. El nivel de carga, el tiempo de respuesta
de otros subsistemas con los cuales está conectado el sistema de deben influir
en el resultados final de las respuesta del sistema. La empresa ofrece las
solucioens con la máxima fiabilidad meidante la aplicación de los métodos de
diseño y los métodos de desarrollo de software innovadores y con fiabilidad
demostrada utilizando los medotod formales de razonamiento. La disponibilidad de
un sistema informático se entiende como la capalidad de la solución funcionar
sin interrupciones. La disponibilidad de las soluciones depende mucho de la
fiabilidad del sistema y de la infraestructura de hardware. La empresa ofrece
las soluciones disponibles 24/7 gracias al diseño de las arquitecturas de
software que permiten construir los sistemas altamente disponibles con el
mínimo de los recursos necesarios.

\Paragraph{Flexibilidad}

La mayor parte de tiempo que se emplea en la elaboración y el mantenimiento de
los sistemas informáticos se dedica a realizar los camios y las modfiaciones en
el funcionamiento de los sistemas. Los cambios y las modificaciones en las
soluciones son inevitables. La empresa ofrece las soluciones flexibles y que
permiten realizar las modificación en plazor cortos de tiempo. La flexibilidad
de las soluciones que ofrece la empresa se consigue mediante las desiciones
arquitectónicas que se toman en la fase de diseño utilizado los componentes
altamente cohesivos con bajo acoplamiento. La flexibilidad de las soluciones
permete modificar o extender la funcionalidad de las soluciones en cortos
plazos y los costes asociados no son altos.

\Paragraph{Cortos plazos de entrega}

La ventaja de las soluciones que ofrece la empresa son cortos plazos de entrega.
Los cortos plazos de entrega se consiguie mediane la aplicación de los métodos
innovadores en todas las faces de la construcción de las soluciones. La
organización de la empresa y la gestión de los proceso dentro de la empresa
tambien contribuyen a la reducción de los plazos de entrega. La empresa aplica
los métodos de desarrollo iterativo. El desarrollo iterativo supone las entregas
frequentes a los clientes de la funcionalidad incrementa de la solución. De esta
manera el cliente puede probar la funcionalidad por partes y modificarla
mientras la solución se está desarrollando. El desarrollo iterativo permite
tener mejor contacto con el cliente, escucharle con mayor atención y responder
a sus necesidades de forma más proactiva.

\Paragraph{Precio aceptable}

La empresa ofrece las soluciones con los precios muy competitivos. La reducción
de precios ha sido posible gracias a utilización de software de código abierto
que no precisa la adquisición de las licencias. Los métodos innovadores y
eficientes del desarrollo de software tabmien contribuyen a la reducción de los
costes. La estructuras planas de ortanización y la gestión eficiente de los
proceso en la empresa permiten ofrecer a los clientes las soluciones de alta
calidad, personalizadas y con precios por debajo de la media del mercado. Alta
calidad de los productos, bajos precios y cortos plazos de entrega hacen los
productos de la empresa unicos. Los productos que engobal estas calidades son
únicos en el mercado y atraen mucho a los clientes.

\subsection{Cliente objetivo}%{Target customers}

El cliente objetivo de la empresa es el cliente de tipo B2B (Business to
Business). La empresa ofrece los servicios a otras empresas que tienen la
necesidad de construir sus sistemas informáticos que implementan su lógica de
negocio. El cliene objetivo de la empresa son las compañías pequeñas y medianas.

\subsection{Beneficios clave para el cliente}%{Key customer benefits}

\Paragraph{Soluciones y consultoría de alta calidad}

La ventaja para el cliente de la empresa consiste en que el cliente siempre
recibe el producto de calidad. La calidad de producto es el compromiso principal
de la empresa. Le excellencia en los procesos internos de funcionamiento de la
empresa y en los procesos de desarrollo de software garantizan la alta calidad
de los productos finales. La innovación y la creatividad en el diseño y el
desarrollo de las soluciones aumentan la calidad de los productos. La soluciones
de la empresa siempre están en conformidad con los requerimientos de los
clientes anticipando los posibles problemas y dificultades.

\Paragraph{Soluciones de alta fiabilidad y disponibilidad}

La soluciones de la empresa son fiables. La fiabilidad de las soluciones
garantiza al cliente los resultados correctos de funcionamiento del sistema en
todos los casos de uso. La fiabilidad beneficia al cliente con la seguridad de
que la solución no se comportará de forma indefinida dado una condiciones no
previstos en los requerimientos. El funcionamiento correcto de las soluciones
mejora la imagen corporativa de los clientes y beneficia a la empresa tambien.
La empresa ofrece a los clientes soluciones altamente disponibles. La ventaja
para el cliente de tener una solución disponible radica en el servicio continuo
y sin interrupciones. El echo de estar seguro de que las plataformas y las
aplicaciones desplegadas están funcionando mejora la posción de la empresa en
los ojos de sus clientes y le permite centrar sus esfuerzos en la actividad
principal.

\Paragraph{Soluciones flexibles}

En el mundo tecnológico disponer de una solución flexible que permite cambios
rápidos en su funcionamiento consta una gran ventaja para los clientes. Todas
las soluciones muy amenudo requieren cambios y modificaciones que pueden suponer
un gran esfuerzo y mucho tiempo. Si la solución permite gestionar los cambios de
manera agil el sistema mejor satisface las necesidades de los clientes y
entonces presenta una ventaja para los clientes. Por otro lado las futuras
extensiones del sistema se realizarán de forma más agil y rápida con las
soluciones que ofrece la empresa. El escalado y el aumento de potencia para
servir la demanda crecciente de las peticiones se hace fácil con los sistemas
diseñadas por la empresa.

\Paragraph{Cortos plazos de entrega de las soluciones}

La ventaja de cortos plazos de entrega de las soluciones a los clientes es sin
duda de mucho valor. Los métodos innovadores permiten reducir los tiempos de
entrega considerablemente. La empresa cuenta con el método de desarrollo
iterativo que permite entergar a los clientes las soluciones frequentemente para
que los clientes puedan probar la funcionalidad ya implementada mientras la
solución está en el desarrollo. Los cambios y las mejoras son casi inmeidatas y
el cliente ve crecer a la solución de manera natural probandola en las
condiciones naturales. La empresa pretende ser reactiva e incluso proactiva
anticipando las necesidades de los clientes.

\Paragraph{Superior relación calidad/precio}

La empresa ofrece a los clienes el producto único porque la relación
calidad/precio en las soluciones es muy alto en comparación con otras solucioens
en el mercado. Mejorar la calidad permiten los métodos de desarrollo
innovadores. Reducir los costes permite el uso del software de código abierto
que no require la adqusición de la licencia. Para el cliente teneru un sistema
de alta calidad por menor precio es la mejor ventaja posible. En mejorar la
calidad y bajar los precios la empresa ve el valor que aporta al cliente.
Satisfacer al cliente es la mayor prioridad para la empresa.

\section{Capacidades de la empresa}%{Business capabilities}

\Comment{Competitive advantages}

\subsection{I+D}%{R+D}

\startitemize
\item Design and develop new products and services
\stopitemize

En la empresa el área de I+D es muy importante porque es estrategia de la
empresa pone mucha enfasis en la innovación. Los productos de la empresa
tienen que ser muy innovadores para ser competitivos en el mercado. La inversión
en I+D permite destacar en el mercado y aportar más valor al cliente. En la
empresa se invierte mucho tiempo en las tareas de I+D con el objetivo de
encontrar las soluciones más adecuadas a los problemas de los clietens.

En la empresa hay varias líneas de las actividades de I+D. La primera línea de
I+D es la innvación en los métodos de diseño y desarrollo de software. La
innovación en los métodos de diseño y desarrollo de software permite encontrar
los métodos más eficientes y diseñar el software de mejor calidad, fiablidad y
disponibilidad. Cada solución para los clientes tiene sus peculiaridades de
los requerimientos y de su funcionamiento. Para entregar al cliente las
soluciones personalizadas de alta calidad es necesario aplicar los métodos más
adecuados en cada caso. Las actividades de I+D se centran en los métodos de
diseño de las arquitecturas de software y en las técnicas de emplementación de
software.

La segunda línea de actividades de I+D es la mejora continua de los procesos QA
(Quality Assurance). El compromiso de la empresa con la calidad y la mejora
contínua en las soluciones induce las actividades de este índole. Los procesos
QA aseguran la calidad y la fiablidad de las soluciones y ahorran mucho tiempo
en la fase de pruebas con cliente de las soluciones. Los procesos QA permiten
automatizar el proceso de pruebas de la solucion sobre los casos de uso
previamente acordados con el cliente. Las actividades de I+D se dedican a
encontrar mejores formas de construir las baterias de pruebas automatizadas,
qué probar cómo probar y cuáles son los invariantes crtíticos para el
funcionamiento correcto de la solución.

La tercera línea de las actividade de I+D se dedíca a la gestión de proyectos.
La gestión de proyecto en mucha parte determina el éxito de la solución a
proporcionar. Los métodos innovadores de gestión de proyecto permiten mejorar y
agilizar tales etapasa del proyecto como la gestión de clientes, la toma de
requerimientos, programación de las etapas de desarrollo y las fechas de los
entregables, las pruebas, la puesta en marcha de las soluciones y la
configuración. La gestión de los proyectos mejora el trabajo en equipo y
buenas relaciones con los clientes. Las actividades de I+D se centran en las
metodologías agiles de gestión de proyectos, la gestión de tiempo, optimización
de los proceso de desarrollo.

\subsection{Operaciones}%{Operations}

\startitemize
\item Produce products and provide services
\item Location (Madrid, near your customers)
\item Equipment (PC, Servers, Software)
\item Labor (design, implement, test, deliver, support)
\item Process (account contacts, product/service definition meetings,
  production, customer support)
\item Suppliers
\item Manufacturing
\item Quality control
\stopitemize

La actividad principal de la empresa es la construcción, la venta de software y
otros servicio IT para el negocio de los clientes. La construcción y la venta
de software es un proceso complejo que requiere una buena gestión y
coordinación. La actividad principal de la empresa se puede dividir en las
siguientes etapas:

\startitemize
\item Gestión de los clientes de la empresa
\item Toma de requerimientos de las soluciones a desarrollar
\item Diseño de la arquitectura de las soluciones a desarrollar
\item Desarrollo de software
\item Realización de los procesos de QA
\item Despliegue y configuración de las soluciones
\item Mantenimient de las soluciones
\stopitemize

\Paragraph{Gestión de los clientes de la empresa}

La tarea más importante para la empresa es conseguir los clientes nuevos,
establecer las relaciónes de confianza y a largo plazo con los clientes y
mantener a los clietnes satisfechos. Para lograr estos objetivos la empresa
ofrece a los clientes las soluciones IT personalizadas de calidad en plazos
cortos de entrega a precios aceptables. La empresa hace las presentaciónes
personalizadas y dirigidas a los clientes potenciales para despertar el
interés de los clientes potenciales en la empresa. En estas presentaciones
se explican las ventajas numerosas de los productos y los servicios que ofrece
la empresa. Las presentciónes de los productos y de los servicios se realizan
con el número mayor posible de los clientes potenciales.

Para los clientes potenciales se hacer evaluaciones gratuitas de los sistemas
a desarrollar por parte de la empresa. Se presentan las ventajas y se evaluan
los riesgos al implantar las nuevas soluciones IT dento del modelo de
fucionamiento de negocio de cada cliente. El objetivo de las evaluaciones de las
ventajas y de los riesgos de las nuevos sistemas es demostrar a los clientes el
valor que aportan las souciones nuevas y las formas de mejorar la gestión de los
procesos de negocio de los clientes.

Para los clietens que han decidido contratar el desarrollo de las nuevas
soluciones se realizan las reuniones donde se concretan las tareas a realizar,
los roles de cada parte, los plazos de entrega de las soluciones, las formas de
pago y otros detalles de los acuerdos contractuales. El resultado de la
actividad de gestión de los clientes es el contrato entre el cliente y la
empresa para el desarrollo de la solución con las caracterísiticas previamente
acordadas.

\Paragraph{Toma de requerimientos de las soluciones a desarrollar}

La toma de requisitos de las soluciones a desarrollar es una actividad muy
importante en el proceso de desarrollo de las soluciones porque determina todo
la trayectoria del proyecto. Mientras mejor se puede endender lo que el cliente
quiere y que problemas pretende resolver más satisfecho estará el cliente
después de la entrega de la solución. Es muy importante entender el problema a
resolver, las limitaciónes que tiene el entorno y las necesidades del cliente.
La comprención profunda de los requerimiento permite anticipar las futuras
evoluciones de la solución y anticipadamente preperar la arquitectura del
sistema informático a desarrollar.

El proceso de la toma de requisitos consiste en una o variar reuniones con el
cliente donde el clieten explica los problemas que quiere resolver, sus
necesidades, el entorno en que opera y las futuras aspiraciones para la
solución a desarrollar. Por otro lado la empresa pregunta al cliente los
casos de uso, los modos de operación, las espificaciones de lógica de
funcionamiento y otros detalles de la solución a desarrollar. En unos casos la
empresa puede sugerir las soluciones alternativas mejores u otras formas de
resolver los problemas del clieten. El proceso de toma de requerimientos es un
proceso iterativo. En las primeras reuniones se concreta la forma general y la
funcionalidad principalde la solución a desarrollar. Después de haber entregado
al cliente los primeros prototipos de la solución el cliente manifestará su
opinión sobre el funcionamiento del sistema y la emprea realizará las
corecciónes correspondientes en la solución. De esta manera el proceso de toma
de los requerimientos es un proceso iterativo y permite refinar y pulir la
solución durante las faces de desarrollo y de esta manera aportar más valor al
cliente. La essencias del proceso de desarrollo iterativos consiste en dejar al
cliete la libertad de continuamente mejorar la solución en base a la
expreciencia adqurida durante las pruebas de los prototipos que se entragan
periódicamente al cliente. El proceso de desarrollo iterativo es la ventaja que
ofrece la empresa a sus clientes. El resultado de la actividad de toma de los
requerimientos es un documento con todas las detalles de funcionamiento de la
solución a desarrollar.

\Paragraph{Diseño de la arquitectura de las soluciones a desarrollar}

El diseño de la arquitectura de sofware es el proceso crucial que determina la
calidad, la fiabilidad y la flexibilidad del futuro sistema. La arquitectura de
software presenta la base de todo el sistema y las posibilidades de futuras
extensiones. El proceso de diseño de la arquitectura de software requiere mucha
experiencia y muchos conocimientos técnicos y de ingenieria de software.

Las decisiones arquitectónicas tienen que tomarse a deferentes niveles del
sistema a desarrollar. Primer nivel supone la definición de los actores del
sistema y su alto nivel de interacción. En esta face se definen los componentes
del sistema de alto nivel y los flujos de información entre ellos. El primer
nivel de arquitectura describe las diferentes aplicaciones que forman el sistema
indegrándose de forma natural. El segundo nivel prentende definir la estructura
de cada uno de los componentes del sistema por separando. Este nivel define la
estructura interna de las aplicaciones, las estructuras de datos, las
estrucuturas de lógica de cada una de los componentes de la aplicación. El
resultado del diseño de la arquitectura de software es el documento técnico que
describe los componentes del futuro sistema y la estructura interna de cada uno
de los componentes junto con los flujos de información entre los componentes.

\Paragraph{Desarrollo de software}

El desarrollo de software presemta la construcción de código funete de la
solución a partir de la arquitectura previamente diseñada. Este proceso
precisa muchas experiencia y muchos conocimientos técnicos. La idea general del
proceso consiste en traducir la arquitectura de alto nivel en instrucciones que
entiende el ordenador de manera que no haya sitio para errores, abiguedades o
comportamientos indefinidos.

Duarnte el proceso de desarrllo de software se aplican diferentes lenguates de
programación. Cada componente del sistema precisa la aplciacion del lenguage más
adecuado para la tarea en questión. Los componentes a desarrllar de dividen en
una estructura jerárquica. Las estructuras jerárquicas constan la essencia de
desarrollo de software porque es la abstracción más potente a la hora de
resolver los problemas complejos. La estructura jerárquica permite devidir un
problema complejo en varias problemas menos complejas que tienen la solución más
simple. Esta tecnica se aplica tantas veces que sea necesario para convertir la
solución de un problema complejo a la la solución de muchas problemas simples.
Por otro lado la presetnación de la solución de un problema complejo en formara
de una jerarquía de los problemas simples aumenta la cohesión entre los
componentes, disminuye el acomplamiento entre los componentes y permite probar
cada componente por separado. El resultado del procose de desarrollo de software
es el código fuente de la socluión. Es la essencia de producto que ofrece la
empresa.

\Paragraph{Realización de los procesos de QA}

Los procesos de QA (Quality Assurance) aseguran la calidad y la fiablidad del
sistema. Los procesos de QA automatizan las pruebas y ofrecen la posibilidad de
probar de forma automática todos los casos de uso acordados con el cliente. La
ventaja de la autimatización de las puebas es que después de cada cambio en el
sistema se puede lanzar la batería de pruebas y asegurar el correto
fucionamiento del sistema.

Hay varios tipos de los procesos de QA. El primer tipo se llama las pruebas
unitarias y consiste en probar por separado cada unidad de funionalidad de
sistema. Las puruebas unitarias permiten detecatar muchos errores de
funcionamiento incorrecto de los componentes del sistema y de esta manera
prevenir las horas de búsqueda de errors cuando el sistema ya está montado. Las
pruebas unitarias se basan en los requerimientos y en el diseño de arquitectuar
de bajo nivel. Las purebas unitarias aseguran que cada compomente del sistema
funciona correctamente. El segundo tipo de las pruebas se llama las pruebas de
integración. Las pruebas de integración verifican la funcionalidad de los
bloques enteros de componentes subiendo la estrucutra jerárquica de los
componentes. Las puebas de integración aseguran que los interfaces de contexión
entre los componentes de bajo nivel funcionan correctamente y los componentes
funcionan bein en su conjunto. Mientras que el proceso de desarrollo baja la
estructura jerárquica de los componente de un sistema dividiendo los problemas
complejos un unos más simples el proceso QA sube es estructura jerárquica de los
componentes verificacndo la funcionaliad de los bloques integrados pos los
componentes funcionales. El resultado de las actividades de QA es la solución
probada y verificada lista para el despliegue.

\Paragraph{Despliegue y configuración de las soluciones}

El despliegie y la configuración de la solución son los procesos de puesta en
marcha de la solución. Dependiendo de la infraestructura de hardware y la
arquitectura de software de alto nivel las actividades de despliege pueden
variar. En casos simples la solución precisa un servidor y todas las actividades
de despliege tienen lugar en una sola máquina. En casos más complejos el
despliege comprende varios servidores que dependen uno de otro.

Para agilizar el proceso de despliege la empresa elabora los scripts de
instalación y configuración. Los scripts de instalación y configuración son
los programas que se ejecuante en el entrono de pordución y preparan la
solución para su funcionamiento correcto en la producción. La configuración de
las soluciones es crucial para el correcto funcionamiento de la solución. La
configuración de la solución permite ajustar los parámetros de la solución a un
entrono concreto y optimizar el servicio que se presta. El resultado de las
actividades de despliege y configuración el el sistema funcionando correctamente
en producción.

\Paragraph{Mantenimient de las soluciones}

El mantenimiento de las soluciones es la fase más larga en el ciclo de vida de
la solución. Las tareas de mantenimiento comprenden la actualización de software
con los parches se seguridad, administración de equipos y servidores,
realización de las copias de seguridad, gestión de certificados digitales,
monitorización de la carga de CPU y de la carga de memoria.

Las tareas se mantenimiento de softaware tambien incluyes la implementación de
los cambios y las modificaciones de la lógica de funcionamiento de las
soluciones. Algunos cambios pequeños habitualmente no llevan coste pero los
cambios grandes pueden incluso convertirse en otros proyectos saparados con
todas las actividades que les corresponden. Es muy importante acordar con el
cliente las fechas de puesta en producción de los cambios pedidos por un lado y
asegurar que los cambios intoducidos no rompen la funcionalidad ateriormente
implementada. Para tal fin sirven muy bien los procesos de QA. Las
modificaciones de las soluciones suponen la ejecución de los procesos de
despliege y configuración. En estos momentos son de mucha utilidad los scripts
de despliege y configuración que automatizan los procesos y aseguran su
ejecución correcta. Los procesos de mantenimiento casi siempre son una fuente de
ingresos adicionales.

\subsection{Marketing}%{Marketing}

\startitemize
\item Present product and services
\item Advertising and promotions
\item Public relations
\item Sales force management
\item Customer contact management
\stopitemize

\subsection{Logística y distribución}%{Distributions and delivery}

\startitemize
\item Hand products and services
\item Distributor relationshps
\item Delivery systems
\item Invetory management
\stopitemize

La espresa se dedica a la producción de las soluciones IT para el negocio de los
clientes. El producto principal de la empresa es el software. El software es el
código fuente, los ficheros de configuración, los sistemas de pruebas
automatizadas y la documentación adjunta. El producto que produce la empresa se
guarda en formato electrónico en la memoria de un ordenador. Las soluciones de
la empresa son personalizadas y son diferentes para cada cliente. Con el
objetivo de agilizar el proceso de gestión de código fuente y los otros
fiecheros que forman parte de la solución en la empresa se utiliza un sistema de
gestión de versiones. El sistema de gesión de versiones se llama Git.

El sistema de gesitón de versiones tiene numerosas ventajas en comparación con
otros métodos de almacenamiento de código fuente. El sistema de gestión de
versiones garantiza la integridad y la consistencia del código fuente, permite
versionar el código fuente, lleva el historial de totas las modificaciónes que
se han echo en el código fuente, permite compartir de manera fácil el código
fuente con otras partes interesadas, gesióna las reglas de acceso y modificación
de código fuente, realiza las copia de seguridad del cógido fuente, permite el
trabajo colaborativo, soporta los modelos centralizados y descentralizados de
trabajo. Para agilizar el proceso de desarrollo del software, los procesos de QA
y los procesos de despliege y configuración la empresa utiliza el sistema de
gesitión de versiones. Para reducir los costes necesarios para mantener los
servicios del sistema de gesitión de versiones la empresa utiliza el servicio
público gratuito de sistema de control de versiones. El serviocio público se
llama github.com.

El proceso de distribución de la solución lista para el despliege consiste en
bajar el código fuene con los ficheros de configuración y los scripts de
instalación desde el servidor público al servidor donde va a estar alojado el
servicio en producción. El sistema de control de veriones Git tiene unos medios
potentes para tal fin lo cual automatiza el procoso y asegura la integridad y la
consistencia de los ficheros de código fuente. El proceso de distribución del
producto consiste en el copiado de los ficheros, la instalación del sistema y su
posterior configuración. Todo el proceso de destribución se gesiona desde un
sitio remoto (por ejemlo desde la oficina de la empresa) y implica ningún tipo
de gasto para la empresa. El sistema de distribución de las soluciones está
construido por la empresa utilizanod componentes comúnes. El sistema de
distribución de las soluciones tiene sus peculiaredades y detalles de
funcionamiento lo cual permite a la empresa gesitonar la distribución sin
preocuparse de que el mismo proceso puede realizar una persona agena al
desarrollo.

El servicio github.com ofrece un abánico de heramientas y solucionas para la
gestión de código fuente que la empresa utiliza para la gestión de inventario de
los proyectos en curso. Todas las versiones de todos los proyectos están
disponibles en cada momento. El historial de las modificaciónes refleca de
manera clara y cómoda la evolución del proyecto. Las estadística de código
fuente permiten evaluar diferetenes parámentro de gesitón de código y de las
modificaciones.

\subsection{Servicio al cliente}%{Customer service}

\startitemize
\item Products and services support
\item Customer support
\item Service fulfillment
\stopitemize

El servicio al clieten pretende resolver todos los problemas que surgen después
de la puesta en producción de las soluciones. El objetivo del servicio al
cliente es mantener al clieten satisfecho durante toda la vida de la solución.
El cliente reporta una incidencia o una petición de cambio a la empresa. La
empresa averigua las razones de la incidencia o el impacto del cambio que ha
pedido el cliente. En el caso cuando se detecta un error en el sistema la
empresa lo soluciona lo antes posible y vuelve a desplegar la solución
corregida. En el caso de que es un cambio puqueño de la lógica de funcionamiento
de la solución la empresa implenenta el cambio y vuelve a desplegar la solución.
Si el cambio es grande el proceos es diferente. Normalmente se habre un
subproyecto con todas las fases necesarias para garantizar la calidad del
sistema futuro. Los cambios grandes pueden convertirse en unos proyectos
separados y ser fuentes de ingresos separados.

El servicio al cliente tambien gestiona los casos cuando clieten pide una
información adicional sobre el funcionamiento de las soluciones como por ejemplo
los indicadores de fiablidad y los indicadores de disponibilidad de servicio,
los niveles de errores, las estadísticas de las peticiones, la minería de datos,
hasta los servicios de BI (Business Intelligence).

\subsection{Gestión de recursos y organización}%{Management}

\startitemize
\item Team, direction and leadership
\item Business planning
\item Goal setting
\item Market analysis
\item Strategy and tactics
\item Investor relations
\item Budgeting
\stopitemize

La empresa es una empresa de un solo empleado yo. Todas las decisiones
estratégicas y tácincias voy a tomar yo. La dirección de la empresa es enfocada
a la eficioneca, la innovación y la proactividad.

\subsection{Organización}%{Organization}

\startitemize
\item Resources structure
\item One person organization with outsorcing some activities (organigram)
\item Organizational model: pack
\item Pack, form follows function, divide and conquer, matrix
\item Responsibilites and procedures
\stopitemize

\subsection{Finanzas}%{Financial condition}

\startitemize
\item Asset control
\item Cash flow tracking
\item Customer builling
\item Accounts payable and receivable
\item Payroll management
\item Financial reporting
\item Tax accounting
\stopitemize

\chapter{Estrategia de empresa}%{Company strategy}

\section{Análisis DAFO}%{SWOT analysis}

\subsection{Industria, cliente, competencia}%{Industry, customer, competition}

\Comment{Opportunities and threats}

\subsection{Productos y servicios, capacidades de la empresa}
%{Products and services, business capabilities}

\Comment{Strengths and weaknesses}

\startitemize
\item Importance to business, company rate
\item Capitalize on strengths for opportunities, improve weaknesses, monitor
threats, eliminate weaknesses on threats
\stopitemize

\section{Modelo de negocio}%{Business model}

\Comment{How and when to make money, financial projection and timeline}

Asignación de precio

Hourly rate = (earnings + costs + profit) / (40h * 50w)

Project rate = project hours * hourly rate

\startitemize
\item Revenue = cost structure (fixed, variable) + profit margin
\item Timeline
\item Financial projection: profit margin = revenue - costs
\stopitemize

\Paragraph{Asignación de precio previsional de las soluciones}

Para calcular el precio previsional de las soluciones voy a partir del precio de
hora de mi trabajo y la cantidad de horas necesarias para entregar la solución
al cliente. Este método de asignación de precio de las soluciones parte del
margen que quiero obtener al dedicarme a este tipo de actividad. Para que las
soluciones tengan precio competitivo en el mercado el precio final habrá que
corregir teniendo en cuenta los precios del mercado y de la competencia. Por
otro lado hay que tener en cuenta los baremos de precios que estan dispuestos a
pagar los consumidores por las soluciones. De esta manera voy a aplicar el
método de asignación de precio a las soluciones que contempla los costes y el
margen que quiero obtener, los precios de mercado y los precios de la
competencia y los precios que están dispuestos a pagar los clientes.

Yo parto del supuesto que me guistaría ganar 50~000~euros brutos por año. Quiero
tener por lo menos 2~semanas de vacaciones al año. Quitando las 2~semanas de
vacaciones me quedan 50~semanas de trabajo. A la semana tengo que ganar
1~000~euros brutos. Quiero tener el fin de semana libre. Voy a tarbajar 5~días a
la semana. Al día tengo que ganar 200~euros brutos. Quiero trabajar 8~horas al
día. A la hora tengo que ganar 25~euros brutos. Resumiendo los cálculos quiero
trabajar 50~semanas al año y tener 2~semanas libres, quiero trabajas 5~días a la
semana y tener 2~días libres, quiero trabajar 8~horas al días. Por otro lado
quiero ganar 50~000~euros brutos al año con lo cual tengo que ganara 1~000~euros
a la semana, 200~euros al día, 25~euros a la hora. El precio bruto prvisional
por hora se presenta en la \in{tabla}[tab:precio-hora].

\placetable[here][tab:precio-hora]
{Precio bruto previsional por hora}
{\starttable[|l|l|]
\HL
\NC \bf Tiempo \NC \bf Dinero \NC\MR
\HL
\NC Año \NC 50 000 euros brutos \NC\MR
\NC Semana \NC 1 000 euros brutos \NC\MR
\NC Día \NC 200 euros brutos \NC\MR
\NC Hora \NC 25 euros brutos \NC\MR
\HL
\stoptable}

La empresa se va a dedicar al principio a los proyectos no muy grandes cuyo
plazo de desarrollo no debería superar 2~\endash~3 meses. Los proyectos
habituales deberían durar de 1 a 3 semanas. Es un plazo razonable teniendo en
cuenta las tecnologías innovadoras e eficientes y las metodologías ágiles que se
van a aplicar a los desarrollos. Teniendo en cuenta los plazos de desarrollo de
las soluciones se pueden evaluar los precios previsionales de las
soluciones. Los precios previsionales de las soluiones se presentan en la
\in{tabla}[tab:precios-soluciones].

\placetable[here][tab:precios-soluciones]
{Precios previsionales de las soluciones}
{\starttable[|l|l|]
\HL
\NC \bf Solución \NC \bf Precio \NC\MR
\HL
\NC 1 semana \NC 1 000 euros \NC\MR
\NC 2 semanas \NC 2 000 euros \NC\MR
\NC 3 semanas \NC 3 000 euros \NC\MR
\NC 1 mes \NC 4 000 euros \NC\MR
\NC 2 meses \NC 8 000 euros \NC\MR
\NC 3 meses \NC 12 000 euros \NC\MR
\HL
\stoptable}

\Paragraph{Estimación de ingresos}

Los ingresos principales vendrán de las ventas de las soluciones IT. Las ventas
de las soluciones IT será la fuente principal de ingresos. Se espera tener de 2
a 4 proyectos pequeños al mes, o de 1 a 2 proyectos grandes al mes. El proyecto
pequeño dura hasta 1 o 2 semanas. El proyecto grande dura 3~\endash~4 semanas o
más. Los ingresos estimados por la venta de las soluciones se podrían evaluar
mediante los datos presentados en la \in{tabla}[tab:precios-soluciones].

La otra fuente de ingresos es la consultoría IT. Este tipo de servicios de la
empresa consume menos tiempo pero es muy rentable. La consultoría IT puede
cobrarse por separado de otros servicios que presta la empresa o puede ser
incluida en los paquetes junto los las soluciones IT con el objetivo de mejor
satisfacer las necesidades de los clientes. Los ingresos por la consultoría IT
son bastante estables debido a que todas las soluciones precisan una consiltoría
IT previa. Es cierto que muchas veces los servicios de consultoría estarán
incluidos dentro del precio de la solición a entregar al cliente. Los ingresos
provenientes de la prestación de los servicios de las consultoria IT es más
difícil de valorar porque estos servicios son menos tangibles y dependen mucho
de cada caso específico pero una aproximación razonable podría ser basada en la
cantidad de tiempo que se emplea al prestar el servicio de consultoría IT
utilizando la \in{tabla}[tab:precio-hora].

Por otro lado la naturaleza de las soluciones IT supone el mantenimiento, las
actualizaciones y los cambios de funcionamiento de las soluciones. Los trabajos
de mantenimiento, la realización de las actualizaciones y la implementación de
los cambios generarán también los ingresos. Estos ingresos no var a ser muy
regulares y dependeran en su mayor parte de las necesidades y los cambios en la
lógica de negocio de los clientes.

Las fuentes de los ingresos en la empresa son tres. La primera fuente de ingreso
es la venta de las soluciones IT y es la principal. La segunda fuente de
ingreso es la venta de los servicios de consultoria IT. La tercera fuente de
ingreso es el mantenimiento, las actualizaciónes de softaware y la
implementación de los cambios en las soluciones de los clientes. Al mes la
empresa espera facturar 4~000~\endash~5~000 eruos brutos dedicandose a las
actividades anteriormente mencionadas. La estimación de ingresos se presenta en
la \in{tabla}[tab:estimacion-ingresos].

\placetable[here][tab:estimacion-ingresos]
{Estimación de ingresos}
{\starttable[|l|l|l|]
\HL
\NC \bf Concepto \NC \bf Tiempo \NC \bf Dinero \NC\MR
\HL
\NC Solución IT \NC 3 días \NC 600 euros brutos \NC\MR
\NC Solución IT \NC 2 semanas \NC 2 000 euros brutos \NC\MR
\NC Solución IT \NC 1 semana \NC 1 000 euros brutos \NC\MR
\NC Consultoria IT \NC 2 horas \NC 50 euros brutos \NC\MR
\NC Consultoria IT \NC 4 horas \NC 100 euros brutos \NC\MR
\NC Mantenimiento \NC 1 día \NC 200 euros brutos \NC\MR
\HL
\NC \use{2}{\hfill Total:} \NC 3 950 euros brutos \NC\MR
\stoptable}

\Paragraph{Estimación de gastos}

Los gastos variables comprenden los gastos de electricidad y son 20~euros al
mes. La electricidad es gasto variable porque se utiliza para alimentar el
portátil de desarrollo de las soluciones para el cliente. Si no hay los
proyectos en curso no se gasta la ecectricidad. Todo el software que se utiliza
en el desarrollo de las soluciones para los clientes es el software de código
abierto y no requiere la adqusición de la licencia. El echo de que no hace falta
adquirir la licencia de software reduce considerablemente los gastos variables
de cada solución.

Los gastos fijos son los gastos de las comunicaciones y los gastos de la
conexión a Internet. Los gastos de las comunicaciones suman 35~euros al mes.
Los gastos de las comunicaciones son los gastos fijos porque son independientes
de la actividad de la empresa. Toda la actividad de la empresa se desarrollará
en mi propio piso lo cual excluye los gastos de alquiler y mantenimiento de los
edificios donde se ubica la empresa.

Las aportizaciones comprenden el valor del portátil de desarrollo que cuesta
700~euros. La vida útil del portatil es de 7 años. La amortización anual resulta
100~euros al año.700~euros. La vida útil del portatil es de 7 años. La
amortización anual resulta 100~euros al año. La estimación de gastos se presenta
en la \in{tabla}[tab:estimacion-gastos].

\placetable[here][tab:estimacion-gastos]
{Estimación de gastos}
{\starttable[|l|l|]
\HL
\NC \use{2}{\bf Gastos variables} \NC\MR
\HL
\NC Electicidad \NC 20 euros/mes \NC\MR
\HL
\NC \use{2}{\bf Gastos fijos} \NC\MR
\HL
\NC Internet \NC 35 euros/mes \NC\MR
\HL
\NC \use{2}{\bf Amortizaciones} \NC\MR
\HL
\NC Portátil \NC 10 euros/mes \NC\MR
\HL
\NC \hfill Total: \NC 65 euros/mes \NC\MR
\HL
\stoptable}

\Paragraph{Beneficio previsional}

El beneficio previsional antes de intereses e impuestos se calcula restando los
gastos previsionales de los ingresos previsionales. El beneficio previsional
antes de intereses e impuestos se presenta en la
\in{tabla}[tab:beneficio-previsional].

\placetable[here][tab:beneficio-previsional]
{Beneficio previsional antes de intereses e impuestos}
{\starttable[|l|l|]
\HL
\NC Ingresos previsionales \NC 3 950 euros \NC\MR
\NC \endash\ Gastos previsionales \NC 65 euros \NC\MR
\NC = Beneficio previsional \NC 3 885 euros \NC\MR
\HL
\stoptable}

Entonces el beneficio previsional antes de intereses e impuestos mensual es
3~885~euros. Todos los valores de ingresos, gastos y beneficios son
previsionales y estimados. El modelo de negocio está basado en el proceso de
producción de software de calidad personalizado a cada cliente y su posterior
venta al cliente. Los servicios de consultoria IT y los servicios de
mantenimiento extenden el portfolio de servicios que ofrece la empresa para
satisfacer las necesidades del cliente. Los ingresos está basados en las horas
necesarias para elaborar la solución IT para el cliente o asesorar el cliente en
la selección de las tecnologias y métodos de construcción de software necesario
para reslover los problemas de clientes. Los gastos no son gradnes debido a que
en el proceso de desarrollo de software se emplea principalmente el propiedad
intelectual y conocimientos propios.

\section{Estrategia de crecimiento}%{Growth strategy}

\startitemize
\item New clients for existing products and services
\item New products and services for existing clients
\item New clients and new products and services
\stopitemize

La estrategia de crecimiento principal de la empresa es de creat productos
nuevos y conseguir los clientes nuevos. Este tipos de estrategias mejor encaja
para la empresa nueva que no tiene ni productos ni clienes todavia.

El enfoque de la empresa son las soluciones IT de calidad personalizadas
basadas en la innovación y la creatividad. En estas condiciones la estrategia
de crear los productos nuevos es la forma natural de dar la salida a las
ideas innovadors y creativas. Por otro lado los productos nuevos vienen con
características nuevas y son más eficientes en comparación con los productos
tradicionales con lo cual al estrategias de los productos nuevos atrae más
clientes nuevos. La misión de la empresa consiste en hacer las soluciones
personalizadas para cada cliente. Las soluciones personalizadas suponen la
creación de los productos únicos para cada cliente. El compromiso con la
innovación implica la búsqueda constente de nuevas formas de afrontar los
problemas y la creación de los nuevos productos. Para desarrollar las soluciones
de calidad hay que aplicar los métodos de mejora contínua ya que la calidad es
un proceso de la mejora contínua y la optimización de las soluciones.

Desarrollando los productos nuevos fortalece la posición de la empresa en el
mercado y coloca a la empresa en una pocisión de ventaja frente a la competencia
debido a la alta probabilidad de introducir los productos sustitutivos en el
mercado tradicional y así ganarse la clientela y las ventas. Las inversiones en
la innovación y la mejora contínua resultan en la creación de productos
sustitutivos y el cambio de las fuerzas en el mercado. Por otro lado trabajando
continuamente con las últimas tecnologías y aplicando las ideas innovadoras en
la producción a veces resulta en la creación de las industrias nuevas y la
creación de mercados nuevos previamente desconocidos. No se trata de creación de
falsas necesidades sino de las formas extraordinarias de satisfacer las
necesidades de los clientes. Entonces la intorducción de productos sustitutivos
y la posibilidad de creación de mercados nuevos son los argumentos importantes
para apostar por la estrategia de intoducción de productos nuevos para la
empresa.

La estrategia de consiguir los clientes nuevos es de grand importancia para la
empresa. Ya que la empresa es nueva todos los clientes van a ser nuevos. No
obstante la empresa sequirá apostando par la estrategia de conseguir más
clientes nuevos. La visión de la empresa es crear el valor para el cliente
mediante la innovación y la tecnologia. La visión de la empresa implica que el
objetivo principal de la empresa es la máxima satisfacción del cliente. En la
empresa el cliente está en el centro de toda la activida de la empresa. La
empresa pretende atraer a los nuevos clientes ofreciendoles los productos
neuvos, eficientes que mejor se adapan a sus necesidades. Por otro lado
al aplicar las tecnología de código abierto que no requieren la adquisición de
la licencia los costes de las soluciones se reducen considerablemente. El
uso de las tecnologias de código abeirto tambien mejora la calidad de las
soluciones y reduce los plazos de entrega.

Resumiendo la estrategia de la empresa consiste en introducción de los
productos nuevos con la posibilidad de crear los productos sustitutivos o
incluso crear las industrias nuevas. De otra parte la estrategia de empresa
va a estar dirigida a oferecer los productos nuevos a los clientes nuevos.
Para consiguir los clientes nuevos se crearán los productos de calidad, con
precios reucidos en los plazos de entrega cortos. Toda la dirección estrategica
de la empresa va es ser dirigida dentro del marco de la innovación y la mejora
contínua.

\chapter{Plan de marketing}%{Marketing plan}

%\section{Situación de mercado}%{Market situation}

\section{Segmentación de mercado}%{Market segmentation}

Muchas veces las empresas compran el software que es demasiado complejo y
costoso para la operativa y las necesidades que habitualmente suelen tener. El
mercado de software está dividido en 5 segmentos principales. Los segmentos de
mercado de software se distinguen por el nivel de complejidad y configuración de
sofware, las opciones funcionales y las opciones de integración con otros
sistemas. Mientras más complejo y más funcionalidad tiene el software más caro y
más costoso es desplegar y configurarlo. Por el contrario en el otro lado de
mercado se encuentras los segmentos con el software menos complejo con los
paquetes preconfigurados que satisfacen las necesidades de típicas empresas de
segmento. La funcionalidad y el número de usuarios son limitados lo cual se
rejleja en el precio de software.

\rightaligned{\Source{ref:software-market-overview}}
\placefigure[here][fig:software-market]{Situación de mercado}
  {\externalfigure[software-market-tier-chart.jpg][width=0.7\textwidth]}

\Paragraph{Enterprise software}

El software de este segmento de mercado es muy complejo y tiene muchas
prestaciones porque pertime ajustarse a cualquier necesidad de las grandes
empresas. El software es dificil de desplegar y configurar. Las empresas que
utilizan el software de esta categoría pertenecen a Fortune 500. Son las
empresas multinacionales con las estructuras matriciales. El software permite
configuraciónes locales y regionales que aseguran el cumplimiento de las
políticas vigentes en cada region. Sistemas de esta categoría ofrecen las
interfaces de conexión a cualquier otro sistema. Los principales proveedores en
este segmento de mercado son Oracle y SAP.

\Paragraph{Upper market software}

Los productos de este segmento de mercado son menos complejos en comaración con
el anterior pero todavía son bastante dificiles para desplegar y configurar. El
software es menos costoso. Las empresas que utilizan este software son muy
grandes pero no llegan al tamaño y complejidad de procesos de las empresas del
segmento anterior. Los principales proveedores en este segmento de mercado son
Infor, Microsoft Dynamics AX y muchos otros.

\Paragraph{Mid market software}

En este segmento de mercado compiten muchos proveedores ya que el número de
empresas que utilizan el software de esta categoría super el número de empresas
en los dos anteriores segmentos de mercado. El software es versatil y ajustable
a las necesidades diversas de las empresas pero la escala es menor en
comparación con los dos primero segmentos de mercado. En este segmento de
mercado se ofrecen los paquetes preconfigurados de fácil desplegue lo cual
permite reducir los costes de software. Los principales proveedores en este
segmento de mercado son Microsoft Dynamics NAV, Sage, Infor, SAP Business One.

\Paragraph{Lower market software}

Las empresas de este segmento de mercado han podido superar los límites del
último segmento de mercado pero sus dimenciones y complejidad no llegan el la
escala del segmento anterior. En este segmento de mercado se precisan las
soluciones escalables con relativamente copa configuración y bajo precio. Los
principales proveedores en este segmento de mercado son Sage MAS y Microsoft
Dynamics GP.

\Paragraph{Shrink wrap software market}

El software de este segmento de mercado no permite mucha configuración y muchos
usuarios en el sistem. Son versiones limitadas en su funcionalidad con el
objetivo de reducir los costes. Los desplegues son fáciles y no permiten mucha
flexibilidad a la hora de satisfacer las necesidades de las empresas. Sin
embargo este software es lo que precisan las pequeñas empresas. Los principales
proveedores de este segmento de mercado son Sage Simply Accouning e Intuit.

\subsection{Cliente objetivo}%{Target customer}



\subsection{Competencia}%{Competition}

La competencia de la empresa son otras empresas del sector que ofrecen las
soluciones IT dentro del modelo B2B. La diferencia más importante que sitúa
a la empresa en una posición de ventaja en comparación con la competencia es el
compromiso fuerte de la empresa con la innovación y la relación calidad/precio
superior.

\section{Metas y objetivos de marketing}%{Marketing goals and objectives}

\Paragraph{Crear una amplia cartera de clietens loyales}

La empresa ofrece las soluciones IT personalizadas para todos y cada uno de los
clientes. La soluciones IT personalizadas atraen los clientes porque con este
enfoque se eliminan las limitaciones e los inconvenientes de las soluciones
estrandarizadas. El cliente sabe que la empresa en cada momento de va a ofrecer
la mejor solución de alta calidad que está en conformidad con todos y cada uno
de los requerimientos y las necediades del cliente. La meta consisten en ampliar
la cartera de clientes al forecer les las soluciones personalizadas que mejor se
ajustan a las necesidades de clientes. El compormiso con la mejora continua y la
máxima atención a cada incidencia o cambio de requerimientos por parte del
cliente asegura la lealtad de los clientes que saben que cada una de sus
necesidades, cambios o actualizaciones va a ser antendiad y resuelta de la mejor
forma. Los objetivos de la meta de crear una amplia cartera de clietenes loyaes
son los siguientes:

\startitemize
\item Organizar 50 reuniones con los clientes potenciales. Hacer a conocer la
empresa al número máximo de los clientes potenciales. Conseguir las reuniones
con los clientes potenciales y presentarles las ventajas y los servicios únicos
de la empresa. Hacerles conocer la utilidad de los servicios que ofrece la
empresa
\item Enviar 200 correos electrónicos con las ofertas de soluciones IT de
interés a los clientes potenciales. Utilizar el correo electrónico y las redes
sociales para extender la red de contactos potenciales y difundir la misión, la
visión, los valores y el propósito principas de la empresa enviando material
informativo, las ofertas y convocatorias a las presentaciones de la empresa
\item Conseguir 10 nuevos clientes. Firmas los contratos con los clientes para
los proyectos ambiciosos insprando la confianza, la responsabilidad, el
profecionalizmo y la importancia de la satisfacción del cliente para la empresa.
Empezar los desarrollos de las soluciones y entregar los proudctos a los
clientes.
\stopitemize

\Paragraph{Incrementer los ingresos por las ventas de las soluciones IT y de
los servicios de consultoria IT}

La principal fuente de ingresos de la empresa es la venta de las soluciones IT y
de la venta de los servicios de consultoria IT. La meta de incrementar los
ignresos por las vetas pretende confirmar que la empresa aporta el valor a los
clietenes, es competitiva en el mercado, tiene una posición superior a la
competencia, ofrece un produción con el conjunto único de las características y
segue la línea de desarrollo sostenible. Para lograr la meta de incrementar los
ingrsos por las ventas de las soluciones IT y de los servcios de consultoria IT
se establecen los siguientes objetivos:

\startitemize
\item Vender 5 proyectos grandes de duración de 1 mes o más. La principal fuente
de ingresos de la empresa son los proyectos grandes se suponen un fuerte
compromiso con el cliente y un trabajo de largo plazo para la empresa. El
objetivo pretende vender por lo menos 5 proyectos grandes durante un año de
actividad de la emrepsa
\item Vender 10 proyectos pequeños de duración de 1 a 2 semanas. Los proyectos
pequeños presentan menor esfuerzo para la empresa pero sirven para un propósito
muy importante de captar un cliente nuevo y demostrarle la eficiencia y el
profecionalizmo con que la empresa gestióna los desarrollos. Los proyectos
pequeños es una oportunidad grande para la empresa demostrar que el prodocto
que ofrece tiene un único conjunto de características importantes para el
clieten y así asegurar la futura colaboracón con el clieten en los proyectos más
grandes. El objetivo pretende vender por lo menos 10 proyectos pequeños durante
un año de actividad de la empresa
\item Vender 5 proyectos de servicios de consultoría IT. Los servicios de
consultoria IT es un fuente de ingresos para la empresa por un lado y por el
otro es una forma de hacerse conocer al número mayor posible de los clientes
potenciales. Los servicios de consultoria IT publicitar los puntos fuertes de la
empresa y preparan a los clientes para la futura colaboración recíproca. El
objetivo pretende vender por lo menos 5 proyectos de consultoria IT durante un
año de actividad de la emrpesa
\stopitemize

\Paragraph{Incrementar el conocimiento de los servicios de IT consultoria}

Aparte de los servicios de diseño y desarrollo de las colusicones para los
clientes la empresa también orienta y sugiere a los clientes en cuanto a las
tecnologias y mejores formas a crear los sistemas informáticos de los clietes.
La meta consiste en incrementar el conocimiento de los servicios de IT
consultoria que son muy importantes para los clientes porque vale mucho más
pensar antes de empredner un desarrollo de un sistema que no es la que mejor
responde a las necesidades del cliente o un sistema que utilice una tecnología
menos adecuada para este tipo de sistemas. Las implicaciones de una decisión no
acertada repercuten en la fiablidad y la disponitilidad del sistema final, en
la calidad del sistema final, en los plazos de entrega y en los costes del
sistema final. La meta pretende ampliar los ingresos por parte de la consultoria
IT explicando a los clientes que contratando estos servicios los clientes
ahorran dinero que se podrían haber gastado liquidando los desperfectos del
sistema mal deseñada desde su principio. Los objetivos de la meta de incrementar
el conocimiento de los servicios de IT consultoria son los siguientes:

\startitemize
\item Ofrecer los servicios de IT consultoría a los clientes potenciales en cada
reunión. Presentar las ventajas de la consultoría IT para los clientes, su
implicación en los resultados finales de la solución. Intentar la contratación
de los servicios de consultoría IT por los clientes
\item Incluir en todas las ofertas de soluciones IT la información sobre los
servicios de IT consultoría. Promocionar los servicios de consultoría IT en
todos los materiales de soluciones IT. Hacer descuentos y paquetes de servicios
que mejor respondad a las necesidades de los clientes
\item Conseguir 5 nuevos proyectos de IT consultoría (añadir el valor a los
proyectos de las soluciones IT). Firmar por lo menos 5 contratos de consultoría
IT con los clientes. Ofrecerles el mejor posible servicio de orientación y
asesoramiento en el mundo de tecnologia y formas de construir los sistemas
informáticos
\stopitemize

\section{Posicionamiento}%{Positioning statement}

\Comment{Your business name + what makes your buisness unique and different
  + your market description}

\startitemize
\item My comppany provides custom-made quality IT solutions and consultancy
  to B2B customers
\item Business IT services company
\stopitemize

La empresa se posiciona en el mercado como {\bf La empresa de soluciones IT de
calidad personalizadas y los servicios de consultoria IT para las empresas
medianas y pequeñas de diversos tipos de actividad}.

La empresa ofrece las soluciones de calidad y personalizadas a los clientes.
Cada solucion se diseña y se desarrolla especificamente para un cliente teniendo
en cuenta las necesidades particulares de cada cliente. Le ventaja de las
soluciones personalizadas consise en ofrecer al cliente la solución diseñanada
especialmente para él que responde a todos los requerimientos y a todas las
necesidades del cliente concreto. Todas las soluciones pasan los procesos de QA
(Quality Assurance) que garantizan la calidad de la solución.

Los costes de las soluciones son bajos en comparación con otras alternativas del
mercado. La reducción de costes ha sido posible gracias a la utilización en el
desarrollo de software de los sistemas de código abierto que no requiere la
adqulición de la licencia. Los sistemas de código abierto son muy eficientes y
de alta calidad que agiliza el proceso de desarrollo, mejora la calidad del
producto final y reduce los costes considerablemente. Tanto los proceso internos
como el proceso de desarrollo de software en la empresa utliza exclusivametne
los sistemas de código abierto.

Los plazos de entrega de la soluciones de calidad personalizadas son muy cortos
en comparación con otras alternativas en el mercado. Los plazos cortos de
entrega han sido posibles gracias a la aplicación de últimas tecnologías y
métodos innovadores en el diseño y el desarrollo de software. Muchas etapas en
el desarrollo de los sistemas están automatizados lo cual permite reducir los
plazos de engrega y mentener la alta calidad de cada solución personalizada.

El valor y el producto único que ofrece la empresa a los clientes consiste en el
siguiente conjunto:
\startitemize
\item Las soluciones IT de alta calidad personalizadas
\item Los servicios de consultoria IT
\item Los precios de las soluciones IT y de los servicios de consultoria IT muy
bajos
\item Los plazos de entrega de las solucioens IT son muy cortos
\stopitemize
La empresa se posiciona en el mercado como un proveedores de servisios IT para
el negocio de los clientes. La empresa pertenece al modelo B2B (Business to
Busitess). Los principales clientes de la empresa son las compañías medianas y
pequenas de diversos tipos de actividad.

\section{Estrategias de marketing}%{Marketing strategies}

\Comment{Marketing mix}

\subsection{Estrategias de producto}%{Product strategies}

\startitemize
\item New uses for existing products
\item New products
\stopitemize

La estrategia de producto de la empresa promueve los productos nuevos para cada
cliente. Las soluciones IT son personalizadas para cada cliente para satisfacer
las neceidades del cliente de la mejor forma. Todos los productos de la empresa
utilizan los métodos innovadores de solución de problemas. La mejora contínua en
la estrategia de producto es muy imporatnte para la empresa porque es la forma
de ofrecer al cliente mayor valor construyendo los productos más innovadores y
más eficientes.

Ofrecer el producto de alta calidad es un objetivo importante de la estrategia
de producto de la empresa. Todos los productos de la empresa se posicionan como
los productos de alta calidad. Las soluciones de calidad responden a todos los
requerimientos y todas las necesidades del cliente de la manera más eficiente.

Los productos nuevos y personalizados para cada clieten, los productos de alta
calidad son las línes principales de la estrategias de producto de la empresa.
Esta estrategia de producto destaca en el mercado porque muchas empresas
alternativas ofrecen las soluciones estándar o las soluciones hechas con las
tecnologías de hace mucho tiempo que tienen las limitaciones considerables. La
estrategia de producto de la empresa proclama que las soluciones IT de la
empresa de ninguna manera no var a limitar los procesos de negocio de los
clientes sino que le va a dar una forma lógica y estructurada.

\subsection{Estrategias de precio}%{Pricing strategies}

\startitemize
\item Costs, value, profit
\item Positioning, competition
\stopitemize

La estrategia de precio posiciona a la empresa en un sector donde no hay muy
poca competencia. El sector es de las soluciones de alta calidad y de bajo
precio. La estartegia de precio de la empresa prentende cambiar el pensamiento
de cliente de que los productos de calidad valen mucho dinero. Reducir el precio
ha sido posible gracias a la utilización de métodos innovadores en el diseño y
el desarrollo del software y por la aplicación del software de código abieto.

Los costes principales de las soluciones IT están en las horas necesarias para
producir el producto final. El precio de hora de trabajo se presenta en la
\in{tabla}[tab:precio-hora]. La empresa utiliza el método de asignar los precios
a los productos basado en los costes y en los margenes. El hecho de que los
precios de los productos dependen solo de los precios de horas de trabajo
necesario para elaborar la solución hace el precio bastante bajo y muy
competitivo en el mercado. Los precios de las soluciones típicas de la empresa
se presentan en la \in{tabla}[tab:precios-soluciones]. Los precios de los
servicios de consultoria IT dependen solo de del precio de hora de trabajo.

Los servicios de mantenimiento de las soluciones normalmente se van a contratar
aparte del desarrollo de las soluciones IT. Los servicios de mentenimientos
están sujetos a los precios de hora de trabajo y dependen en mucha parte de los
precios de los proveedores de la infraestructura de hardware.

\subsection{Estrategias de distribución}%{Distribution strategies}

\startitemize
\item Distribution and delivery system
\item Distribution channels
\item Packaging
\stopitemize

La distribución de software la empresa realiza mediante la utilización del
servicio público del sistema de gesión de versiones github.com. El proceso de
distribución es totalmente automático gracias a los servicios y las herramientas
que provee github.com. La peculiaridade de distribución de cada solución
personalizada se autimatizan en los scripts de instalación y de configuración
que implementa la empresa para cada solución. La distribución del software se
realiza desde un sitio remoto que puede ser la oficina de la emrpesa o cualquier
otro ordenador.

Muchas veces la distribución de realiza desde el servidor de sistema de gesitón
de versiones al servidor de producción que esta bajo el control de la empresa.
En este caso la empresa tiene el control sobre el servicio que presta la
solución desarrollada. En otros casos el cliente prefiere alojar la solución en
su propio servidor y necesita el código fuente y los ficheros de configuración
junto con los scripts de instalación y de configuración. Dependiendo de las
condiciones del contrato con el cliente la empresa puede realizar las tareas de
instalación y configuración o delegarlas al equipo técnico del cliente.

\subsection{Estrategias de promoción}%{Promotion strategies}

\Comment{Interest}

\startitemize
\item Marketing communication
\stopitemize
Crear una amplia cartera de clietens loyales es la meta muy importante de la
empresa.  La mayor parte de las actividades de marketing va a ser dedicada a
conseguir la meta. Los objetivos para conseguir la meta de crear una amplia
cartera de clientes loyales son:

\startitemize
\item Organizar 50 reuniones con los clientes potenciales. Hacer a conocer la
empresa al número máximo de los clientes potenciales. Conseguir las reuniones
con los clientes potenciales y presentarles las ventajas y los servicios únicos
de la empresa. Hacerles conocer la utilidad de los servicios que ofrece la
empresa
\item Enviar 200 correos electrónicos con las ofertas de soluciones IT de
interés a los clientes potenciales. Utilizar el correo electrónico y las redes
sociales para extender la red de contactos potenciales y difundir la misión, la
visión, los valores y el propósito principas de la empresa enviando material
informativo, las ofertas y convocatorias a las presentaciones de la empresa
\item Conseguir 10 nuevos clientes. Firmas los contratos con los clientes para
los proyectos ambiciosos insprando la confianza, la responsabilidad, el
profecionalizmo y la importancia de la satisfacción del cliente para la empresa.
Empezar los desarrollos de las soluciones y entregar los proudctos a los
clientes.
\stopitemize

Para conseguir los objetivos marcados la estrategias de promotion de los
productos de la empresa se basa en explicar a los clietens el valor que aporta
la empresa a los clientes, en hacer a los clientes conocer las ventajas y las
características únicas de los productos que ofrece la empresa. En cada reunión
con los clietenes la empresa va a hacer el enfasis sobrea la calidad, la
fiabilidad, la disponibilidad y la flexibilidad de los productos. La empresa
incluso va a hacer un paso más y va a demostrar todas las características únicas
de las soluciones que ofrece en práctica para fidelizar al clieten y crear mayor
credibilidad de lo que dice la empresa.

La empresa hará las presentaciones de sus productos y servicios para los
clientes potenciales. En las presentaciónes se detallarán las vantajas y el
valor que aportan al cliente las soluciones de la emrpesa. Se haran las campañas
de envío de los correos electrónicos a los clientes potenciales con el objetivo
de darse a conocer a los clientes y transmitir a los clientes el valor que
aporta la emrpesa.

\subsection{Programa de ventas}%{Sales program}

\Comment{Purchases}

El programa de ventas se presenta en la \in{tabla}[tab:programa-ventas].

\placetable[here][tab:programa-ventas]
{Programa de ventas}
{\starttable[|l|r|]
\HL
\NC \bf Proyecto \NC \bf Ventas \NC\MR
\HL
\NC \use{2}{\hfill\bf Mayo \endash\ Julio\hfill} \NC\MR
\HL
\NC Solución IT de 1 mes \NC 4 000 euros \NC\MR
\NC Solución IT de 2 semanas \NC 2 000 euros \NC\MR
\NC Solución IT de 1 semana \NC 1 000 euros \NC\MR
\NC Consultoría IT 1 de día \NC 200 euros \NC\MR
\NC Solución IT de 2 semanas \NC 2 000 euros \NC\MR
\NC Solución IT de 1 semana \NC 1 000 euros \NC\MR
\NC Consultoría IT 1 de día \NC 200 euros \NC\MR
\NC \use{2}{\hfill\bf Agosto \endash\ Octubre\hfill} \NC\MR
\HL
\NC Solución IT de 2 meses \NC 8 000 euros \NC\MR
\NC Solución IT de 2 semanas \NC 2 000 euros \NC\MR
\NC Solución IT de 1 semana \NC 1 000 euros \NC\MR
\NC Solución IT de 1 semana \NC 1 000 euros \NC\MR
\NC Solución IT de 2 semanas \NC 2 000 euros \NC\MR
\NC Consultoría IT 1 de día \NC 200 euros \NC\MR
\NC \use{2}{\hfill\bf Noviembre \endash\ Enero\hfill} \NC\MR
\HL
\NC Solución IT de 1 mes \NC 4 000 euros \NC\MR
\NC Solución IT de 2 meses \NC 8 000 euros \NC\MR
\NC Solución IT de 2 semanas \NC 2 000 euros \NC\MR
\NC Solución IT de 1 semana \NC 1 000 euros \NC\MR
\NC Consultoría IT 2 de días \NC 400 euros \NC\MR
\NC \use{2}{\hfill\bf Febrero \endash\ Abril\hfill} \NC\MR
\HL
\NC Solución IT de 1 mes \NC 4 000 euros \NC\MR
\NC Solución IT de 1 semana \NC 1 000 euros \NC\MR
\NC Solución IT de 2 semanas \NC 2 000 euros \NC\MR
\NC Consultoría IT 2 de días \NC 400 euros \NC\MR
\NC Solución IT de 2 semanas \NC 2 000 euros \NC\MR
\NC Solución IT de 1 semana \NC 1 000 euros \NC\MR
\HL
\NC \hfill Total: \NC 50 400 euros \NC\MR
\HL
\stoptable}

\subsection{Servicio al cliente}%{Customer service}

\Comment{Satisfaction}

\startitemize
\item Reward frequent or large purchases
\item Enhance service quality
\stopitemize

Con el objetivo de mejorar la calidad de los productos y la satisfacción de los
clientes por las prestaciones de los servicios la empresa va a realizar la
medición de la satisfacción de los clientes. Después de la engrega de las
soluciones y un tiemo de la utilización de las soluciones la empresa mandendrá
una reunión con cada cliente para conocer su opition sobre los productos IT de
la empresa y sobre los servicios que la empresa había prestado al cliente. La
empresa medirá el nivel de satisfacción de cada cliente y en función de los
resultados se tomarán las medidas correctivas para mejorar en las actividades de
la empresa que precisan la mejora.

Para la empresa el cliente contento significa crecimiento y futuro desarrollo.
Las reuniones de medición de satisfacción son muy emportantes para la empresa
porque es la fuente de las ideas para mejorar la empresa en su totalidad. Por
otro lado la empresa analizará cada incidencia o error en las soluciones
desarrolladas y toma medidas correctivas para que situaciónes parecidas no
ocurrán en el futuro. Para la empresa es muy importante aprender de sus propios
errores y prevenir los en el futuro. Toda la crítica constructiva por parte de
los clientes se tomará como la oportunidad para mejorar la empresa. Las
reuniones con los clientes para saber su optión sobre los productos de la
emrpesa y el análisis de las incidencias reportadas son las políticas
principales de la empresa para emjorar el servicio al cliente.

\section{Presupuesto de marketing}%{Marketing budget}

\startitemize
\item Zero-based budget
\stopitemize

\chapter{Finanzas}%{Financial review}

\section{Cuenta de resultados previsional}%{Income statement projection}

\Comment{Ingresos, gastos y beneficios en un período (1 año)}

La cuenta de resultados previsional de la empresa para 1 año se presenta en
la \in{tabla}[tab:cuenta-resultados].

\placetable[here][tab:cuenta-resultados]
{Cuenta de resultados previsional}%{Income statement projection}
{\starttable[|l|r|]
\HL
\NC \bf Ingresos por ventas% Gross revenue (from Sales program)
  \NC \bf 50 400 euros \NC\MR
\HL
\NC Gastos variables% Goods costs
  \NC \NC\MR
\NC \hspace[big] Electricidad (12 meses por 20 euros) \NC - 240 euros \NC\MR
\NC \bf Margen bruto% Gross profit
  \NC \bf = 50 160 euros \NC\MR
\HL
\NC Gastos fijos% Operating expenses
  \NC \NC\MR
\NC \hspace[big] Internet (12 meses por 35 euros) \NC - 420 euros \NC\MR
\NC Amortizaciones% Depreciation expenses
  \NC \NC\MR
\NC \hspace[big] Portátil (12 meses por 10 euros) \NC - 120 euros \NC\MR
\NC \bf Beneficio antes de intereses e impuestos, BAII
  % Operating profit, EBIT
  \NC \bf = 49 620 euros \NC\MR
\HL
\NC Intereses (no hay financiación)% Interest income/expenses
  \NC - 0 euros \NC\MR
\NC \bf Beneficio antes de impuestos, BAI
  % Profit before taxes
  \NC \bf = 49 620 euros \NC\MR
\HL
\NC Impuestos (25\%)% Taxes
  \NC - 12 405 euros \NC\MR
\NC \bf Beneficio neto% Net profit
  \NC \bf = 37 215 euros \NC\MR
\HL
\stoptable}

Los ingresos vienen por las ventas de las soluciones IT, los servicios de
consultoria IT y los servicios de mantenimiento de las soluciones IT. Los gastos
variables de desarrollo de software es la electricidad. Los gastos fijos son las
comunicaciones e Internet. Las aporticaciones vienen de la compra del portátil
para el desarrollo y la gestión de la empresa. El precio del portátil es de
700~euros y la vida útil del portatil es de 7 años. Para calcualr las cuatas de
amortización del portátil se aplica el método lineal de amortización con lo cual
la cuota anual de amortización del portátil es de 100~euros/año y la cuota
mensual de amortización del portátil es de aproximadamente 10~euros/mes. No hay
intereses porque no es necesario ningún tipo de financiación para desarrollar
las actividades de la empresa. Los impuestos son 25\% de los beneficios antes de
impuestos porque le empresa pertenece al tipo de empresas pqueñas.

\section{Análisis económico previsional}%{Economic analysis}

\Paragraph{Punto de equilibrio (umbral de rentabilidad)}%{Break-even analysis}

\Comment{Volumen de ventas para cubrir los gastos\crlf
cuando el beneficio es cero}

El análisis de punto de equilibrio permite calcular el volumen de ventas
necesario para cubrir los gastos cuando el beneficio es cero. Se parte de que
los ingresos totales \m{IT} son iguales a los costes totales \m{CT}. Los
ingresos totales se presentan como el producto del precio unitario \m{PU} y la
cantidad de las unidades vendida \m{US}. \m{IT = PU \cdot US}. Los costes
totales se prsentan como los costes fijos totales \m{CFT} más el producto de los
costes variables unitarios \m{CVU} y la cantidad de las unidades vendidas.
\m{CT = CFT + CVU \cdot US}. La cantidad de las unidades vendidas se presenta a
continuación:

\startformula
IT = CT
\stopformula
\startformula
PU \cdot US = CFT + CVU \cdot US
\stopformula
\startformula
US \cdot (PU - CVU) = CFT
\stopformula
\startformula
US = \frac{CFT}{PU - CVU}
\stopformula
donde
\startlegend
\leg IT \\ Ingresos Totales \\ euros/mes \\
\leg CT \\ Costes Totales \\ euros/mes \\
\leg PU \\ Precio Unitario \\ euros/unidad \\
\leg US \\ Unidades vendidos \\ unidad \\
\leg CFT \\ Costes Fijos Totales \\ euros/mes \\
\leg CVU \\ Costes Variables Unitarios \\ euros/unidad \\
\stoplegend
Los costes fijos totales son los costes de las comunicaicones e Internet. El
precio unitario es el precio de una solución IT de 1~semana de duración. Los
costes variables unitarios son los costes de electricidad 20~euros/mes y
5~euros/semana.
\startfact
\fact Costes Fijos Totales \\ CFT \\ 35~euros/mes \\
\fact Precio Unitario \\ PU \\ 1000~euros/semana \\
\fact Costes Variables Unitarios \\ CVU \\ 5~euros/semana \\
\stopfact
La cantidad de las unidades vendidas que en este caso es la cantidad de
soluciones IT de 1~semana de duración vendidas es:
\startformula
US = \frac{CFT}{PU - CVU} = \frac{35}{1000 - 5} \approx 0.035
\stopformula
Convertimos el resultado a las horas a la semana necesarias para cubrir los
gastos:
\startformula
0.035~\cdot~5~días/sem~\cdot~8~horas/día \approx 1.5~horas/semana
\stopformula
El punto de equilibrio para la empresa es vender por lo menos 1.5~horas de
solucioens IT por la semana. Como se puede ver los gastos de la empresa son muy
pequeños lo cual significa que la empresa es muy rentable y la actividad de
empresa no supone un riesgo grande para llevarla al cabo.

\page[yes]
\section{Balance de situación previsional}%{Balance sheet projection}

\Comment{Contabilidad accural (activos, pasivos y patrimonio)\crlf
Situación partimonial en un momento (al final del año)}

El balance de situación previsional de la empresa para al fin del primer año se
presenta el la \in{tabla}[tab:balance-situacion].

\placetable[here][tab:balance-situacion]
{Balance de situación previsional}%{Balance sheet projection}
{\starttable[|lw(0.5\textwidth)|r|]
\HL
\NC \use{2}{\hfill\bf Activos\hfill} \NC\MR% Assets
\HL
\NC \use{2}{\bf Activos no corrientes\hfill} \NC\MR% Fixed assets
\NC \hspace[big] Capital social (mi aportación) \NC 3 000 euros \NC\MR
\NC \hspace[big] Portátil (precio de adqusición) \NC 700 euros \NC\MR
\NC \bf Total activos no corrientes: \NC \bf 3 700 euros \NC\MR
\NC \use{2}{\bf Activos corrientes\hfill} \NC\MR% Current assets
\NC \hspace[big] Clientes (venta de soluciones IT) \NC 50 400 euros \NC\MR
% from Sales program
\NC \bf Total activos corrientes: \NC \bf 50 400 euros \NC\MR
\NC \bf Total activos: \NC \bf 54 100 euros \NC\MR
\NC \use{2}{\hfill\bf Pasivos\hfill} \NC\MR% Liabilities
\HL
\NC \use{2}{\bf Pasivos no corrientes\hfill} \NC\MR% Current liabilities
\NC \bf Total pasivos no corrientes: \NC \bf - 0 euros \NC\MR
\NC \use{2}{\bf Pasivos corrientes\hfill} \NC\MR% Long-term liabilities
\NC \hspace[big] Electricidad (12 meses por 20 euros) \NC - 240 euros \NC\MR
\NC \hspace[big] Internet (12 meses por 35 euros) \NC - 420 euros \NC\MR
\NC \hspace[big] Amortizaciones (portátil) \NC - 120 euros \NC\MR
\NC \hspace[big] Impuestos (25\%) \NC - 12 405 euros \NC\MR
\NC \bf Total pasivos corrientes: \NC \bf - 13 185 euros \NC\MR
\NC \bf Total pasivos: \NC \bf - 13 185 euros \NC\MR
\NC \use{2}{\hfill\bf Patrimonio neto\hfill} \NC\MR% Equity
\HL
\NC \hspace[big] Capital (mi aportación) \NC 3 700 euros \NC\MR
\NC \hspace[big] Reservas (beneficios no distribuidos) \NC 37 215 euros \NC\MR
\NC \bf Total partimonio neto: \NC \bf 40 915 euros \NC\MR
\HL
\stoptable}

Los activos no corrientes se forman por el capital social y el portátil para el
desarrollo y la gestión de la empresa. El capital social de 3~000~euros es
aportado por mí y necesario por la ley para emprender la actividad empresarial.
El valor de adquisición de portátil es de 700~euros. La empresa no dispone de
edificios y no asquila los locales porque toda la actividad de empresa se
desarrolla desde el piso que está en mi propiedad. La empresa no tiene los
pasivos no corrientes porque la empresa no cuenta con ningún prestamo a largo
plazo. Los pasivos correntes se forman por la electricidad, las comunicaciones e
Internet, las amortización del portátil y los impuestos. El patrimonio neto se
desglosa en el capital que es mi aprotación para crear la empresa más el precio
de adquisición del portátil y las reservas que son los beneficios no
distribuidos. El patrimonio neto al final del primer año de actividad de la
empresa es bastante alto lo cual significa que la solvencia (a largo plazo) de
la empresa es muy buena.

\section{Análisis patrimonial y financiero previsional}%{Financial analysis}

\Paragraph{Liquidez}

Los activos corrientes \m{AC} son mucho mayores que los pasivos corrientes
\m{PC} y se cumple la expresción:
\startformula
AC \times 2 \geq PC
\stopformula
donde
\startlegend
\leg AC \\ Activos Corrientes \\ euros \\
\leg PC \\ Pasivos Corrientes \\ euros \\
\stoplegend
Con los valores actuales del balance de situación previsional tenemos:
\startfact
\fact Activos Corrientes \\ AC \\ 50~400~euros \\
\fact Pasivos Corrientes \\ AC \\ 13~185~euros \\
\stopfact
\startformula
50~400~euros \times 2 \geq 13~185~euros
\stopformula
La liquidez de la empresa (la capacidad de la empresa de responder a los
compromisos de pago a corto plazo) está muy bien.

\Paragraph{Solvencia}

El patrimonio neto \m{PN} tiene que ser aproximadamente 40\% de los activos
totales \m{AT}.
\startformula
PN \approx 40\%~AT
\stopformula
donde
\startlegend
\leg PN \\ Patrimonio Neto \\ euros \\
\leg AT \\ Activos Totales \\ euros \\
\stoplegend
Con los valores actuales del balance de situación previsional tenemos:
\startfact
\fact Patrimonio Neto \\ PN \\ 40~415~euros \\
\fact Activos Totales \\ AT \\ 54~100~euros \\
\stopfact
\startformula
40~415~euros \approx 76\%~54~100~euros
\stopformula
El patrimonio neto asegura la solvencia de la empresa (la capacidad de la
empresa de responder a los compromisos de pago a largo plazo).

\Paragraph{Fondo de maniobra}%{Working capital}

El fondo de maniobra \m{FM} de la empresa es la diferencia entre los activos
corrientes \m{AC} y los pasivos corrientes \m{PC}.
\startformula
FM = AC - PC
\stopformula
donde
\startlegend
\leg AC \\ Activos Corrientes \\ euros \\
\leg PC \\ Pasivos Corrientes \\ euros \\
\stoplegend
Con los valores actuales del balance de situación previsional tenemos:
\startfact
\fact Activos Corrientes \\ AC \\ 50~400~euros \\
\fact Pasivos Corrientes \\ AC \\ 13~185~euros \\
\stopfact
\startformula
FM = 50~400~euros - 13~185~euros \gt 0
\stopformula
La situación del equilibrio financiero en la empresa es normal porque el fondo
de maniobra es positivo. La empresa es capaz de hacer frente a los compromisos
de de pago en las fechas de sus vencimientos.

\Paragraph{Ratio de liquidez}%{Liquidity ratio}

El ratio de liquidez \m{RL} de la empresa es la relacion entre los activos
corrientes \m{AC} y los pasivos corrientes \m{PC}:
\startformula
RL = \frac{AC}{PC} \approx 2
\stopformula
donde
\startlegend
\leg AC \\ Activos Corrientes \\ euros \\
\leg PC \\ Pasivos Corrientes \\ euros \\
\stoplegend
Con los valores actuales del balance de situación previsional tenemos:
\startfact
\fact Activos Corrientes \\ AC \\ 50~400~euros \\
\fact Pasivos Corrientes \\ AC \\ 13~185~euros \\
\stopfact
\startformula
RL = \frac{50~400~euros}{13~185~euros} \approx 3.8
\stopformula
El ratio de liquidez de la empresa es bastante alto lo cual significa que la
empresa no tiene problemas al hacer frente a los compromisos de pago en las
fechas de sus vencimientos.

\Paragraph{Ratio de endeudamiento}%{Leverage ratio}

El ratio de endeudamiento \m{RE} es la relación entre los pasovos totales \m{PT}
y los activos totales \m{AT}:
\startformula
RE = \frac{PT}{AT} \approx 0.6
\stopformula
donde
\startlegend
\leg PT \\ Pasivos Totales \\ euros \\
\leg AT \\ Activos Totales \\ euros \\
\stoplegend
Con los valores actuales del balance de situación previsional tenemos:
\startfact
\fact Pasivos Totales \\ PT \\ 13~185~euros \\
\fact Activos Totales \\ AT \\ 54~100~euros \\
\stopfact
\startformula
RE = \frac{13~185~euros}{54~100~euros} \approx 0.24
\stopformula
El ratio de endeudamiento es muy pequeño porque la empresa no hace uso de los
instrumentos financieros y no financia su actividad principal lo cual significa
que la empresa tiene bastante autonomia financiera.

\section{Análisis de rentabilidad}%{Profitability analysis}

El margen \m{M} es la relación entre el beneficio \m{B} y las ventas \m{V}. Para
aumentar el margen hay que subir el precio de venta de los productos aplicando
la estrategia de diferenciación de producto.
\startformula
M = \frac{B}{V}
\stopformula
donde
\startlegend
\leg B \\ BAII o Beneficio Neto \\ euros \\
\leg V \\ Ventas \\ euros \\
\stoplegend

La rotación de activos \m{RA} es la relación entre las ventas \m{V} y los
activos totales \m{AT}. Para aumentar la rotación de activos hay que vender más
productos aplicando la estrategias de liderazgo en costes.
\startformula
RA = \frac{V}{AT}
\stopformula
donde
\startlegend
\leg V \\ Ventas \\ euros \\
\leg AT \\ Activos Totales \\ euros \\
\stoplegend

El apalancamiento financiero \m{AF} el la relación entre los activos totales
\m{AT} y el patrimonio neto \m{PN}.
\startformula
AF = \frac{AT}{PN}
\stopformula
donde
\startlegend
\leg AT \\ Activos Totales \\ euros \\
\leg PN \\ Patrimonio Neto \\ euros \\
\stoplegend
El apalancamiento financiero positivo implica aumentra la deuda y disminuir el
patrimonio neto lo cual aumenta la rentabilidad financiera. Hay que mentener el
equilibrio entre la capitalización (el ratio de endeudamiento) que supone
aumentar el patrimonio neto y la rentabilidad financiera que supone disminuir el
patrimonio neto. Hay que encontrar el volumen óptimo del patrimonio neto.

\Paragraph{Rentabilidad económica (rendimiento de activos)}
%{Economic profitability}

La rentabilidad económica (rendimiento de activos) \m{RE} es la relación entre
el beneficios antes de intereses e impuestos \m{BAII} y los activos totales
\m{AT}:
\startformula
RE = \frac{BAII}{AT} = \frac{BAII}{V} \cdot \frac{V}{AT} = M \cdot RA
\stopformula
donde
\startlegend
\leg BAII \\ Beneficios Antes de Intereses e Impuestos \\ euros \\
\leg AT \\ Activos Totales \\ euros \\
\leg V \\ Ventas \\ euros \\
\leg M \\ Margen \\ \\
\leg RA \\ Rotación de Activos \\ \\
\stoplegend
Con los valores actuales de la cuenta de resultados previsional y del balance de
situación previsional tenemos:
\startfact
\fact Beneficios Antes de Intereses e Impuestos \\ BAII \\ 49~620~euros \\
\fact Activos Totales \\ AT \\ 54~100~euros \\
\stopfact
\startformula
RE = \frac{49~620~euros}{54~100~euros} \approx 0.92
\stopformula
La rentabilidad económica es bastante alta lo cual significa que el negocio
economicamente es rentable. Para aumentar la rentabilidad económica hay que
subir el precio de venta de los productos y aumentar las ventas de los
productos.

\Paragraph{Rentabilidad financiera (rentabilidad)}%{Financial profitability}

La rentabilidad financiera (rentabilidad) \m{RF} es la relación entre el
beneficio neto \m{BN} y el patrimonio neto \m{PN}:
\startformula
RF = \frac{BN}{PN} = \frac{BN}{V} \cdot \frac{V}{AT} \cdot \frac{AT}{PN}
  = M \cdot RA \cdot AF
\stopformula
donde
\startlegend
\leg BN \\ Beneficio Neto \\ euros \\
\leg PN \\ Patrimonio Neto \\ euros \\
\leg AT \\ Activos Totales \\ euros \\
\leg V \\ Ventas \\ euros \\
\leg M \\ Margen \\ \\
\leg RA \\ Rotación de Activos \\ \\
\leg AF \\ Apalancamiento financiero \\ \\
\stoplegend
Con los valores actuales de la cuenta de resultados previsional y del balance de
situación previsional tenemos:
\startfact
\fact Beneficio Neto \\ BN \\ 37~215~euros \\
\fact Patrimonio Neto \\ PN \\ 49~915~euros \\
\stopfact
\startformula
RF = \frac{37~215~euros}{49~915~euros} \approx 0.75
\stopformula
La rentabilidad financiera es bastante alta lo cual significa que desde el punto
de vista financiero el negocio es rentable. Para aumentar la rentabilidad
financiera hay que subir el precio de venta de los productos, aumentar las
ventas de los productos, aumentar la deuda y disminuir el patrimonio neto
aprovechando de apalancamiento financiero positivo.

\page[yes]
\section{Estado de flujos de tesoreria previsional}
%{Cash-flow statement projection}

\Comment{Contabilidad de efectivo en un período\crlf
Cobros y pagos tan pronto como se producen\crlf
ni un segundo antes}

El stado de flujos de tesoreria previsional para la empresa se presenta en la
\in{tabla}[tab:flujos-tesoreria].

\placetable[here][tab:flujos-tesoreria]
{Estado de flujos de tesoreria previsional}%{Cash-flow statement projection}
{\starttable[|l|r|]
\HL
\NC \bf Saldo inicial \NC \bf 0 euros \NC\MR
\HL
\NC \use{2}{\bf Actividades ordinarias\hfill} \NC\MR% Operations
\NC Cobros \NC \NC\MR
\NC \hspace[big] Clientes (venta de soluciones IT) \NC 50 400 euros \NC\MR
% from Sales program
\NC Pagos \NC \NC\MR
\NC \hspace[big] Electricidad (12 meses por 20 eruos) \NC - 240 euros \NC\MR
\NC \hspace[big] Internet (12 meses por 35 euros) \NC - 420 euros \NC\MR
\NC \hspace[big] Impuestos (25\%) \NC - 12 405 euros \NC\MR
\NC \bf Saldo actividades ordinarias \NC \bf 37 335 euros \NC\MR
\HL
\NC \use{2}{\bf Actividades de inversión\hfill} \NC\MR% Investment activity
\NC \bf Saldo actividades de inversión \NC \bf 0 euros \NC\MR
\HL
\NC \use{2}{\bf Actividades de financiación\hfill} \NC\MR% Financing activity
\NC \bf Saldo actividades de financiación \NC \bf 0 euros \NC\MR
\HL
\NC \bf Saldo final \NC \bf 37 335 euros \NC\MR
\HL
\stoptable}

En la empresa las diferencias entre el saldo inicial y el saldo final viene dado
solo pos las actividades ordinarias. Los cobros de las actividades ordinarias
constan las ventas de las soluciones IT, los servicios de consultoria IT y el
mantenimiento de las solucioens IT. Los pagos de las actividades ordinarias se
forman por la electricidad, las comunicaciones e Internet y los impuestos. La
empresa no tiene ni las actividades de inversión y las actividade de
financiación por la actividad de la empresa en el primer año no precisa las
inversiones grandes y respectivamente la financiación. No obstante para
asegurar el crecimiento de la empresa en el futuro se harán las inversiones en
las direcciones potenciales de crecimiento de la empresa con su financiación
correspondiente ya sea la autofinanciación u otros instrumentos financieros.
El saldo final al final del primer año demuestra que la capacidad de la emrepsa
de generar la riqueza es bastante alta lo cual es buen argumento para emprender
el negocio.

\section{Presupuesto}%{Master budget}

\Comment{Estimaciones futuras de gastos de la empresa}

El presupuesto de la empresa se presenta en la \in{tabla}[tab:presupuesto].

\placetable[here][tab:presupuesto]
{Presupuesto}
{\starttable[|lw(0.5\textwidth)|r|]
\HL
\NC Gastos variables \NC \NC\MR
\NC \hspace[big] Electricidad (12 meses por 20 euros) \NC 240 euros \NC\MR
\NC Gastos fijos \NC \NC\MR
\NC \hspace[big] Internet (12 meses por 35 euros) \NC 420 euros \NC\MR
\NC \bf Total presupuesto \NC \bf 660 euros \NC\MR
\HL
\stoptable}

\chapter{Plan de acción}%{Action plan}

\startitemize
\item Actions (goals, objectives, strategies)
\item Responsabilities
\item Timetable (priorities)
\item Control
\stopitemize

\startalignment[middle]
\bf Crear una amplia cartera de clietens loyales
\stopalignment
\midaligned{\starttable[|lp(0.25\textwidth)|lp(0.5\textwidth)|lp(0.15\textwidth)|lp(0.1\textwidth)|]
\HL
\NC \bf Objetivo \NC \bf Acción \NC \bf Fecha \NC \bf Resp. \NC\MR
\HL
\NC Organizar 50 reuniones con los clientes potenciales
  \NC \startitemize[packed]
  \item Crear la lista de los clientes potenciales
  \item Establecer los contactos con los clientes potenciales
  \item Preparar las reuniones con los clientes potenciales
  \item Liderar las reuniones con los clientes potenciales
  \item Vender los productos a lso clientes potenciales
  \stopitemize
  \NC 30/04/2016 \NC Yo \NC\MR
\NC Enviar 200 correos electrónicos con las ofertas de soluciones IT de interés
  a los clientes potenciales
  \NC \startitemize[packed]
  \item Analizar los clientes potenciales y sus necesidades
  \item Crear las ofertas de interés y enviarlas a los clientes
  \item Analizar las respuestas de los clientes y adaptar las ofertas
  \stopitemize
  \NC 30/04/2016 \NC Yo \NC\MR
\NC Conseguir 10 nuevos clientes
  \NC \startitemize[packed]
  \item Analizar los nuevos clietes y sus necesidades
  \item Prepara los contratos para los clientes nuevos
  \item Firmar los contratos con los clientes nuevos
  \stopitemize
  \NC 30/04/2016 \NC Yo \NC\MR
\stoptable}

\startalignment[middle]
\bf Incrementer los ingresos por las ventas de las soluciones IT y de los
servicios de consultoria IT
\stopalignment
\midaligned{\starttable[|lp(0.25\textwidth)|lp(0.5\textwidth)|lp(0.15\textwidth)|lp(0.1\textwidth)|]
\HL
\NC \bf Objetivo \NC \bf Acción \NC \bf Fecha \NC \bf Resp. \NC\MR
\HL
\NC Vender 5 proyectos grandes de duración de 1 mes o más
  \NC \startitemize[packed]
  \item Buscar los clientes con proyectos grandes
  \item Hacer las ofertas de las soluciones IT
  \item Firmar los contratos de desarrollo de las soluciones IT
  \item Entregar las soluciones IT a los clientes
  \stopitemize
  \NC 30/04/2016 \NC Yo \NC\MR
\NC Vender 10 proyectos pequeños de duración de 1 a 2 semanas
  \NC \startitemize[packed]
  \item Buscar los clientes con proyectos pequeños
  \item Hacer las ofertas de las soluciones IT
  \item Firmar los contratos de desarrollo de las soluciones IT
  \item Entregar las soluciones IT a los clientes
  \stopitemize
  \NC 30/04/2016 \NC Yo \NC\MR
\NC Vender 5 proyectos de servicios de consultoría IT
  \NC \startitemize[packed]
  \item Buscar los clientes que necesitan los servicios de consultoria IT
  \item Establecer los contactos con los clietens
  \item Ofrecer a los clientes los servicios de consultoria IT
  \item Prestar a los clientes los servicios de consultoria IT
  \stopitemize
  \NC 30/04/2016 \NC Yo \NC\MR
\stoptable}

\startalignment[middle]
\bf Utilizar la última tecnologia en las soluciones
\stopalignment
\midaligned{\starttable[|lp(0.25\textwidth)|lp(0.5\textwidth)|lp(0.15\textwidth)|lp(0.1\textwidth)|]
\HL
\NC \bf Objetivo \NC \bf Acción \NC \bf Fecha \NC \bf Resp. \NC\MR
\HL
\NC Elegir la nueva tecnología más apropiada para cada solución
  \NC \startitemize[packed]
  \item Investigar las tecnologías nuevas relevantes para el problema en
  questión
  \item Evalura cada una de las tecnologías y elegir la más adecuada
  \stopitemize
  \NC 30/04/2016 \NC Yo \NC\MR
\NC Aplicar efectivamente la tecnología elegida en cada solución
  \NC \startitemize[packed]
  \item Aprender la nueva tecnología elegida
  \item Entender los contextos de aplicación de la nueva tecnología, sus
  ventajas e inconvenientes
  \item Utilizar la nueva tecnología en los contextons correspondientes
  \item Medir la eficiencia de la aplicación de la nueva tecnología elegida
  \stopitemize
  \NC 30/04/2016 \NC Yo \NC\MR
\NC Incrementar la calidad de las soluciones en 10\%
  \NC \startitemize[packed]
  \item Analizar las incidencias en las soluciones y los resultados de las
  reuniones con los clientes sobre la satisfacción de los clieintes
  \item Determinar puntos de mejora comúnes
  \item Decidir los métodos de mejora de calidad de las soluciones
  \item Aplicar los métodos de mejora para eliminar los desperfectos
  \stopitemize
  \NC 30/04/2016 \NC Yo \NC\MR
\NC Reducir los plazos de entrega de las soluciones en 10\%
  \NC \startitemize[packed]
  \item Analizar la distribución de tiempo entre las etapas de la cadena de
  valor
  \item Identificar las etapas en los cuales se gasta mucho tiempo y que
  tienen potencial para reducir el tiempo necesario
  \item Decidir los métodos de reducción de tiempo
  \item Aplciar los métidos de reducción de tiempo
  \stopitemize
  \NC 30/04/2016 \NC Yo \NC\MR
\stoptable}

\startalignment[middle]
\bf Establecer los procesos de QA (Quality Assurance) para las soluciones
\stopalignment
\midaligned{\starttable[|lp(0.25\textwidth)|lp(0.5\textwidth)|lp(0.15\textwidth)|lp(0.1\textwidth)|]
\HL
\NC \bf Objetivo \NC \bf Acción \NC \bf Fecha \NC \bf Resp. \NC\MR
\HL
\NC Establecer los más apropiados procesos de QA
  \NC \startitemize[packed]
  \item Analizar la arquitectura y la funcionalidad de las soluciones con el
  objetivo de determinar las partes críticas
  \item Identificar los criterios más importante que determinan la calidad de
  las soluciones
  \item Establecer los controles más apropiados para garantizar la calidad de
  las soluciones
  \stopitemize
  \NC 30/04/2016 \NC Yo \NC\MR
\NC Elegir el más adecuado framework de pruebas unitarias y de pruebas de
  integración
  \NC \startitemize[packed]
  \item Analizar los frameworks de pruebas unitarias y de integración
  \item Elegir el framework más apropiado
  \stopitemize
  \NC 30/04/2016 \NC Yo \NC\MR
\NC Aplicar los procesos y los frameworks en cada solución
  \NC \startitemize[packed]
  \item Relacionar los controles establecidos de procesos de QA con los métodos
  de control que proporcionan los frameworks elegidos
  \item Implementar los controles establecidos en terminos de los frameworks
  \item Utilizar los procesos de QA impelementados
  \stopitemize
  \NC 30/04/2016 \NC Yo \NC\MR
\NC Incrementar la calidad de las soluciones en 10\%
  \NC \startitemize[packed]
  \item Analizar los puntos débiles de las solucioens
  \item Implementar los procesos QA para eleminar los desperfectos
  \item Aplicar los procesos QA implementados
  \stopitemize
  \NC 30/04/2016 \NC Yo \NC\MR
\stoptable}

\startalignment[middle]
\bf Incrementar el conocimiento de los servicios de IT consultoria
\stopalignment
\midaligned{\starttable[|lp(0.25\textwidth)|lp(0.5\textwidth)|lp(0.15\textwidth)|lp(0.1\textwidth)|]
\HL
\NC \bf Objetivo \NC \bf Acción \NC \bf Fecha \NC \bf Resp. \NC\MR
\HL
\NC Ofrecer los servicios de IT consultoría a los clientes potenciales en cada
  reunión
  \NC \startitemize[packed]
  \item Analizar a los clietnes potenciales y estudiar sus necesidades
  \item Prepara las oferatas de los servicios de consultoria IT para cada
  cliente
  \item Ofrecer las propuestas de los servicios de consultoría IT a cada cliente
  \item Analizar la respuesta del cliente
  \stopitemize
  \NC 30/04/2016 \NC Yo \NC\MR
\NC Incluir en todas las ofertas de soluciones IT la información sobre los
  servicios de IT consultoría
  \NC \startitemize[packed]
  \item Preparar las ofertas de consultoría IT para los clientes
  \item Adjuntar las ofertas de consultoría IT a las ofertas de soluciones IT
  \item Analizar las respuesta y la reacción de los clientes
  \stopitemize
  \NC 30/04/2016 \NC Yo \NC\MR
\NC Conseguir 5 nuevos proyectos de IT consultoría
  \NC \startitemize[packed]
  \item Analizar las respuesta y la reacción de los clientes a las ofertas de
  los servicios de consultoria IT
  \item Adaptar las ofertas de los servicios de consultoria IT a las necesidades
  de los clientes
  \item Conseguir por lo menos 5 nuevos proyectos de consultoria IT
  \stopitemize
  \NC 30/04/2016 \NC Yo \NC\MR
\stoptable}

%\chapter{Plan de contingencia}%{Contingency plan}
%\Comment{What if ...?}

\chapter{Referencias}%{References}

\useURL[url:18-trends-in-app-dev]
  [http://www.billchamberlin.com/top-18-trends-in-application-software-development-for-2014/][]
  [Bill Chamberlin \endash\ Top 18 Trends in Application Software Development for 2014]
\useURL[url:5-it-industry-trends]
  [http://www.goabacus.com/index.php?page=5-growing-it-industry-trends-and-developments][]
  [ABACUS IT \endash\ 5 Growing IT Industry Trends \& Developments]
\useURL[url:3-software-dev-trends]
  [http://www.cybercoders.com/insights/3-big-software-development-trends-to-watch-in-2014/][]
  [Cyber Coders \endash\ 3 Big Software Development Trends to Watch in 2014]
\useURL[url:software-market-overview]
  [http://www.softresources.com/resource-room/software-market-overview/][]
  [Soft Resources \endash\ Software Market Overview]
\useURL[url:b2b-market-differences]
  [http://www.ronbrauner.com/11-differences-between-b2b-b2c-marketing/][]
  [Ron Brauner \endash\ The Differences Between B2B \& B2C Marketing]

\startitemize[n]
\item[ref:18-trends-in-app-dev] \from[url:18-trends-in-app-dev]
\item[ref:5-it-industry-trends] \from[url:5-it-industry-trends]
\item[ref:3-software-dev-trends] \from[url:3-software-dev-trends]
\item[ref:software-market-overview] \from[url:software-market-overview]
\item[ref:b2b-market-differences] \from[url:b2b-market-differences]
\stopitemize

\stopbodymatter
\stoptext
