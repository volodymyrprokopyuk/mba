\environment layout

% - business environment (industry, customers, competition)
%   - company overview (mission, vision, values, goals, objectives)
%     - company strategy (SWOT, business model, growth strategy)
%       - company description (products, business capabilities)
%       - marketing plan
%       - financial review
%       - action plan
%       - contingency plan
%       - executive summary

%\usemodule[fnt-10]
%\ShowCompleteFont{name:fontawesome}{20pt}{1}

\starttext

\startfrontmatter

\midaligned{Escuela de Administración de Empresas}\blank[14*big]
\midaligned{Plan de empresa}\blank[2*big]
\midaligned{Volodymyr Prokopyuk}
\midaligned{volodymyrprokopyuk@gmail}
\vfill\midaligned{Madrid}
\midaligned{\currentdate}

\completecontent
\stopfrontmatter

\startbodymatter

\chapter{Resumen ejecutivo}%{Executive summary}

\Comment{Key points, clear, brief}

\section{Elevator speech}

\startitemize
\item What business you are in
\item What is your industry and target customer
\item What are your products and services (features)
\item What is your customer value and benefits
\item What sets your business apart (uniqueness)
\item What is your competition
\stopitemize

\section{Modelo de negocio}%{Business model}

\section{Estrategia de crecimiento}%{Growth strategy}

\chapter{Breve descripcón de la empresa}%{Company overview}

\section{Idea de negocio}%{Business idea}

\Comment{Business model summary}

Custom-made quality IT solutions and consultancy

\section{Misión}%{Mission}

\Comment{Who you are, what you do}

Constant innovation in custom-made quality IT solutions and consultancy

\Paragraph{Innovación constante en soluciones IT personalizadas de calidad y IT
consultoria}

La empresa se crea con la intención de ser un lider en la aplicación de
soluciones innovadoras y creativas al diseño y desarrollo de las soluciones a
los clientes. La empresa pretende promover la innovación y mejora continua en
todos los niveles de su actividad desde la organización y gestión de los
procesos internos hasta las relaciónes con los clientes y elaboración del
producto final. La espiración de la empresa está en transmitir a los clientes
el espiritu de innovación y creatividad como los pilares principales en los que
se basa toda la actividad de la empresa y enfoque prinipal en busqueda de las
soluciones y la resolución de los problemas.

La empresa se construye como una emprese de servicios IT para los negocios de
distinto índole. La actividad de empresa principalmente va a estar dirigida a
dar las soluciones IT a otras empresas utilizando del modelo de interacción B2B
(Business To Business). La dos principales ramas de actividad son el diseño y el
desarrollo de productos IT y la consultoria IT. El diseño y el desarrollo de
productos IT comprende las aplicaciones que implementan la lógica de negocio de
los clientes, automiatizan sus procesos internos, los sistemas de integración
con terceros, sitios y aplicaciones Web y las aplicaciones para los móviles. La
consultoría IT se dedica a orientar y asesorar a los clientes para elegir la
mejor solución y diseñar el sistema más adecuado según las necesidades de cada
cliente en particular.

En todas las actividades de la empresa la innovación es el enfoque más
importante porque gracias a la innovación se pueden encontrar las soluciónes
óptimas y nuevas formas de afrontar los problemas. Cada solución echa por la
empresa reflejará el espiritu creativo e inovador en todas sus facetas. Es
porque cada proyecto va a ser personalizado con la seleción de métodos más
adequados en cada caso. Es importante que la innovación sea constante porque
el mundo de tecnología cambia muy rápido y para ofrecer a los clientes el
mejor servicios es necesario constantemente actualizarse.

La mejora continua es otro pilar de la actividad de empresa. Cada solución va a
ser fiable y va a satisfacer todas las necesidades de cliente. La satisfacción
del cliente es el mayor objetivo de la empresa. Las soluciones se van a
desarrollar en el marco de mejora contunua ya que solo las relaciones a largo
plazo aseguran el éxtito para la empresa y el desarrollo sostenible de la
empresa.

IT consultoría pretende orientar a los clientes en el mundo de tecnologia de
informaicón y sugerirles las óptimas soluciones y métodos de afrontar los
problemas que surgen y satisfacer las necesidades. Una vez elegido la solución
que mejor se adapta a las necesidades de los clientes se procedera a su
implementación e implantación en el cliente.

\section{Visión}%{Vision}

\Comment{Precise set of words that defines business highest aspiration you
hope to achieve, enduring purpose of the business}

Create client value through innovation and IT

\Paragraph{Crear el valor para el cliente mediante la innovación y la
tecnología}

La aspiración más alta de la empresa consiste en ser el líder en la aplicación
de la innovación y la creatividad en las soluciones tecnológicas para los
clientes. La innovación constante es lo que quia toda la actividad de la
empresa. El apostar por la innovación constante asegura el desarrollo sostenible
de la empresa y aumenta el valor aportado a los clientes. Las soluciones
basadas en la innovación y la creatividad aportan más valor a los clientes
sobre todo en el campo de la tecnología de información.

La visión de aportar valor a los clientes mediante la innovación y la tecnología
garantiza la sostenibilidad y interes futuro de los clientes en la empresa. La
visión de la empresa pretende construir las relaciones duraderas, las relaciones
de confianza y las relaciones a largo plazo con los clientes. La satisfacción de
los clientes el lo más importente en la actividad de la empresa. Las tecnologías
aplicadas en cada solución se cambiarán con el paso de tiempo pero el compromiso
con la innovación, la mejora continua y la satisfación del cliente perdurarán el
el tiempo garantizando en futuro de la empresa y el futuro de sus clientes.

\section{Valores}{Values}

\Comment{Principles to guide your business activities, preferable mode of
conduct}

\startitemize
\item Client value creation
\item Commitment to customer satisfaction
\item Commitment to innovation and excellence
\item Commitment to quality and continuous improvement
\item Passion for what I do
\item Build open and honest relationships
\item Commitment to sustainability and respansibility
\stopitemize

\section{Metas}%{Goals}

\Comment{Where to go and when to get there}

\startitemize
\item Constant innovation \AwRArrowLong\
Use ultimate technology in the solutions
\item Custom-made IT solutions \AwRArrowLong\
Create broad and loyal clientele
\item Quality IT solutions \AwRArrowLong\
Establish quality assurance processes for the solutions
\item Custom-made quality consultancy \AwRArrowLong\
Increase IT consultancy service awareness
\stopitemize

\Paragraph{Innovación constante \AwRArrowLong\ Utilizar la última tecnologia en
las soluciones}

La innovación constante supote la utilización de la última tecnología en las
soluciones para los clientes. La meta consiste en estar siempre actualizado en
el campo de las novedades de tecnología y aplicar la tecnologia la más adecuada
en cada caso específico. La aplicación de la tecnología más adecuada para cada
solución aumenta el valor aportado al cliente por parte de la empresa. Al
aplicar la innovación en cada solución el cliente se beneficiará de mejor
calidad de los clientes, de la fiabilidad y la disponibilidad más alta de la
solución, de los plazos de entrega más cortos y de mayor satisfacción total.

\Paragraph{Soluciones IT personalizadas \AwRArrowLong\ Crear una amplia cartera
de clietens loyales}

La empresa ofrece las soluciones IT personalizadas para todos y cada uno de los
clientes. La soluciones IT personalizadas atraen los clientes porque con este
enfoque se eliminan las limitaciones e los inconvenientes de las soluciones
estrandarizadas. El cliente sabe que la empresa en cada momento de va a ofrecer
la mejor solución de alta calidad que está en conformidad con todos y cada uno
de los requerimientos y las necediades del cliente. La meta consisten en ampliar
la cartera de clientes al forecer les las soluciones personalizadas que mejor se
ajustan a las necesidades de clientes. El compormiso con la mejora continua y la
máxima atención a cada incidencia o cambio de requerimientos por parte del
cliente asegura la lealtad de los clientes que saben que cada una de sus
necesidades, cambios o actualizaciones va a ser antendiad y resuelta de la mejor
forma.

\Paragraph{Soluciones IT de calidad \AwRArrowLong\ Establecer los procesos de QA
(Quality Assurance) para las soluciones}

El compromiso de la empresa con la calidad de soluciones a entregar supone el
establecimiento de los procesos de QA en todas las etapas de la elaboración de
las soluciones y en todas las actividades de empresa en general. Los procesos de
QA garantizan la conformidad del funcionaniento de los sistemas desarrollados
con los requerimientos previamente acordados con los clientes. Los procesos de
QA asegurand que las actualizaciones de software inducidos por los cambios de
los requerimientos solo implementas estos cambios y no influyen de forma
destructiva en la funcionalidad anteriormente desarrollada. Los procesos de QA
permiten mejorar la eficiencias del procesos de desarrollo, reducir plazos de
entrega y mejorar la imagen de la empresa como un proveedores de soluciones
fiables. El establecimiento de los procesos de calidad mejora la eficacia y la
eficiencia de todas las actividades de empresa en general.

\Paragraph{Servicios personalizados de calidad de consultoria \AwRArrowLong\
Incrementar el conocimiento de los servicios de IT consultoria}

Aparte de los servicios de diseño y desarrollo de las colusicones para los
clientes la empresa también orienta y sugiere a los clientes en cuanto a las
tecnologias y mejores formas a crear los sistemas informáticos de los clietes.
La meta consiste en incrementar el conocimiento de los servicios de IT
consultoria que son muy importantes para los clientes porque vale mucho más
pensar antes de empredner un desarrollo de un sistema que no es la que mejor
responde a las necesidades del cliente o un sistema que utilice una tecnología
menos adecuada para este tipo de sistemas. Las implicaciones de una decisión no
acertada repercuten en la fiablidad y la disponitilidad del sistema final, en
la calidad del sistema final, en los plazos de entrega y en los costes del
sistema final. La meta pretende ampliar los ingresos por parte de la consultoria
IT explicando a los clientes que contratando estos servicios los clientes
ahorran dinero que se podrían haber gastado liquidando los desperfectos del
sistema mal deseñada desde su principio.

\section{Objetivos}%{Objectives}

\Comment{How to get there, specific steps to reach your goals, what you must
do and when}

\startitemize
\item Use ultimate technology in the solutions, decrease time to market,
reduce operational costs (innovation goal - new ways to improve)
  \startitemize
  \item Select the most appropriate new technology for each solution
  \item Apply effectively selected techonology in each solution
  \item Increase solution quality and reduce solution delivery time by 10\%
  \stopitemize
\item Create broad and loyal clientele, find new markets/clients (day-to-day
work - increase every day effectiveness)
  \startitemize
  \item Organize 20 meetings with potential clients
  \item Send 100 emails with business offerings to potential clients
  \item Get 5 new clients
  \stopitemize
\item Establish quality assurance processes for the solutions, improve
customer satisfaction (development goal - skills, experience)
  \startitemize
  \item Select the most appropriate testing framework
  \item Apply selected testing framework in each solution
  \item Increase solution quality by 10\%
  \stopitemize
\item Establish IT consultancy service awareness, expand product/service
lines, generate new sorces of revenue (problem-solving goal - address
specific challenges)
  \startitemize
  \item Offer IT consultancy services to potential clients in each meeting
  \item Include in all business offerings emails information about IT
  consultancy services
  \item Get 3 new IT consultancy projects (added value to the solution)
  \stopitemize
\stopitemize

\Paragraph{Utilizar la última tecnologia en las soluciones}

\startitemize
\item Elegir la nueva tecnología más apropiada para cada solución. Antes de
empezar a diseñar y desarrollar cada nuevo proyecto familiarizarse y elegir
la tecnologia que mejor solucione los problemas de la solución y así congribuye
a la calidad del producto, los plazos de engrega y a la satisfacción de cliente
\item Aplicar efectivamente la tecnología elegida en cada solución. Controlar la
aplicación correcta de la tecnología elegida, analizar los resultados y medir
las mejoras consiguidas por utilizar la nueva tecnología. Prevenir los usos de
tecnologia elegida que no dan los resultados esperados
\item Incrementar la calidad de las soluciones en 10\%. Aumentar la satisfacción
de los clientes con las soluciones engregadas, reducir el número de incidencias,
mejorar la estabilidad, la fiabilidad y las disponibilidad de las soluciones
\item Reducir los plazos de entrega de las soluciones en 10\%. Aprovechando las
nuevas tecnologias y la aplicación de proceso de automatización reducir los
plazos de entrega y tiempo necesario para ajustar y adaptar la solución a las
necesidades del cliente
\stopitemize

\Paragraph{Crear una amplia cartera de clietens loyales}

\startitemize
\item Organizar 50 reuniones con los clientes potenciales. Hacer a conocer la
empresa al número máximo de los clientes potenciales. Conseguir las reuniones
con los clientes potenciales y presentarles las ventajas y los servicios únicos
de la empresa. Hacerles conocer la utilidad de los servicios que ofrece la
empresa
\item Enviar 200 correos electrónicos con las ofertas de soluciones IT de
interés a los clientes potenciales. Utilizar el correo electrónico y las redes
sociales para extender la red de contactos potenciales y difundir la misión, la
visión, los valores y el propósito principas de la empresa enviando material
informativo, las ofertas y convocatorias a las presentaciones de la empresa
\item Conseguir 10 nuevos clientes. Firmas los contratos con los clientes para
los proyectos ambiciosos insprando la confianza, la responsabilidad, el
profecionalizmo y la importancia de la satisfacción del cliente para la empresa.
Empezar los desarrollos de las soluciones y entregar los proudctos a los
clientes.
\stopitemize

\Paragraph{Establecer los procesos de QA (Quality Assurance) para las
soluciones}

\startitemize
\item Establecer los más apropiados procesos de QA. Determinar los criterios de
funcionamiento correcto de la empresa y de la elaboración deseada de las
soluciones IT. Esteplecer los controles y los procesos QA que controlarán las
actividades de la empresa y las etapas del desarrollo de las soluciones para los
clientes indicando en cada momento las posición de la empresa o la solución
respecto al objetivo
\item Elegir el más adecuado framework de pruebas unitarias y de pruebas de
integración. Incorporar el uso de los frameworks de pruebas en el flujo de
desarrollo de las soluciones. Determinar los aspectos que se podrían mejorar y
utilizar los automatizmos adecuados para agilizar el proceso de desarrollo
\item Aplicar los procesos y los frameworks en cada solución. Implantar en la
infraestructura de desarrollo los sistemas de pruebas unitarias y de pruebas de
integración. Controlar los resultados de las pruebas y hacer las correciones
correspondientes cuando procede
\item Incrementar la calidad de las soluciones en 10\%. Mejorar la conformidad
de las soluciones entregadas con los requerimiento de los clientes. Anticipar
los cambios futuros de modificación o ampliación de la fucionalidad. Reduric
el número de incidencias y fallos de los sistemas entregados. Asesorar las
decisiones con errores potenciales típicos y subir el nivel de satisfacción de
los clientes
\stopitemize

\Paragraph{Incrementar el conocimiento de los servicios de IT consultoria}

\startitemize
\item Ofrecer los servicios de IT consultoría a los clientes potenciales en cada
reunión. Presentar las ventajas de la consultoría IT para los clientes, su
implicación en los resultados finales de la solución. Intentar la contratación
de los servicios de consultoría IT por los clientes
\item Incluir en todas las ofertas de soluciones IT la información sobre los
servicios de IT consultoría. Promocionar los servicios de consultoría IT en
todos los materiales de soluciones IT. Hacer descuentos y paquetes de servicios
que mejor respondad a las necesidades de los clientes
\item Consiguir 5 nuevos proyectos de IT consultoría (añadir el valor a los
proyectos de las soluciones IT). Firmar por lo menos 5 contratos de consultoría
IT con los clientes. Ofrecerles el mejor posible servicio de orientación y
asesoramiento en el mundo de tecnologia y formas de construir los sistemas
informáticos
\stopitemize

\chapter{Entorno}%{Business environment}

\Comment{Opportunities, threats}

\section{Industria}%{Industry}

\startitemize
\item Information technology (IT) industry
\stopitemize

\subsection{Tendencias}%{Trends}

\Comment{Economy, technologies, regulations}

\Paragraph{Virtualización/IaaS (Infrastructure as a Service)}
\hfill\Source{ref:5-it-industry-trends}

En los últimos años han tenido mucho éxito las tecnologías de virtualización de
hardware para ofrecer la infraestructura y la plataforma de ejecución para las
aplicaciones. La virtualización permite reducir costes de la infraestructura
mediante devisión y reparto de las prestaciones de un servidor físico entre
multiples servidores virtuales. La tecnologia ofrece los medios de asignación de
las prestaciones y de las prioridades de servicio entres los servidores
virtuales.

La tecnología de virtualización mejora la disponibilidad y la fiabilidad de la
infraestructura de hardware ofreciendo posibilidad de realojar en tiempo real
una máquina virtual de un servidor físico donde se ha encontrado un problema a
otro servidor físico disponible sin interrupción de servicio al cliente.

La virtualización facilita la forma de aumento incremental de pontencia
necesaria de cálculo añadiendo o quitando las máquinas virtuales bajo demanda
para satisfacer las necesidades de la carga de sistema en cada momento. El
enfoque de aumento incremental permite reducir el número de máquinas virtuales
en los momentos de carga baja y aumentorlo solo cuando haya necesidad.

La última tencologia de virtualización hace un paso más y virtualiza no los
sistemas operativos sino las aplicaciones concretas. Esta tecnología se llama
contenerización y es aún más eficiente en comparación con la virtualización
tradicional. Contenerización virtualiza y aisla las aplicaciones dentro del
mismo servidor virtual y aprovecha al máximo los recursos de servidor.

Tanto virtualizacion como contenerización presentan un interes grande para las
empresas porque permiten reducir los costes de la infraestructura de sistemas de
información, mejorar la disponibilidad y la fiabilidad de sistemas, escalar las
prestaciones segun demanda.

\Paragraph{Nube/SaaS (Software as a Service)}
\hfill\Source{ref:5-it-industry-trends}

La nube presenta una alternativa muy atractiva al modelo tradocional de mentener
un centro de proceso de datos por parte de la empresa. El modelo de la nube
supone prestación de servicio tanto de hardware como de software por parte de
proveedor. El cliente cuenta con un servidor fiable, disponible y seguro
funcionando 24/7.

Las tecnologías de virtualización han hecho posible el servicio en la
nube. Habitualmente un servidor en la nube tiene menor coste en comapración con
el mantenimiento de un servidor físico por parte de la empresa. La nube es una
opción barata y fiable para las empresas pequeñas y medianas ya que permite
externalizar los servicios de infraestructura de hardware, la red de
comunicaciones y la administración de sistemas operativos centrandose en las
tareas de propio negocio.

Los proveedores de servicios en la nube han echo un paso más y ofrecen a los
clientes no solo los servidores virtuales aptos para servir cualquier tipo de
aplicaciones sino las aplicaciones en sí ya desplegadas y configuradas para el
cliente concreto. Esta estrategia de externalizar el mantenimiento y la
administración de las aplicaciones minimiza los costes de infraestructura de
software para las empresas.

La selección de las aplicaciones disponibles en la nube satisface las
necesidades del ciclo completo de funcionamiento de una empresa desde el
análisis del entorno y la fijación de los objetivos, la elaboración de las
estrategias, la planificación, la ejecución de los planes hasta el control y la
toma de medidas correctivas. Las aplicaciones más comunes ofrecidos en la nube
son las siguientes:

\startitemize
\item ERP (Enterprise Resource Planning)
\item PM (Project Management)
\item CRM (Customer Relationship Management)
\item e-Commerce (Comercio electrónico)
\item BI (Business Intelligence)
\item HR (Human Resources)
\stopitemize

\Paragraph{Software de código abierto}
\hfill\Source{ref:3-software-dev-trends}

El papel de software de código abierto sigue creciendo en la infraestructura de
sistemas de información tanto en las empresas pequeñas y medidas como en las
grandes. El software de código abierto ofrece muchas ventajas tanto para los
proveedores como para los consumidores de los productos de software.

Para los consumidores la ventaja principal consiste en reducción considerable de
costes de licencias en comparación con la adquisición del software propietario.
Por otro lado el software de código abierto es más fiable y más seguro porque lo
revisan y corrigen conginuamente muchos ingenieros de sofware altamente
cualifacados no por dinero sino por vocación y su buena voluntad a contribuir a
proyectos ambiciosos y grandes. Otra ventaja que aprovechan los consumidores de
software de código abierto es el soporte del producto y resolución de
incidencias. La pregunta o una incidencia puede ser resuelta en pocos días por
los profecionales de todo el mundo que tienen ganas de ayudar y explicar los
detalles de funcionamiento de su producto.

Para los proveedores de software de código abierto las ventajas consisten en la
oportunidad de influir en la evolución de productos y contribuir a la innovación
de tecnologias de informaicón. Muchas veces su trabajo es esponsorizado por las
grandes empresas que tienen interes en el desarrollo de la tecnologia concreta.
A veces las grandes empresas contribuyen los módulos a los proyectos de código
abierto y de esta manera promocionan las tecnologias de código abierto.

Últimamente muchos proyectos con código cerrado han sido publicados para ganar
el mercado y mejorar su posición frente a la competencia. El modelo de código
abierto ha demostrado su viabilidad y sostenibilidad en las tecnologías de
informaicón y seguirá creciendo.

\Paragraph{Dispositivos móviles}
\hfill\Source{ref:5-it-industry-trends}

Cada vez más accesos a Internet se hacen desde los dispositivos móviles. En 2014
el trafico total de acceso a Internet generado por los teléfonos móviles y los
táblets ha superado el trafico total de acceso a internet generado por los PCs.

Cada vez más los empresarios y los hombres de negocio prefieren un teléfono
móvil o un táblet a un portatil o PC. La nube, las comunicaciones de alta
velocidad y las redes sociales contribuyen al proceso de transformación en los
dispositivos de acceso a Internet de portatil/PC a móvil/táblet.

Para una empresa es muy importante estar presente en el ámbito de dispositivos
móviles con el fin de ofrecer el mejor servicio a sus clientes a traves de
todos los canales de acceso a los servicios. En el desarrollo de las
aplicaciones hay que hacer el enfoque en las plataformas de dispositivos
móviles debido a su aceptación y uso universal.

\Paragraph{Adopción de HTML5 y CSS3}
\hfill\Source{ref:18-trends-in-app-dev}

Desde hace unos años tiene lugar el dilema con que se enfrentan muchas empresas
que ofrecen servicios en la Web tanto para los navegadores de PCs como para los
dispositivos móviles. El dilema consiste en que para ofrecer las mejores
experiencias para los usuarios de los dispositivos móviles hay que desarrollar
las aplicaciones nativas para todas las plataformas que se necesita soportar
(iOS, Androdid). Este enfoque supote un coste elevado para las empresas de
desarrollar y mantener la aplicación en varias plataformas. Otro enconveniente
de este método es la falta de consistencia de contenido y de experiencia al
usuario debido a las diferencias en las plataformas de dispositivos móviles.
Otra alternativa supote utilización de una única tecnología para servir la
aplicación en todas las paltaformas. En este caso se utiliza la tecnologia
HTML5 con CSS3 y JavaScript. El inconventiente que anteriormente dominaba en
este ámbito era la experiencia visual limitada que podía ofrecer HTML5 en
aquel momento.

La ventaja de HTML5 con CSS3 consiste en que esta tecnologia es un estandar que
soportan todos los navegadores. HTML5 con CSS3 está en continuo desarrollo
aplicando los métodos más innovadores y probados en la industria de tecnologia
de información. HTML5 asegura la consistencia en todas las plataformas de
dispositivos móviles. Las empresas tienen que desarrollar y mantener solo una
única aplicación que serve todo el mercado. Últimamente las experiencias
visuales que ofrece HTML5 se comparan con las experiencias visuales de las
aplicaciones nativas lo cual nivela ambas tecnologias en cuanto y las
experiencias visuales. Mientras que la ventaja de una única aplicaciones para
todo el mercado apunta a favor de HTML5.

Un argumento más a favor de HTML5 con CSS3 y JavaScript radica en que Facebook
ha desarrollado un framework para desarrollo de aplicaciones nativas para iOS y
Android utilizando HTML5 y JavaScript. La tecnología se llama React.js Native.
La publicación del proyecto como software de código abierto demuestra un alto
interes en promoción de HTML5 por parte de las grandes compañías como la
tecnología para el desarrollo de las aplicaciones para dispositivos móviles.

\Paragraph{Modelo asíncrono de ejecución en el servidor/cliente}
\hfill\Source{ref:3-software-dev-trends}

Con el aumento de trafico en Internet la carga a los servidores Web ha crecido
mucho. Ha surgido la necesidad de incrementar el número de servidores para
satisfacer la demanda creciente de clientes. El modelo de ejecución en los
servidores tradicionales está basado en hilos. Cada petición requiere el acceso
a subsistemas relativamente lentas tales como el sistema de ficheros, las
bases de datos relacionales, el acceso a Web APIs de terceros. Estos acceso
lentos bloquean hilos en el servidor Web consumiendo los recursos limitados del
servidor. Cuando el número de peticiones y su frequencia es grande el modelo de
ejecución síncrona no es aceptable porque no escala bien.

Para resolver el problema de muchas peticiones concurrentes con el mínimo de
recursos disponibles en los úlitmos años se ha aplicado el modelo asíncrono de
ejecución en el servidor. El modelo asícrono de ejecución no bloquea la ejeción
en el servidor Web mientras se espara la respuesta de sistema de ficheros o la
respuesta de la bases de datos o la respuesta de Web API, sino que se crea una
función que se va a llamar cuando la respuesta esté lista para su procesamiento
posterior y el servidor Web sigue serviendo más peticiones.

El modelo asícrono de ejecución ha sido la clave de éxito del servidor Web
NGINX. Muchas empresas de todo tipo han podido reducir considerablemente sus
granjas de servidores Web tradicionales sustituyendolos por servidores Web
NGINX. El servidor Web NGINX está dominando el mercado de servidores Web. Otro
ejemplo de modelo asíncrono de ejecución es el framework Node.js. Node.js es
un entorno de entrada/salida asíncrono que utiliza JavaScript para implementar
la funcionalidad de aplicaciones en la parte de servidor.

En la parte de cliente el modelo asíncrono de ejecución dominaba desde su
invento. JavaScript es el lenguage de programación de la Web. Con la
introducción de Node.js en la parte de servidor JavaScript se ha convertido
un un lenguage de programación universal de la Web tanto en la parte de cliente
como en la parte de servidor. Las aplicaciones web más innovadores hoy en día
son isomorficas. Las aplicaciones isomorficas se pueden ejecutar tanto en el
servidor como en el cliente. Si la peticion al servidor Web se realiza desde un
cliente con capacidades de procesamiento de JavaScript (por ejemplo un navegador
Web) el servidor responde inmediatemente al cliente con la aplicación que se va
a ejecutar en el cliente omitiendo la carga de ejecutar la aplicación en el
servidor Web. Este tipo de aplicaciones se llaman aplicaciones SPA (Single Page
Application). Si por el contrario el servidor recibe la peticion desde un
cliente sin capacidad de procesamiento de JavaScript (por ejemplo un robot de
una máquina de búsqueda) el servidor ejecuta la aplicación y responde al cliente
con el HTML ya procesado. Este modelo de funcionamiento de servidor Web se llama
SEO (Search Engine Optimization) y permite servir el contenido ya procesado a
los clientes sin capacidad de procesamiento de JavaScript.

El modelo asíncrono de ejecución es la evlución natural de proceso de
interacción entre distintos agentes en Internet ya que resuelve los problemas de
eficiencia y escalado y además es más cercado a la conversasión natural.
JavaScript se ha convertido en un lenguage universal de la Web y su naturaleza
asíncrona encaja perfectamente en el modelo asíncrono de ejecución.

\subsection{Fuerzas competitivas básicas}%{Porter forces}

\subsection{Barreras de entrada}%{Entry barriers}

\startitemize
\item Capital cost
\item Economy of scale
\item Customer loyalty
\item Distribution system
\item Organiztion
\item Raw materials
\item New technology
\item Regulations
\item Patents
\stopitemize

\section{Clientes}%{Customers}

\Comment{B2B}

\subsection{Segmentación de clientes}%{Customer segmentation}

\subsection{Perfil de cliente}%{Customer profile}

\startitemize
\item Geographics (where they are)
\item Demographics (who they are) (age, income, marital status, education level)
\item Behavioral (how/why they buy)
\stopitemize

Customers:
\startitemize
\item Small and medium businesses of any kind
\stopitemize

Needs:
\startitemize
\item Benefit from IT advantages for business
\item Automate production and administration processes
\item Control production and administration processes
\item Innovate production and administration processes
\stopitemize

\subsection{Perfil de cliente ideal}%{Ideal customer profile}

\section{Competencia}%{Competition}

\startitemize
\item Understand competition
\item Distinguish yourself (advantage over competition)
\item Unique product/service (quality, reliability, flexibility, short
delivery time)
\stopitemize

\chapter{Descripción de la empresa}%{Company description}

\Comment{Strengths, weaknesses}

\section{Productos y servicios}%{Products and services}

\Comment{Unique}

\subsection{Resumen de productos y servicios}%{Products and services summary}

\startitemize
\item Custom-made quality software systems
\item IT consultancy services
\stopitemize

\subsection{Características de productos y servicios clave}
%{Key product and service features}

\startitemize
\item Quality, reliability, flexibility, short delivery time
\stopitemize

\subsection{Cliente objetivo}%{Target customers}

\subsection{Beneficios clave para el cliente}%{Key customer benefits}

\section{Capacidades de la empresa}%{Business capabilities}

\Comment{Competitive advantages}

\subsection{I+D}%{R+D}

\startitemize
\item Design and develop new products and services
\stopitemize

\subsection{Operaciones}%{Operations}

\startitemize
\item Produce products and provide services
\item Location (Madrid, near your customers)
\item Equipment (PC, Servers, Software)
\item Labor (design, implement, test, deliver, support)
\item Process (account contacts, product/service definition meetings,
  production, customer support)
\item Suppliers
\item Manufacturing
\item Quality control
\stopitemize

\subsection{Marketing}%{Marketing}

\startitemize
\item Present product and services
\item Advertising and promotions
\item Public relations
\item Sales force management
\item Customer contact management
\stopitemize

\subsection{Logística y distribución}%{Distributions and delivery}

\startitemize
\item Hand products and services
\item Distributor relationshps
\item Delivery systems
\item Invetory management
\stopitemize

\subsection{Servicio al cliente}%{Customer service}

\startitemize
\item Products and services support
\item Customer support
\item Service fulfillment
\stopitemize

\subsection{Gestión de recursos}%{Management}

\startitemize
\item Team, direction and leadership
\item Business planning
\item Goal setting
\item Market analysis
\item Strategy and tactics
\item Investor relations
\item Budgeting
\stopitemize

\subsection{Organización}%{Organization}

\startitemize
\item Resources structure
\item One person organization with outsorcing some activities (organigram)
\item Organizational model: pack
\item Pack, form follows function, divide and conquer, matrix
\item Responsibilites and procedures
\stopitemize

\subsection{Finanzas}%{Financial condition}

\startitemize
\item Asset control
\item Cash flow tracking
\item Customer builling
\item Accounts payable and receivable
\item Payroll management
\item Financial reporting
\item Tax accounting
\stopitemize

\chapter{Estrategia de empresa}%{Company strategy}

\section{Análisis DAFO}%{SWOT analysis}

\subsection{Industria, cliente, competencia}%{Industry, customer, competition}

\Comment{Opportunities and threats}

\subsection{Productos y servicios, capacidades de la empresa}
%{Products and services, business capabilities}

\Comment{Strengths and weaknesses}

\startitemize
\item Importance to business, company rate
\item Capitalize on strengths for opportunities, improve weaknesses, monitor
threats, eliminate weaknesses on threats
\stopitemize

\section{Modelo de negocio}%{Business model}

\Comment{How and when to make money, financial projection and timeline}

\startitemize
\item Revenue = cost structure (fixed, variable) + profit margin
\item Timeline
\item Financial projection: profit margin = revenue - costs
\stopitemize

\section{Estrategia de crecimiento}%{Growth strategy}

\startitemize
\item New clients for existing products and services
\item New products and services for existing clients
\item New clients and new products and services
\stopitemize

\chapter{Plan de marketing}%{Marketing plan}

\section{Situación de mercado}%{Market situation}

Muchas veces las empresas compran el software que es demasiado complejo y
costoso para la operativa y las necesidades que habitualmente suelen tener. El
mercado de software está dividido en 5 segmentos principales. Los segmentos de
mercado de software se distinguen por el nivel de complejidad y configuración de
sofware, las opciones funcionales y las opciones de integración con otros
sistemas. Mientras más complejo y más funcionalidad tiene el software más caro y
más costoso es desplegar y configurarlo. Por el contrario en el otro lado de
mercado se encuentras los segmentos con el software menos complejo con los
paquetes preconfigurados que satisfacen las necesidades de típicas empresas de
segmento. La funcionalidad y el número de usuarios son limitados lo cual se
rejleja en el precio de software.

\rightaligned{\Source{ref:software-market-overview}}
\placefigure[here][fig:software-market]{Situación de mercado}
  {\externalfigure[software-market-tier-chart.jpg][width=0.7\textwidth]}

\Paragraph{Enterprise software}

El software de este segmento de mercado es muy complejo y tiene muchas
prestaciones porque pertime ajustarse a cualquier necesidad de las grandes
empresas. El software es dificil de desplegar y configurar. Las empresas que
utilizan el software de esta categoría pertenecen a Fortune 500. Son las
empresas multinacionales con las estructuras matriciales. El software permite
configuraciónes locales y regionales que aseguran el cumplimiento de las
políticas vigentes en cada region. Sistemas de esta categoría ofrecen las
interfaces de conexión a cualquier otro sistema. Los principales proveedores en
este segmento de mercado son Oracle y SAP.

\Paragraph{Upper market software}

Los productos de este segmento de mercado son menos complejos en comaración con
el anterior pero todavía son bastante dificiles para desplegar y configurar. El
software es menos costoso. Las empresas que utilizan este software son muy
grandes pero no llegan al tamaño y complejidad de procesos de las empresas del
segmento anterior. Los principales proveedores en este segmento de mercado son
Infor, Microsoft Dynamics AX y muchos otros.

\Paragraph{Mid market software}

En este segmento de mercado compiten muchos proveedores ya que el número de
empresas que utilizan el software de esta categoría super el número de empresas
en los dos anteriores segmentos de mercado. El software es versatil y ajustable
a las necesidades diversas de las empresas pero la escala es menor en
comparación con los dos primero segmentos de mercado. En este segmento de
mercado se ofrecen los paquetes preconfigurados de fácil desplegue lo cual
permite reducir los costes de software. Los principales proveedores en este
segmento de mercado son Microsoft Dynamics NAV, Sage, Infor, SAP Business One.

\Paragraph{Lower market software}

Las empresas de este segmento de mercado han podido superar los límites del
último segmento de mercado pero sus dimenciones y complejidad no llegan el la
escala del segmento anterior. En este segmento de mercado se precisan las
soluciones escalables con relativamente copa configuración y bajo precio. Los
principales proveedores en este segmento de mercado son Sage MAS y Microsoft
Dynamics GP.

\Paragraph{Shrink wrap software market}

El software de este segmento de mercado no permite mucha configuración y muchos
usuarios en el sistem. Son versiones limitadas en su funcionalidad con el
objetivo de reducir los costes. Los desplegues son fáciles y no permiten mucha
flexibilidad a la hora de satisfacer las necesidades de las empresas. Sin
embargo este software es lo que precisan las pequeñas empresas. Los principales
proveedores de este segmento de mercado son Sage Simply Accouning e Intuit.

\subsection{Tendencias de mercado}%{Market trends}

\subsection{Perfil de cliente}%{Customer profile}

\subsection{Segmentación de clientes}%{Customer segmentation}

\subsection{Competencia}%{Competition}

\section{Metas y objetivos de marketing}%{Marketing goals and objectives}

\section{Posicionamiento}%{Positioning statement}

\Comment{Your business name + what makes your buisness unique and different
  + your market description}

\startitemize
\item My comppany provides custom-made quality IT solutions and consultancy
  to B2B customers
\item Business IT services company
\stopitemize

\section{Estrategias de marketing}%{Marketing strategies}

\Comment{Marketing mix}

\subsection{Estrategias de producto}%{Product strategies}

\startitemize
\item New uses for existing products
\item New products
\stopitemize

\subsection{Estrategias de precio}%{Pricing strategies}

\startitemize
\item Costs, value, profit
\item Positioning, competition
\stopitemize

\subsection{Estrategias de distribución}%{Distribution strategies}

\startitemize
\item Distribution and delivery system
\item Distribution channels
\item Packaging
\stopitemize

\subsection{Estrategias de promoción}%{Promotion strategies}

\Comment{Interest}

\startitemize
\item Marketing communication
\stopitemize

\subsection{Programa de ventas}%{Sales program}

\Comment{Purchases}

\subsection{Servicio al cliente}%{Customer service}

\Comment{Satisfaction}

\startitemize
\item Reward frequent or large purchases
\item Enhance service quality
\stopitemize

\section{Presupuesto de marketing}%{Marketing budget}

\startitemize
\item Zero-based budget
\stopitemize

\chapter{Finanzas}%{Financial review}

\section{Asignación de precio}%{Pricing}

Hourly rate = (earnings + costs + profit) / (40h * 50w)

Project rate = project hours * hourly rate

\section{Cuenta de resultados previsional}%{Income statement projection}

\startlines
{\bf Gross revenue}
- Goods costs (direct costs)
= {\bf Gross profit}
- Operating expenses (indirect costs)
- Depreciation expenses (amortización)
= {\bf Operating profit}, EBIT (Earnings Before Interest and Taxes)
+ Interest income
- Interest expenses
= {\bf Profit before taxes}
- Taxes (impuestos)
= {\bf Net profit} (over a period)
\stoplines

\section{Balance de situación previsional}%{Balance sheet projection}

\Comment{Accrual basis accounting, own/owe}

\startlines
{\bf Assets} (activos, all)
\hspace[medium]Current assets (liquidity within a year, {\it liquidity} indicator)
\hspace[big] Cash
\hspace[big]Investment portfolio
\hspace[big]Accounts receivable
\hspace[big]Inventories
\hspace[big]Prepaid expenses
\hspace[medium]Fixed assets (liquidity more then a year)
\hspace[big]Buildings
\hspace[big]Equipment
\hspace[big]Accumulated depreciation (value loss over years)
\hspace[medium]Intangibles (Licences, patents, goodwill)
= {\bf Liabilities} (pasivos, owe, financial instruments)
\hspace[medium]Current liabilities (payable within a year)
\hspace[big]Accounts payable
\hspace[big]Accrued expenses payable
\hspace[medium]Long-term liabilities (payable more then a year)
\hspace[big]
+ {\bf Equity} (patrimonio neto, own, {\it solvency} indicator) (in a moment)
\hspace[medium]Invested capital (capital social, investors)
\hspace[medium]Accumulated retained earnings = Net profit - dividends
\stoplines

\startitemize
\item Working capital (fondo de maniobra) = Current assets - Current
liabilities
\stopitemize

\section{Estado de flujos de tesoreria previsional}
%{Cash-flow statement projection}

\Comment{cash basis accounting, collect/spend, not a second before}

\startlines
{\bf Total funds in} (inflow, cobros, after paid)
\hspace[medium]Receipts on sales
\hspace[medium]Interest income
\hspace[medium]Invested capital
- {\bf Total funds out} (outflow, pagos, after paid)
\hspace[medium]Cost of goods acquired
\hspace[medium]Overhead expenses
\hspace[medium]Interest expenses
\hspace[medium]Taxes
\hspace[medium]Buildings and equipment
\hspace[medium]Long-term debt reduction
\hspace[medium]Dividends
= {\bf Net change in cash position (external)/changes in liquid assets (internal)} (over a period)
\hspace[medium]Cash
\hspace[medium]Investment portfolio
\stoplines

\startitemize
\item Operation + investing + financing = net cash flow (over a period)
\stopitemize

\section{Presupuesto}%{Master budget}

\Comment{Future}

\chapter{Plan de acción}%{Action plan}

\startitemize
\item Actions (goals, objectives, strategies)
\item Responsabilities
\item Timetable (priorities)
\item Control
\stopitemize

\chapter{Plan de contingencia}%{Contingency plan}

\Comment{What if ...?}

\stopbodymatter

\chapter{Apendices}%{Appendices}

\section{Análisis de mercado}%{Market analysis}

\section{Descripción de tecnología}%{Technology overview}

\section{Especificación de producto}%{Product specification}

\chapter{Referencias}%{References}

\useURL[url:18-trends-in-app-dev]
  [http://www.billchamberlin.com/top-18-trends-in-application-software-development-for-2014/][]
  [Bill Chamberlin \endash\ Top 18 Trends in Application Software Development for 2014]
\useURL[url:5-it-industry-trends]
  [http://www.goabacus.com/index.php?page=5-growing-it-industry-trends-and-developments][]
  [ABACUS IT \endash\ 5 Growing IT Industry Trends \& Developments]
\useURL[url:3-software-dev-trends]
  [http://www.cybercoders.com/insights/3-big-software-development-trends-to-watch-in-2014/][]
  [Cyber Coders \endash\ 3 Big Software Development Trends to Watch in 2014]
\useURL[url:software-market-overview]
  [http://www.softresources.com/resource-room/software-market-overview/][]
  [Soft Resources \endash\ Software Market Overview]

\startitemize[n]
\item[ref:18-trends-in-app-dev] \from[url:18-trends-in-app-dev]
\item[ref:5-it-industry-trends] \from[url:5-it-industry-trends]
\item[ref:3-software-dev-trends] \from[url:3-software-dev-trends]
\item[ref:software-market-overview] \from[url:software-market-overview]
\stopitemize

\stoptext
