\environment layout

% - business environment (industry, customers, competition)
%   - company overview (mission, vision, values, goals, objectives)
%     - company strategy (SWOT, business model, growth strategy)
%       - company description (products, business capabilities)
%       - marketing plan
%       - financial review
%       - action plan
%       - contingency plan
%       - executive summary

%\usemodule[fnt-10]
%\ShowCompleteFont{name:fontawesome}{20pt}{1}

\starttext

\startfrontmatter

\midaligned{Escuela de Administración de Empresas}\blank[14*big]
\midaligned{Plan de empresa}\blank[2*big]
\midaligned{Volodymyr Prokopyuk}
\midaligned{volodymyrprokopyuk@gmail}
\vfill\midaligned{Madrid}
\midaligned{\currentdate}

\completecontent
\stopfrontmatter

\startbodymatter

\chapter{Resumen ejecutivo}%{Executive summary}

\Comment{Key points, clear, brief}

\section{Elevator speech}

\startitemize
\item What business you are in
\item What is your industry and target customer
\item What are your products and services (features)
\item What is your customer value and benefits
\item What sets your business apart (uniqueness)
\item What is your competition
\stopitemize

\section{Modelo de negocio}%{Business model}

\section{Estrategia de crecimiento}%{Growth strategy}

\chapter{Breve descripcón de la empresa}%{Company overview}

\section{Idea de negocio}%{Business idea}

\Comment{Business model summary}

Custom-made quality IT solutions and consultancy

\section{Misión}%{Mission}

\Comment{Who you are, what you do}

Constant innovation in custom-made quality IT solutions and consultancy

\section{Visión}%{Vision}

\Comment{Precise set of words that defines business highest aspiration you
hope to achieve, enduring purpose of the business}

Create client value through innovation and IT technology

\section{Valores}{Values}

\Comment{Principles to guide your business activities, preferable mode of
conduct}

\startitemize
\item Client value creation
\item Commitment to customer satisfaction
\item Commitment to innovation and excellence
\item Commitment to quality and continuous improvement
\item Passion for what I do
\item Build open and honest relationships
\item Commitment to sustainability and respansibility
\stopitemize

\section{Metas}%{Goals}

\Comment{Where to go and when to get there}

\startitemize
\item Constant innovation \AwRArrowLong\
Use ultimate technology in the solutions
\item Custom-made IT solutions \AwRArrowLong\
Create broad and loyal clientele
\item Quality IT solutions \AwRArrowLong\
Establish quality assurance processes for the solutions
\item Custom-made quality consultancy \AwRArrowLong\
Increase IT consultancy service awareness
\stopitemize

\section{Objetivos}%{Objectives}

\Comment{How to get there, specific steps to reach your goals, what you must
do and when}

\startitemize
\item Use ultimate technology in the solutions, decrease time to market,
reduce operational costs (innovation goal - new ways to improve)
  \startitemize
  \item Select the most appropriate new technology for each solution
  \item Apply effectively selected techonology in each solution
  \item Increase solution quality and reduce solution delivery time by 10\%
  \stopitemize
\item Create broad and loyal clientele, find new markets/clients (day-to-day
work - increase every day effectiveness)
  \startitemize
  \item Organize 20 meetings with potential clients
  \item Send 100 emails with business offerings to potential clients
  \item Get 5 new clients
  \stopitemize
\item Establish quality assurance processes for the solutions, improve
customer satisfaction (development goal - skills, experience)
  \startitemize
  \item Select the most appropriate testing framework
  \item Apply selected testing framework in each solution
  \item Increase solution quality by 10\%
  \stopitemize
\item Establish IT consultancy service awareness, expand product/service
lines, generate new sorces of revenue (problem-solving goal - address
specific challenges)
  \startitemize
  \item Offer IT consultancy services to potential clients in each meeting
  \item Include in all business offerings emails information about IT
  consultancy services
  \item Get 3 new IT consultancy projects (added value to the solution)
  \stopitemize
\stopitemize

\chapter{Entorno}%{Business environment}

\Comment{Opportunities, threats}

\section{Industria}%{Industry}

\startitemize
\item Information technology (IT) industry
\stopitemize

\subsection{Tendencias}%{Trends}

\Comment{Economy, technologies, regulations}

\Paragraph{Virtualización/IaaS (Infrastructure as a Service)}

En los últimos años han tenido mucho éxito las tecnologías de virtualización de
hardware para ofrecer la infraestructura y la plataforma de ejecución para las
aplicaciones. La virtualización permite reducir costes de la infraestructura
mediante devisión y reparto de las prestaciones de un servidor físico entre
multiples servidores virtuales. La tecnologia ofrece los medios de asignación de
las prestaciones y de las prioridades de servicio entres los servidores
virtuales.

La tecnología de virtualización mejora la disponibilidad y la fiabilidad de la
infraestructura de hardware ofreciendo posibilidad de realojar en tiempo real
una máquina virtual de un servidor físico donde se ha encontrado un problema a
otro servidor físico disponible sin interrupción de servicio al cliente.

La virtualización facilita la forma de aumento incremental de pontencia
necesaria de cálculo añadiendo o quitando las máquinas virtuales bajo demanda
para satisfacer las necesidades de la carga de sistema en cada momento. El
enfoque de aumento incremental permite reducir el número de máquinas virtuales
en los momentos de carga baja y aumentorlo solo cuando haya necesidad.

La última tencologia de virtualización hace un paso más y virtualiza no los
sistemas operativos sino las aplicaciones concretas. Esta tecnología se llama
contenerización y es aún más eficiente en comparación con la virtualización
tradicional. Contenerización virtualiza y aisla las aplicaciones dentro del
mismo servidor virtual y aprovecha al máximo los recursos de servidor.

Tanto virtualizacion como contenerización presentan un interes grande para las
empresas porque permiten reducir los costes de la infraestructura de sistemas de
información, mejorar la disponibilidad y la fiabilidad de sistemas, escalar las
prestaciones segun demanda.

\Paragraph{Nube/SaaS (Software as a Service)}

La nube presenta una alternativa muy atractiva al modelo tradocional de mentener
un centro de proceso de datos por parte de la empresa. El modelo de la nube
supone prestación de servicio tanto de hardware como de software por parte de
proveedor. El cliente cuenta con un servidor fiable, disponible y seguro
funcionando 24/7.

Las tecnologías de virtualización han hecho posible el servicio en la
nube. Habitualmente un servidor en la nube tiene menor coste en comapración con
el mantenimiento de un servidor físico por parte de la empresa. La nube es una
opción barata y fiable para las empresas pequeñas y medianas ya que permite
externalizar los servicios de infraestructura de hardware, la red de
comunicaciones y la administración de sistemas operativos centrandose en las
tareas de propio negocio.

Los proveedores de servicios en la nube han echo un paso más y ofrecen a los
clientes no solo los servidores virtuales aptos para servir cualquier tipo de
aplicaciones sino las aplicaciones en sí ya desplegadas y configuradas para el
cliente concreto. Esta estrategia de externalizar el mantenimiento y la
administración de las aplicaciones minimiza los costes de infraestructura de
software para las empresas.

La selección de las aplicaciones disponibles en la nube satisface las
necesidades del ciclo completo de funcionamiento de una empresa desde el
análisis del entorno y la fijación de los objetivos, la elaboración de las
estrategias, la planificación, la ejecución de los planes hasta el control y la
toma de medidas correctivas. Las aplicaciones más comunes ofrecidos en la nube
son las siguientes:

\startitemize
\item ERP (Enterprise Resource Planning)
\item PM (Project Management)
\item CRM (Customer Relationship Management)
\item e-Commerce (Comercio electrónico)
\item BI (Business Intelligence)
\item HR (Human Resources)
\stopitemize

\Paragraph{Software de código abierto}

El papel de software de código abierto sigue creciendo en la infraestructura de
sistemas de información tanto en las empresas pequeñas y medidas como en las
grandes. El software de código abierto ofrece muchas ventajas tanto para los
proveedores como para los consumidores de los productos de software.

Para los consumidores la ventaja principal consiste en reducción considerable de
costes de licencias en comparación con la adquisición del software propietario.
Por otro lado el software de código abierto es más fiable y más seguro porque lo
revisan y corrigen conginuamente muchos ingenieros de sofware altamente
cualifacados no por dinero sino por vocación y su buena voluntad a contribuir a
proyectos ambiciosos y grandes. Otra ventaja que aprovechan los consumidores de
software de código abierto es el soporte del producto y resolución de
incidencias. La pregunta o una incidencia puede ser resuelta en pocos días por
los profecionales de todo el mundo que tienen ganas de ayudar y explicar los
detalles de funcionamiento de su producto.

Para los proveedores de software de código abierto las ventajas consisten en la
oportunidad de influir en la evolución de productos y contribuir a la innovación
de tecnologias de informaicón. Muchas veces su trabajo es esponsorizado por las
grandes empresas que tienen interes en el desarrollo de la tecnologia concreta.
A veces las grandes empresas contribuyen los módulos a los proyectos de código
abierto y de esta manera promocionan las tecnologias de código abierto.

Últimamente muchos proyectos con código cerrado han sido publicados para ganar
el mercado y mejorar su posición frente a la competencia. El modelo de código
abierto ha demostrado su viabilidad y sostenibilidad en las tecnologías de
informaicón y seguirá creciendo.

\Paragraph{Dispositivos móviles}

Cada vez más accesos a Internet se hacen desde los dispositivos móviles. En 2014
el trafico total de acceso a Internet generado por los teléfonos móviles y los
táblets ha superado el trafico total de acceso a internet generado por los PCs.

Cada vez más los empresarios y los hombres de negocio prefieren un teléfono
móvil o un táblet a un portatil o PC. La nube, las comunicaciones de alta
velocidad y las redes sociales contribuyen al proceso de transformación en los
dispositivos de acceso a Internet de portatil/PC a móvil/táblet.

Para una empresa es muy importante estar presente en el ámbito de dispositivos
móviles con el fin de ofrecer el mejor servicio a sus clientes a traves de
todos los canales de acceso a los servicios. En el desarrollo de las
aplicaciones hay que hacer el enfoque en las plataformas de dispositivos
móviles debido a su aceptación y uso universal.

\Paragraph{Adopción de HTML5 y CSS3}

Desde hace unos años tiene lugar el dilema con que se enfrentan muchas empresas
que ofrecen servicios en la Web tanto para los navegadores de PCs como para los
dispositivos móviles. El dilema consiste en que para ofrecer las mejores
experiencias para los usuarios de los dispositivos móviles hay que desarrollar
las aplicaciones nativas para todas las plataformas que se necesita soportar
(iOS, Androdid). Este enfoque supote un coste elevado para las empresas de
desarrollar y mantener la aplicación en varias plataformas. Otro enconveniente
de este método es la falta de consistencia de contenido y de experiencia al
usuario debido a las diferencias en las plataformas de dispositivos móviles.
Otra alternativa supote utilización de una única tecnología para servir la
aplicación en todas las paltaformas. En este caso se utiliza la tecnologia
HTML5 con CSS3 y JavaScript. El inconventiente que anteriormente dominaba en
este ámbito era la experiencia visual limitada que podía ofrecer HTML5 en
aquel momento.

La ventaja de HTML5 con CSS3 consiste en que esta tecnologia es un estandar que
soportan todos los navegadores. HTML5 con CSS3 está en continuo desarrollo
aplicando los métodos más innovadores y probados en la industria de tecnologia
de información. HTML5 asegura la consistencia en todas las plataformas de
dispositivos móviles. Las empresas tienen que desarrollar y mantener solo una
única aplicación que serve todo el mercado. Últimamente las experiencias
visuales que ofrece HTML5 se comparan con las experiencias visuales de las
aplicaciones nativas lo cual nivela ambas tecnologias en cuanto y las
experiencias visuales. Mientras que la ventaja de una única aplicaciones para
todo el mercado apunta a favor de HTML5.

Un argumento más a favor de HTML5 con CSS3 y JavaScript radica en que Facebook
ha desarrollado un framework para desarrollo de aplicaciones nativas para iOS y
Android utilizando HTML5 y JavaScript. La tecnología se llama React.js Native.
La publicación del proyecto como software de código abierto demuestra un alto
interes en promoción de HTML5 por parte de las grandes compañías como la
tecnología para el desarrollo de las aplicaciones para dispositivos móviles.

\Paragraph{Modelo asíncrono de ejecución en el servidor/cliente}

\subsection{Fuerzas competitivas básicas}%{Porter forces}

\subsection{Barreras de entrada}%{Entry barriers}

\startitemize
\item Capital cost
\item Economy of scale
\item Customer loyalty
\item Distribution system
\item Organiztion
\item Raw materials
\item New technology
\item Regulations
\item Patents
\stopitemize

\section{Clientes}%{Customers}

\Comment{B2B}

\subsection{Perfil de cliente}%{Customer profile}

\startitemize
\item Geographics (where they are)
\item Demographics (who they are) (age, income, marital status, education level)
\item Behavioral (how/why they buy)
\stopitemize

Customers:
\startitemize
\item Small and medium businesses of any kind
\stopitemize

Needs:
\startitemize
\item Benefit from IT advantages for business
\item Automate production and administration processes
\item Control production and administration processes
\item Innovate production and administration processes
\stopitemize

\subsection{Segmentación de clientes}%{Customer segmentation}

\subsection{Cliente ideal}%{Ideal customer}

\section{Competencia}%{Competition}

\startitemize
\item Understand competition
\item Distinguish yourself (advantage over competition)
\item Unique product/service (quality, reliability, flexibility, short
delivery time)
\stopitemize

\chapter{Descripción de la empresa}%{Company description}

\Comment{Strengths, weaknesses}

\section{Productos y servicios}%{Products and services}

\Comment{Unique}

\subsection{Resumen de productos y servicios}%{Products and services summary}

\startitemize
\item Custom-made quality software systems
\item IT consultancy services
\stopitemize

\subsection{Características de productos y servicios clave}
%{Key product and service features}

\startitemize
\item Quality, reliability, flexibility, short delivery time
\stopitemize

\subsection{Cliente objetivo}%{Target customers}

\subsection{Beneficios clave para el cliente}%{Key customer benefits}

\section{Capacidades de la empresa}%{Business capabilities}

\Comment{Competitive advantages}

\subsection{I+D}%{R+D}

\startitemize
\item Design and develop new products and services
\stopitemize

\subsection{Operaciones}%{Operations}

\startitemize
\item Produce products and provide services
\item Location (Madrid, near your customers)
\item Equipment (PC, Servers, Software)
\item Labor (design, implement, test, deliver, support)
\item Process (account contacts, product/service definition meetings,
  production, customer support)
\item Suppliers
\item Manufacturing
\item Quality control
\stopitemize

\subsection{Marketing}%{Marketing}

\startitemize
\item Present product and services
\item Advertising and promotions
\item Public relations
\item Sales force management
\item Customer contact management
\stopitemize

\subsection{Logística y distribución}%{Distributions and delivery}

\startitemize
\item Hand products and services
\item Distributor relationshps
\item Delivery systems
\item Invetory management
\stopitemize

\subsection{Servicio al cliente}%{Customer service}

\startitemize
\item Products and services support
\item Customer support
\item Service fulfillment
\stopitemize

\subsection{Gestión de recursos}%{Management}

\startitemize
\item Team, direction and leadership
\item Business planning
\item Goal setting
\item Market analysis
\item Strategy and tactics
\item Investor relations
\item Budgeting
\stopitemize

\subsection{Organización}%{Organization}

\startitemize
\item Resources structure
\item One person organization with outsorcing some activities (organigram)
\item Organizational model: pack
\item Pack, form follows function, divide and conquer, matrix
\item Responsibilites and procedures
\stopitemize

\subsection{Finanzas}%{Financial condition}

\startitemize
\item Asset control
\item Cash flow tracking
\item Customer builling
\item Accounts payable and receivable
\item Payroll management
\item Financial reporting
\item Tax accounting
\stopitemize

\chapter{Estrategia de empresa}%{Company strategy}

\section{Análisis DAFO}%{SWOT analysis}

\subsection{Industria, cliente, competencia}%{Industry, customer, competition}

\Comment{Opportunities and threats}

\subsection{Productos y servicios, capacidades de la empresa}
%{Products and services, business capabilities}

\Comment{Strengths and weaknesses}

\startitemize
\item Importance to business, company rate
\item Capitalize on strengths for opportunities, improve weaknesses, monitor
threats, eliminate weaknesses on threats
\stopitemize

\section{Modelo de negocio}%{Business model}

\Comment{How and when to make money, financial projection and timeline}

\startitemize
\item Revenue = cost structure (fixed, variable) + profit margin
\item Timeline
\item Financial projection: profit margin = revenue - costs
\stopitemize

\section{Estrategia de crecimiento}%{Growth strategy}

\startitemize
\item New clients for existing products and services
\item New products and services for existing clients
\item New clients and new products and services
\stopitemize

\chapter{Plan de marketing}%{Marketing plan}

\section{Situación de mercado}%{Market situation}

\subsection{Tendencias de mercado}%{Market trends}

\subsection{Perfil de cliente}%{Customer profile}

\subsection{Segmentación de clientes}%{Customer segmentation}

\subsection{Competencia}%{Competition}

\section{Metas y objetivos de marketing}%{Marketing goals and objectives}

\section{Posicionamiento}%{Positioning statement}

\Comment{Your business name + what makes your buisness unique and different
  + your market description}

\startitemize
\item My comppany provides custom-made quality IT solutions and consultancy
  to B2B customers
\item Business IT services company
\stopitemize

\section{Estrategias de marketing}%{Marketing strategies}

\Comment{Marketing mix}

\subsection{Estrategias de producto}%{Product strategies}

\startitemize
\item New uses for existing products
\item New products
\stopitemize

\subsection{Estrategias de precio}%{Pricing strategies}

\startitemize
\item Costs, value, profit
\item Positioning, competition
\stopitemize

\subsection{Estrategias de distribución}%{Distribution strategies}

\startitemize
\item Distribution and delivery system
\item Distribution channels
\item Packaging
\stopitemize

\subsection{Estrategias de promoción}%{Promotion strategies}

\Comment{Interest}

\startitemize
\item Marketing communication
\stopitemize

\subsection{Programa de ventas}%{Sales program}

\Comment{Purchases}

\subsection{Servicio al cliente}%{Customer service}

\Comment{Satisfaction}

\startitemize
\item Reward frequent or large purchases
\item Enhance service quality
\stopitemize

\section{Presupuesto de marketing}%{Marketing budget}

\startitemize
\item Zero-based budget
\stopitemize

\chapter{Finanzas}%{Financial review}

\section{Asignación de precio}%{Pricing}

Hourly rate = (earnings + costs + profit) / (40h * 50w)

Project rate = project hours * hourly rate

\section{Cuenta de resultados previsional}%{Income statement projection}

\startlines
{\bf Gross revenue}
- Goods costs (direct costs)
= {\bf Gross profit}
- Operating expenses (indirect costs)
- Depreciation expenses (amortización)
= {\bf Operating profit}, EBIT (Earnings Before Interest and Taxes)
+ Interest income
- Interest expenses
= {\bf Profit before taxes}
- Taxes (impuestos)
= {\bf Net profit} (over a period)
\stoplines

\section{Balance de situación previsional}%{Balance sheet projection}

\Comment{Accrual basis accounting, own/owe}

\startlines
{\bf Assets} (activos, all)
\hspace[medium]Current assets (liquidity within a year, {\it liquidity} indicator)
\hspace[big] Cash
\hspace[big]Investment portfolio
\hspace[big]Accounts receivable
\hspace[big]Inventories
\hspace[big]Prepaid expenses
\hspace[medium]Fixed assets (liquidity more then a year)
\hspace[big]Buildings
\hspace[big]Equipment
\hspace[big]Accumulated depreciation (value loss over years)
\hspace[medium]Intangibles (Licences, patents, goodwill)
= {\bf Liabilities} (pasivos, owe, financial instruments)
\hspace[medium]Current liabilities (payable within a year)
\hspace[big]Accounts payable
\hspace[big]Accrued expenses payable
\hspace[medium]Long-term liabilities (payable more then a year)
\hspace[big]
+ {\bf Equity} (patrimonio neto, own, {\it solvency} indicator) (in a moment)
\hspace[medium]Invested capital (capital social, investors)
\hspace[medium]Accumulated retained earnings = Net profit - dividends
\stoplines

\startitemize
\item Working capital (fondo de maniobra) = Current assets - Current
liabilities
\stopitemize

\section{Estado de flujos de tesoreria previsional}
%{Cash-flow statement projection}

\Comment{cash basis accounting, collect/spend, not a second before}

\startlines
{\bf Total funds in} (inflow, cobros, after paid)
\hspace[medium]Receipts on sales
\hspace[medium]Interest income
\hspace[medium]Invested capital
- {\bf Total funds out} (outflow, pagos, after paid)
\hspace[medium]Cost of goods acquired
\hspace[medium]Overhead expenses
\hspace[medium]Interest expenses
\hspace[medium]Taxes
\hspace[medium]Buildings and equipment
\hspace[medium]Long-term debt reduction
\hspace[medium]Dividends
= {\bf Net change in cash position (external)/changes in liquid assets (internal)} (over a period)
\hspace[medium]Cash
\hspace[medium]Investment portfolio
\stoplines

\startitemize
\item Operation + investing + financing = net cash flow (over a period)
\stopitemize

\section{Presupuesto}%{Master budget}

\Comment{Future}

\chapter{Plan de acción}%{Action plan}

\startitemize
\item Actions (goals, objectives, strategies)
\item Responsabilities
\item Timetable (priorities)
\item Control
\stopitemize

\chapter{Plan de contingencia}%{Contingency plan}

\Comment{What if ...?}

\stopbodymatter

\chapter{Apendices}%{Appendices}

\section{Análisis de mercado}%{Market analysis}

\section{Descripción de tecnología}%{Technology overview}

\section{Especificación de producto}%{Product specification}

\stoptext
